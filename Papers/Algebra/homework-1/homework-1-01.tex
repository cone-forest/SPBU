\documentclass[a4paper, 12pt]{article}
%----------------------------------------------------------------------------------------
%	PACKAGES AND OTHER DOCUMENT CONFIGURATIONS
%----------------------------------------------------------------------------------------
\usepackage[a4paper, total={7in, 10in}]{geometry}
\setlength{\parskip}{0pt}
\setlength{\parindent}{0in}

\usepackage[T2A]{fontenc}% Внутренняя T2A кодировка TeX
\usepackage[utf8]{inputenc}% кодировка файла
\usepackage[russian]{babel}% поддержка переносов в русском языке
\usepackage{amsthm, amsmath, amssymb} % Mathematical typesetting
\usepackage{float} % Improved interface for floating objects
\usepackage{graphicx, multicol} % Enhanced support for graphics
\usepackage{xcolor} % Driver-independent color extensions
\usepackage{mdframed}

\usepackage[yyyymmdd]{datetime} % Uses YEAR-MONTH-DAY format for dates
\renewcommand{\dateseparator}{.} % Sets dateseparator to '.'

\usepackage{fancyhdr} % Headers and footers
\pagestyle{fancy} % All pages have headers and footers
\fancyhead{}\renewcommand{\headrulewidth}{0pt} % Blank out the default header
\fancyfoot[L]{} % Custom footer text
\fancyfoot[C]{} % Custom footer text
\fancyfoot[R]{\thepage} % Custom footer text

\newenvironment{problem}[2][Задача]
    { \begin{mdframed}[backgroundcolor=gray!10] \textbf{#1 #2.} \\}
    {  \end{mdframed}}

\newenvironment{solution}
    {\textit{Решение: }}
    {\noindent\rule{7in}{1.5pt}}

\begin{document}

%-------------------------------
%	TITLE SECTION
%-------------------------------

\fancyhead[C]{}
\hrule \medskip % Upper rule
\begin{minipage}{0.295\textwidth}
\raggedright\footnotesize
Михаил Цуканов \hfill\\
st117303 \hfill\\
st117303@student.spbu.ru
\end{minipage}
\begin{minipage}{0.4\textwidth}
\centering\large
Homework Assignment 1\\
\normalsize
Алгебра и геометрия, 1 семестр\\
\end{minipage}
\begin{minipage}{0.295\textwidth}
\raggedleft
\today\hfill\\
\end{minipage}
\medskip\hrule
\bigskip

%------------------------------------------------
%	CONTENTS
%------------------------------------------------

%%%%%%%%%%%%%%%%%%%%%%%%%%%%%%%%%%%%%%%%%%%%%%%%%%%%
% Задача 1
\begin{problem}{102}
Вычислить $(x-1-i)(x-1+i)(x+1+i)(x+1-i)$.
\end{problem}
\begin{solution}
%%%%%%%%%%%%%%%%%%%%%%%%%%%%%%%%%%%%%%%%%%%%%%%%%%%%
%% Ваше решение задачи здесь

$
(x - (1 + i))(x - (1 - i))(x + (1 + i))(x + (1 - i)) = \\
(x - (1 + i))(x + (1 + i))(x + (1 - i))(x - (1 - i)) = \\
(x^2 - (1 + i)^2)(x^2 - (1 - i)^2) = \\
(x^2 - 2i)(x^2 + 2i) = \\
(x^4 + 4) \\
$
%%%%%%%%%%%%%%%%%%%%%%%%%%%%%%%%%%%%%%%%%%%%%%%%%%%%
\end{solution}


%%%%%%%%%%%%%%%%%%%%%%%%%%%%%%%%%%%%%%%%%%%%%%%%%%%%
% Задача 2
\begin{problem}{105(c)}
Вычислить $(1+2i)^5-(1-2i)^5$.
\end{problem}
\begin{solution}
%%%%%%%%%%%%%%%%%%%%%%%%%%%%%%%%%%%%%%%%%%%%%%%%%%%%
%% Ваше решение задачи здесь

$
(1 + 2i)^5 - (1 - 2i)^5 = \\
(1 + 2i)^5 - (1 - 2i)^5 = \\
(1 + 5*2i + 10*-4 + 10*-8i + 5*16 + 1*32i) - (1 + 5*-2i + 10*-4 + 10*8i + 5*16 + 1*-32i) = \\
(1 + 10i - 40 - 80i + 80 + 32i) - (1 - 10i - 40 + 80i + 80 - 32i) = \\
(41 - 38i) - (41 + 38i) = \\
-76i \\
$
%%%%%%%%%%%%%%%%%%%%%%%%%%%%%%%%%%%%%%%%%%%%%%%%%%%%
\end{solution}


%%%%%%%%%%%%%%%%%%%%%%%%%%%%%%%%%%%%%%%%%%%%%%%%%%%%
% Задача 3
\begin{problem}{107(d)}
Вычислить $\displaystyle\frac{(1-i)^5-1}{(1+i)^5+1}$.
\end{problem}
\begin{solution}
%%%%%%%%%%%%%%%%%%%%%%%%%%%%%%%%%%%%%%%%%%%%%%%%%%%%
%% Ваше решение задачи здесь

$
\displaystyle\frac{(1 - i)^5 - 1}{(1 + i)^5 + 1} = \\
$
\\
По треугольнику паскаля, как и в прошлом задании
\\
$
\displaystyle\frac{(-4 + 4i) - 1}{(-4 - 4i) + 1} =
\displaystyle\frac{-5 + 4i}{-3 - 4i} =
\displaystyle\frac{5 - 4i}{3 + 4i} =
\displaystyle\frac{(5 - 4i)(3 - 4i)}{25} =
\displaystyle\frac{((4 - 4i) + 1)((4 - 4i) - 1)}{25} = \\
\displaystyle\frac{16(1 - i)^2 - 1}{25} =
\displaystyle\frac{16 * -2i - 1}{25} =
\displaystyle\frac{-32i - 1}{25} =
-\displaystyle\frac{1 + 32i}{25} =
-\displaystyle\frac{1}{25} - \displaystyle\frac{32i}{25}
$

%%%%%%%%%%%%%%%%%%%%%%%%%%%%%%%%%%%%%%%%%%%%%%%%%%%%
\end{solution}


%%%%%%%%%%%%%%%%%%%%%%%%%%%%%%%%%%%%%%%%%%%%%%%%%%%%
% Задача 4
\begin{problem}{108(b)}
Решить систему уравнений
$$
\left\{
\begin{aligned}
  (2+i)x+(2-i)y &=6,\\
  (3+2i)x+(3-2i)y &=8.
\end{aligned}
\right.
$$
\end{problem}
\begin{solution}
%%%%%%%%%%%%%%%%%%%%%%%%%%%%%%%%%%%%%%%%%%%%%%%%%%%%
%% Ваше решение задачи здесь

  $
  \left\{
    \begin{aligned}
      (2 + i)x + (2 - i)y &= 6, | * (3 + 2i) \\
      (3 + 2i)x + (3 - 2i)y &= 8, | * (2 + i)
    \end{aligned}
  \right\}
  $
\\

Вычитаем из первого второе

$
  ((2 - i)(3 + 2i) - (3 - 2i)(2 + i))y = -2 \\
  ((8 + i) - (8 - i))y = -2 \\
  2iy = -2 \\
  y = i \\
\left\{
\begin{aligned}
  (2 + i)x + (2 - i)y &= 6 \\
  y &= i
\end{aligned}
\right.
\left\{
\begin{aligned}
  (2 + i)x + 2i + 1 &= 6 \\
  y &= i
\end{aligned}
\right.
\\
(2 + i)x = 5 - 2i \\
x = \displaystyle\frac{5 - 2i}{2 + i} = \displaystyle\frac{(5 - 2i)(2 - i)}{5} = \displaystyle\frac{8 - 9i}{5} = \displaystyle\frac{8}{5} - \displaystyle\frac{9i}{5}
\\
$

Итого \\

$
\left\{
\begin{aligned}
  x &= \displaystyle\frac{8}{5} - \displaystyle\frac{9i}{5} \\
  y &= i
\end{aligned}
\right.
$

%%%%%%%%%%%%%%%%%%%%%%%%%%%%%%%%%%%%%%%%%%%%%%%%%%%%
\end{solution}


%%%%%%%%%%%%%%%%%%%%%%%%%%%%%%%%%%%%%%%%%%%%%%%%%%%%
% Задача 5
\begin{problem}{112(b, g)}
Вычислить: b) $\sqrt{-8i}$, g) $\sqrt{2-3i}$.
\end{problem}
\begin{solution}
%%%%%%%%%%%%%%%%%%%%%%%%%%%%%%%%%%%%%%%%%%%%%%%%%%%%
%% Ваше решение задачи здесь

b) \\
$
i = (a + bi)^2 \leftrightarrow \\
a = \pm \frac{\sqrt2}{2} \\
b = \pm \frac{\sqrt2}{2} \\
$(знаки a и b совпадают)$ \\
\\
\sqrt{-8i} = 2i\sqrt{2i} = 2i\sqrt2\sqrt{i} =
\pm 2i\sqrt2 (\frac{\sqrt2}{2} + i\frac{\sqrt2}{2}) =
\pm 2i (1 + i) = \pm (-2 + 2i)
$

g) \\
$
2 - 3i = (a + bi)^2 \leftrightarrow \\
\left\{
  \begin{array}{rr}
    2 = a^2 - b^2 \\
    -3 = 2ab
  \end{array}
\right.
\left\{
  \begin{array}{rr}
    a = \frac{-3}{2b} \\
    b^2 - \frac{9}{4b^2} + 2 = 0
  \end{array}
\right.
\left\{
  \begin{array}{rr}
    a = \frac{-3}{2b} \\
    b^4 - \frac{9}{4} + 2b^2 = 0
  \end{array}
\right.
\left\{
  \begin{array}{rr}
    a = \frac{-3}{2b} \\
    4b^4 + 8b^2 - 9 = 0
  \end{array}
\right.
\\
b^2 = \frac{-8 \pm \sqrt{64 + 144}}{-8} =
1 \pm \frac{\sqrt{208}}{8} = 1 \pm \frac{\sqrt{13}}{2} \\
a^2 = b^2 + 2 = 3 \pm \frac{\sqrt{13}}{2} \\
$
Итого: \\
$
\sqrt{2 - 3i} =
\pm(\sqrt{3 \pm \frac{\sqrt{13}}{2}} + i\sqrt{1 \pm \frac{\sqrt{13}}{2}}) \\
$

%%%%%%%%%%%%%%%%%%%%%%%%%%%%%%%%%%%%%%%%%%%%%%%%%%%%
\end{solution}


%%%%%%%%%%%%%%%%%%%%%%%%%%%%%%%%%%%%%%%%%%%%%%%%%%%%
% Задача 6
\begin{problem}{113(b)}
Решить уравнение $x^2-(3-2i)x+(5-5i)=0$.
\end{problem}
\begin{solution}
%%%%%%%%%%%%%%%%%%%%%%%%%%%%%%%%%%%%%%%%%%%%%%%%%%%%
%% Ваше решение задачи здесь

$
x^2 - (3 - 2i)x + (5 - 5i) = 0 \\
x = \frac{-b \pm \sqrt{b^2 - 4ac}}{2a} \\
x = \frac{-(3 - 2i) \pm \sqrt{(3 - 2i)^2 - 4(1)(5 - 5i)}}{2(1)} \\
$
\\
Значение под корнем: \\
$
(3 - 2i)^2 - 4(1)(5 - 5i) \\
= (3 - 2i)(3 - 2i) - 20 + 20i \\
= 9 - 6i - 6i + 4i^2 - 20 + 20i \\
= 9 - 12i + 4(-1) - 20 + 20i \\
= -7 + 8i \\
x = \frac{-(3 - 2i) \pm \sqrt{-7 + 8i}}{2} \\
$
Найдем, чему равен $\sqrt{-7 + 8i}$: \\
$
\sqrt{a + bi} = \pm \sqrt{\frac{\sqrt{a^2 + b^2} + a}{2}} + \frac{\text{sgn}(b) \sqrt{\sqrt{a^2 + b^2} - a}}{2}i \\
-7 + 8i \\
a = -7 \\
b = 8 \\
\sqrt{-7 + 8i} = \pm \sqrt{\frac{\sqrt{(-7)^2 + 8^2} - 7}{2}} + \frac{\sqrt{\sqrt{(-7)^2 + 8^2} + 7}}{2}i \\
\sqrt{-7 + 8i} = \pm \sqrt{\frac{\sqrt{49 + 64} - 7}{2}} + \frac{\sqrt{\sqrt{49 + 64} + 7}}{2}i \\
\sqrt{-7 + 8i} = \pm \sqrt{\frac{\sqrt{113} - 7}{2}} + \frac{\sqrt{\sqrt{113} + 7}}{2}i \\
$
Итого:
$
\pm \sqrt{\frac{\sqrt{113} - 7}{2}} + \frac{\sqrt{\sqrt{113} + 7}}{2}i \\
$
Подставим это значение в полученное ранее равенство \\
Итого: \\
$
x =
\frac{
-(3 - 2i)
\pm
\sqrt{\frac{\sqrt{113} - 7}{2}} \pm -\frac{\sqrt{\sqrt{113} + 7}}{2}i \\
}
{2} \\
$
Ответ: \\
$
x =
\frac{
-3 \pm \sqrt{\frac{\sqrt{113} - 7}{2}}
}{
  2
}
+
i\frac{
  2 \pm - \frac{\sqrt{\sqrt{113} + 7}}{2}
}{
  2
}
$

%%%%%%%%%%%%%%%%%%%%%%%%%%%%%%%%%%%%%%%%%%%%%%%%%%%%
\end{solution}

%------------------------------------------------
\end{document}
