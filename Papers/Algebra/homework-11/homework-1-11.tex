\documentclass[a4paper, 12pt]{article}
%----------------------------------------------------------------------------------------
%	PACKAGES AND OTHER DOCUMENT CONFIGURATIONS
%----------------------------------------------------------------------------------------
\usepackage[a4paper, total={7in, 10in}]{geometry}
\setlength{\parskip}{0pt}
\setlength{\parindent}{0in}

\usepackage[T2A]{fontenc}% Внутренняя T2A кодировка TeX
\usepackage[utf8]{inputenc}% кодировка файла
\usepackage[russian]{babel}% поддержка переносов в русском языке
\usepackage{amsthm, amsmath, amssymb} % Mathematical typesetting
\usepackage{float} % Improved interface for floating objects
\usepackage{graphicx, multicol} % Enhanced support for graphics
\usepackage{xcolor} % Driver-independent color extensions
\usepackage{mdframed}

\usepackage[yyyymmdd]{datetime} % Uses YEAR-MONTH-DAY format for dates
\renewcommand{\dateseparator}{.} % Sets dateseparator to '.'

\usepackage{fancyhdr} % Headers and footers
\pagestyle{fancy} % All pages have headers and footers
\fancyhead{}\renewcommand{\headrulewidth}{0pt} % Blank out the default header
\fancyfoot[L]{} % Custom footer text
\fancyfoot[C]{} % Custom footer text
\fancyfoot[R]{\thepage} % Custom footer text

\newenvironment{problem}[2][Задача]
    { \begin{mdframed}[backgroundcolor=gray!10] \textbf{#1 #2.} \\}
    {  \end{mdframed}}

\newenvironment{solution}
    {\textit{Решение: }}
    {\noindent\rule{7in}{1.5pt}}

\begin{document}

%-------------------------------
%	TITLE SECTION
%-------------------------------

\fancyhead[C]{}
\hrule \medskip % Upper rule
\begin{minipage}{0.295\textwidth}
\raggedright\footnotesize
Михаил Цуканов \hfill\\
st117303 \hfill\\
st117303@student.spbu.ru
\end{minipage}
\begin{minipage}{0.4\textwidth}
\centering\large
Homework Assignment 11\\
\normalsize
Алгебра и геометрия, 1 семестр\\
\end{minipage}
\begin{minipage}{0.295\textwidth}
\raggedleft
\today\hfill\\
\end{minipage}
\medskip\hrule
\bigskip

%------------------------------------------------
%	CONTENTS
%------------------------------------------------



%%%%%%%%%%%%%%%%%%%%%%%%%%%%%%%%%%%%%%%%%%%%%%%%%%%%
% Задача 1
\begin{problem}{1032(j)}
Найти собственные значения и собственные векторы матрицы
$\left(\begin{array}{rrr}2 & 5 & -6 \\ 4 & 6 & -9 \\ 3 & 6 & -8 \end{array}\right)$

\end{problem}
\begin{solution}
%%%%%%%%%%%%%%%%%%%%%%%%%%%%%%%%%%%%%%%%%%%%%%%%%%%%
%% Ваше решение задачи здесь

$
\left|
\begin{array}{rrr}
2 - \lambda & 5 & -6 \\
4 & 6 - \lambda & -9 \\
3 & 6 & -8 - \lambda
\end{array}
\right|
=
$

%%%%%%%%%%%%%%%%%%%%%%%%%%%%%%%%%%%%%%%%%%%%%%%%%%%%
\end{solution}

%%%%%%%%%%%%%%%%%%%%%%%%%%%%%%%%%%%%%%%%%%%%%%%%%%%%
% Задача 2
\begin{problem}{1032(h)}
Найти собственные значения и собственные векторы матрицы
$\left(\begin{array}{rrr}0 & 2 & 1 \\ -2 & 0 & 3 \\ -1 & -3 & 0 \end{array}\right)$

\end{problem}
\begin{solution}
%%%%%%%%%%%%%%%%%%%%%%%%%%%%%%%%%%%%%%%%%%%%%%%%%%%%
%% Ваше решение задачи здесь

$
\left|
\begin{array}{rrr}
-\lambda & 2 & 1 \\
-2 & -\lambda & 3 \\
-1 & -3 & -\lambda
\end{array}
\right|
=
(-\lambda^3 - 6 + 6) - (\lambda + 4\lambda + 9\lambda) =
-\lambda^3 - 15\lambda = 0 \\
-\lambda(\lambda^2 + 15) = 0
\\
\lambda \in \{0\}
$
Найдем собственный вектор: \\
$
\left(
\begin{array}{r}
x \\
y \\
z
\end{array}
\right)^T 
*
\left(
\begin{array}{rrr}
0 & 2 & 1 \\
-2 & 0 & 3 \\
-1 & -3 & 0
\end{array}
\right)
=
\left(
\begin{array}{r}
-2y - z \\
2x - 3z \\
x + 3y
\end{array}
\right)
=
\left(
\begin{array}{r}
0 \\
0 \\
0
\end{array}
\right)
$
Решим систему \\
$
\left(
\begin{array}{rrrr}
0 & -2 & -1 & 0 \\
2 &  0 & -3 & 0 \\
1 &  3 &  0 & 0
\end{array}
\right)
\rightarrow
\left(
\begin{array}{rrrr}
0 & -2 & -1 & 0 \\
0 & -6 & -3 & 0 \\
1 &  3 &  0 & 0
\end{array}
\right)
\rightarrow
\left(
\begin{array}{rrrr}
0 &  0 &  0 & 0 \\
0 & -2 & -1 & 0 \\
1 &  3 &  0 & 0
\end{array}
\right)
$
\\
Итого: \\
$
z
\left(
\begin{array}{rrrr}
\frac{3}{2} \\
-\frac{1}{2} \\
1
\end{array}
\right)
=
z
\left(
\begin{array}{rrrr}
3 \\
-1 \\
2
\end{array}
\right)
$

%%%%%%%%%%%%%%%%%%%%%%%%%%%%%%%%%%%%%%%%%%%%%%%%%%%%
\end{solution}

%%%%%%%%%%%%%%%%%%%%%%%%%%%%%%%%%%%%%%%%%%%%%%%%%%%%
% Задача 3
\begin{problem}{}
Найти собственные значения и собственные векторы матрицы
$\left(\begin{array}{rrrr}0 & 1 & 1 & -1 \\ 1 & 0 & -1 & 1 \\ 1 & -1 & 0 & 1 \\ -1 & 1 & 1 & 0 \end{array}\right)$

\end{problem}
\begin{solution}
%%%%%%%%%%%%%%%%%%%%%%%%%%%%%%%%%%%%%%%%%%%%%%%%%%%%
%% Ваше решение задачи здесь

$
\left(
\begin{array}{rrrr}
0 & 1 & 1 & -1 \\
1 & 0 & -1 & 1 \\
1 & -1 & 0 & 1 \\
-1 & 1 & 1 & 0
\end{array}
\right)
$

%%%%%%%%%%%%%%%%%%%%%%%%%%%%%%%%%%%%%%%%%%%%%%%%%%%%
\end{solution}

%%%%%%%%%%%%%%%%%%%%%%%%%%%%%%%%%%%%%%%%%%%%%%%%%%%%
% Задача 4
\begin{problem}{1033.b}
Найти собственные значения матрицы\\
$\left( \begin{array}{rrrrrrr}0 & 1 & 0 & \ldots & 0 & 0 & 0\\ -1 & 0 & 1 & \ldots & 0 & 0 & 0 \\ 0 & -1 & 0 & \ldots & 0 & 0 & 0 \\ \vdots & \vdots & \vdots & \ddots & \vdots & \vdots & \vdots \\ 0 & 0 & 0 & \ldots & 0 & 1 & 0 \\ 0 & 0 & 0 & \ldots & -1 & 0 & 1 \\ 0 & 0 & 0 & \ldots & 0 & -1 & 0 \end{array} \right)$

\end{problem}
\begin{solution}
%%%%%%%%%%%%%%%%%%%%%%%%%%%%%%%%%%%%%%%%%%%%%%%%%%%%
%% Ваше решение задачи здесь



%%%%%%%%%%%%%%%%%%%%%%%%%%%%%%%%%%%%%%%%%%%%%%%%%%%%
\end{solution}

%%%%%%%%%%%%%%%%%%%%%%%%%%%%%%%%%%%%%%%%%%%%%%%%%%%%
% Задача 5
\begin{problem}{1034}
Найти собственные значения матрицы\\
$\left( \begin{array}{rrrrrr}-1 & 1 & 0 & \ldots & 0 & 0 \\ 1 & 0 & 1 & \ldots & 0 & 0 \\ 0 & 1 & 0 & \ldots & 0 & 0 \\ \vdots & \vdots & \vdots & \ddots & \vdots & \vdots \\ 0 & 0 & 0 & \ldots & 0 & 1 \\ 0 & 0 & 0 & \ldots & 1 & 0 \end{array} \right)$

\end{problem}
\begin{solution}
%%%%%%%%%%%%%%%%%%%%%%%%%%%%%%%%%%%%%%%%%%%%%%%%%%%%
%% Ваше решение задачи здесь



%%%%%%%%%%%%%%%%%%%%%%%%%%%%%%%%%%%%%%%%%%%%%%%%%%%%
\end{solution}

%%%%%%%%%%%%%%%%%%%%%%%%%%%%%%%%%%%%%%%%%%%%%%%%%%%%
% Задача 6
\begin{problem}{1035}
Найти собственные значения матрицы\\
$\left( \begin{array}{rrrrr}0 & x & x & \ldots & x \\ y & 0 & x & \ldots & x \\ y & y & 0 & \ldots & x \\ \vdots & \vdots & \vdots & \ddots & \vdots \\ y & y & y & \ldots & 0 \end{array} \right)$

\end{problem}
\begin{solution}
%%%%%%%%%%%%%%%%%%%%%%%%%%%%%%%%%%%%%%%%%%%%%%%%%%%%
%% Ваше решение задачи здесь



%%%%%%%%%%%%%%%%%%%%%%%%%%%%%%%%%%%%%%%%%%%%%%%%%%%%
\end{solution}

%------------------------------------------------
\end{document}
