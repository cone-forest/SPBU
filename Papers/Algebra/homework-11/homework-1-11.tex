\documentclass[a4paper, 12pt]{article}
%----------------------------------------------------------------------------------------
%	PACKAGES AND OTHER DOCUMENT CONFIGURATIONS
%----------------------------------------------------------------------------------------
\usepackage[a4paper, total={7in, 10in}]{geometry}
\setlength{\parskip}{0pt}
\setlength{\parindent}{0in}

\usepackage[T2A]{fontenc}% Внутренняя T2A кодировка TeX
\usepackage[utf8]{inputenc}% кодировка файла
\usepackage[russian]{babel}% поддержка переносов в русском языке
\usepackage{amsthm, amsmath, amssymb} % Mathematical typesetting
\usepackage{float} % Improved interface for floating objects
\usepackage{graphicx, multicol} % Enhanced support for graphics
\usepackage{xcolor} % Driver-independent color extensions
\usepackage{mdframed}

\usepackage[yyyymmdd]{datetime} % Uses YEAR-MONTH-DAY format for dates
\renewcommand{\dateseparator}{.} % Sets dateseparator to '.'

\usepackage{fancyhdr} % Headers and footers
\pagestyle{fancy} % All pages have headers and footers
\fancyhead{}\renewcommand{\headrulewidth}{0pt} % Blank out the default header
\fancyfoot[L]{} % Custom footer text
\fancyfoot[C]{} % Custom footer text
\fancyfoot[R]{\thepage} % Custom footer text

\newenvironment{problem}[2][Задача]
    { \begin{mdframed}[backgroundcolor=gray!10] \textbf{#1 #2.} \\}
    {  \end{mdframed}}

\newenvironment{solution}
    {\textit{Решение: }}
    {\noindent\rule{7in}{1.5pt}}

\begin{document}

%-------------------------------
%	TITLE SECTION
%-------------------------------

\fancyhead[C]{}
\hrule \medskip % Upper rule
\begin{minipage}{0.295\textwidth}
\raggedright\footnotesize
Михаил Цуканов \hfill\\
st117303 \hfill\\
st117303@student.spbu.ru
\end{minipage}
\begin{minipage}{0.4\textwidth}
\centering\large
Homework Assignment 11\\
\normalsize
Алгебра и геометрия, 1 семестр\\
\end{minipage}
\begin{minipage}{0.295\textwidth}
\raggedleft
\today\hfill\\
\end{minipage}
\medskip\hrule
\bigskip

%------------------------------------------------
%	CONTENTS
%------------------------------------------------



%%%%%%%%%%%%%%%%%%%%%%%%%%%%%%%%%%%%%%%%%%%%%%%%%%%%
% Задача 1
\begin{problem}{1032(j)}
Найти собственные значения и собственные векторы матрицы
$\left(\begin{array}{rrr}2 & 5 & -6 \\ 4 & 6 & -9 \\ 3 & 6 & -8 \end{array}\right)$

\end{problem}
\begin{solution}
%%%%%%%%%%%%%%%%%%%%%%%%%%%%%%%%%%%%%%%%%%%%%%%%%%%%
%% Ваше решение задачи здесь

  $
det(A-\lambda E) = \left|\begin{array}{rrr}2-\lambda & 5 & -6 \\ 4 & 6-\lambda & -9 \\ 3 & 6 & -8-\lambda \end{array}\right| = \left|\begin{array}{rrr}2-\lambda & 5 & -6 \\ 1 & -\lambda & \lambda-1 \\ 3 & 6 & -8-\lambda \end{array}\right| = \left|\begin{array}{rrr}2-\lambda & 5 & -1 \\ 1 & -\lambda & -1 \\ 3 & 6 & -2-\lambda \end{array}\right| = \left|\begin{array}{rrr}1-\lambda & 5+\lambda & 0 \\ 1 & -\lambda & -1 \\ 3 & 6 & -2-\lambda \end{array}\right| = (1-\lambda)\left|\begin{array}{rr}-\lambda & -1 \\ 6 & -2-\lambda \end{array}\right| - (5+\lambda)\left|\begin{array}{rr}1 & -1 \\ 3 & -2-\lambda \end{array}\right| = (1-\lambda)(\lambda^2+2\lambda+6)-(5+\lambda)(-2-\lambda+3)=1-\lambda^3
$
\\
$\displaystyle
f(\lambda) = 0 \Rightarrow \lambda = 1$
\\
$\displaystyle
\lambda = 1: \left(\begin{array}{rrr}1 & 5 & -6 \\ 4 & 5 & -9 \\ 3 & 6 & -9 \end{array}\right)X=0$
\\
$\displaystyle
\left(\begin{array}{rrr}1 & 5 & -6 \\ 4 & 5 & -9 \\ 3 & 6 & -9 \end{array}\right|\left.\begin{array}{rrr}0\\0 \\0 \end{array}\right)_{S_2-4S_1;S_3-3S_1}\Rightarrow\left(\begin{array}{rrr}1 & 5 & -6 \\ 0 & -15 & 15 \\ 0 & -9 & 9 \end{array}\right|\left.\begin{array}{rrr}0\\0 \\0 \end{array}\right)_{5S_3-3S_2}\Rightarrow\left(\begin{array}{rrr}1 & 5 & -6 \\ 0 & -15 & 15 \\ 0 & 0 & 0 \end{array}\right|\left.\begin{array}{rrr}0\\0 \\0 \end{array}\right)
$
\\
$\displaystyle
\left\{
\begin{aligned}
x_1+5x_2-6x_3=0 \\
-15x_2+15x_3=0 
\end{aligned}
\right.
\Rightarrow
\left\{
\begin{aligned}
x_1=x_3 \\
x_2=x_3 
\end{aligned}
\right.
\Rightarrow
\\
\left(\begin{array}{r}x_1\\x_2\\x_3\end{array}\right)=\left(\begin{array}{r}1\\1\\1\end{array}\right)x_3
\Rightarrow
\vartheta=\left(\begin{array}{rrr}1 & 1 & 1\end{array}\right)^{T}
$
\\
Ответ: $\displaystyle \lambda=1;\vartheta=\left(\begin{array}{rrr}1 & 1 & 1\end{array}\right)^{T}$

%%%%%%%%%%%%%%%%%%%%%%%%%%%%%%%%%%%%%%%%%%%%%%%%%%%%
\end{solution}

%%%%%%%%%%%%%%%%%%%%%%%%%%%%%%%%%%%%%%%%%%%%%%%%%%%%
% Задача 2
\begin{problem}{1032(h)}
Найти собственные значения и собственные векторы матрицы
$\left(\begin{array}{rrr}0 & 2 & 1 \\ -2 & 0 & 3 \\ -1 & -3 & 0 \end{array}\right)$

\end{problem}
\begin{solution}
%%%%%%%%%%%%%%%%%%%%%%%%%%%%%%%%%%%%%%%%%%%%%%%%%%%%
%% Ваше решение задачи здесь

$
\left|
\begin{array}{rrr}
-\lambda & 2 & 1 \\
-2 & -\lambda & 3 \\
-1 & -3 & -\lambda
\end{array}
\right|
=
(-\lambda^3 - 6 + 6) - (\lambda + 4\lambda + 9\lambda) =
-\lambda^3 - 15\lambda = 0 \\
-\lambda(\lambda^2 + 15) = 0
\\
\lambda \in \{0\}
$
Найдем собственный вектор: \\
$
\left(
\begin{array}{r}
x \\
y \\
z
\end{array}
\right)^T 
*
\left(
\begin{array}{rrr}
0 & 2 & 1 \\
-2 & 0 & 3 \\
-1 & -3 & 0
\end{array}
\right)
=
\left(
\begin{array}{r}
-2y - z \\
2x - 3z \\
x + 3y
\end{array}
\right)
=
\left(
\begin{array}{r}
0 \\
0 \\
0
\end{array}
\right)
$
Решим систему \\
$
\left(
\begin{array}{rrrr}
0 & -2 & -1 & 0 \\
2 &  0 & -3 & 0 \\
1 &  3 &  0 & 0
\end{array}
\right)
\rightarrow
\left(
\begin{array}{rrrr}
0 & -2 & -1 & 0 \\
0 & -6 & -3 & 0 \\
1 &  3 &  0 & 0
\end{array}
\right)
\rightarrow
\left(
\begin{array}{rrrr}
0 &  0 &  0 & 0 \\
0 & -2 & -1 & 0 \\
1 &  3 &  0 & 0
\end{array}
\right)
$
\\
Итого: \\
$
\lambda = 0 \\
v =
\left(
\begin{array}{rrrr}
\frac{3}{2} \\
-\frac{1}{2} \\
1
\end{array}
\right)
$

%%%%%%%%%%%%%%%%%%%%%%%%%%%%%%%%%%%%%%%%%%%%%%%%%%%%
\end{solution}

%%%%%%%%%%%%%%%%%%%%%%%%%%%%%%%%%%%%%%%%%%%%%%%%%%%%
% Задача 3
\begin{problem}{}
Найти собственные значения и собственные векторы матрицы
$\left(\begin{array}{rrrr}0 & 1 & 1 & -1 \\ 1 & 0 & -1 & 1 \\ 1 & -1 & 0 & 1 \\ -1 & 1 & 1 & 0 \end{array}\right)$

\end{problem}
\begin{solution}
%%%%%%%%%%%%%%%%%%%%%%%%%%%%%%%%%%%%%%%%%%%%%%%%%%%%
%% Ваше решение задачи здесь

$\displaystyle
det(A-\lambda E) = \left|\begin{array}{rrrr}-\lambda & 1 & 1 & -1 \\ 1 & -\lambda & -1 & 1 \\ 1 & -1 & -\lambda & 1 \\ -1 & 1 & 1 & -\lambda \end{array}\right|_{S_4+S_3}
=
\left|\begin{array}{rrrr}-\lambda & 1 & 1 & -1 \\ 1 & -\lambda & -1 & 1 \\ 1 & -1 & -\lambda & 1 \\ 0 & 0 & 1 - \lambda & 1 - \lambda \end{array}\right|
=
$
\\
$\displaystyle
(1-\lambda)(-1)^{4+3}\left|\begin{array}{rrr}-\lambda & 1 & -1 \\ 1 & -\lambda & 1 \\ 1 & -1 & 1\end{array}\right|
+
(1-\lambda)(-1)^{4+4}\left|\begin{array}{rrr}-\lambda & 1 & 1 \\ 1 & -\lambda & -1 \\ 1 & -1 & -\lambda\end{array}\right|
=
$
\\
$\displaystyle
(\lambda-1)(\lambda^2-2\lambda+1)+(1-\lambda)(-\lambda^3+3\lambda-2)
=
\lambda^4-6\lambda^2+8\lambda-3=(\lambda-1)^3(\lambda+3)
$
\\
$\displaystyle
f(\lambda) = 0
\Rightarrow
\left[
\begin{gathered}
\lambda = 1,\\
\lambda = -3
\end{gathered}
\right.
$
\\
$\displaystyle
\lambda = 1: \left(\begin{array}{rrrr}-1 & 1 & 1 & -1 \\ 1 & -1 & -1 & 1 \\ 1 & -1 & -1 & 1 \\ -1 & 1 & 1 & -1 \end{array}\right)X=0$
\\
$\displaystyle
\left(\begin{array}{rrrr}-1 & 1 & 1 & -1 \\ 1 & -1 & -1 & 1 \\ 1 & -1 & -1 & 1 \\ -1 & 1 & 1 & -1 \end{array}\right|\left.\begin{array}{r}0 \\ 0 \\ 0 \\ 0 \end{array}\right)
\Rightarrow
\left(\begin{array}{rrrr}-1 & 1 & 1 & -1 \\ 1 & -1 & -1 & 1 \\ 1 & -1 & -1 & 1 \\ -1 & 1 & 1 & -1 \end{array}\right|\left.\begin{array}{r}0 \\ 0 \\ 0 \\ 0 \end{array}\right)_{S_2+S_1;S_3+S_1;S_4-S_1}
\Rightarrow
\left(\begin{array}{rrrr}-1 & 1 & 1 & -1 \\ 0 & 0 & 0 & 0 \\ 0 & 0 & 0 & 0 \\ 0 & 0 & 0 & 0 \end{array}\right|\left.\begin{array}{r}0 \\ 0 \\ 0 \\ 0 \end{array}\right)
$
\\
$\displaystyle
-x_1+x_2+x_3-x_4=0
\Rightarrow
x_1=-x_2-x_3+x_4
$
\\
$\displaystyle
\left(\begin{array}{r}x_1\\x_2\\x_3\\x_4\end{array}\right)=\left(\begin{array}{r}-1\\1\\0\\0\end{array}\right)x_2
+
\left(\begin{array}{r}-1\\0\\1\\0\end{array}\right)x_3
+
\left(\begin{array}{r}1\\0\\0\\1\end{array}\right)x_4
$
\\
$\displaystyle
\vartheta_1=\left(\begin{array}{r}-1\\1\\0\\0\end{array}\right); \vartheta_2=\left(\begin{array}{r}-1\\0\\1\\0\end{array}\right); \vartheta_3=\left(\begin{array}{r}1\\0\\0\\1\end{array}\right)
$
\\
$\displaystyle
\lambda = -3: \left(\begin{array}{rrrr}3 & 1 & 1 & -1 \\ 1 & 3 & -1 & 1 \\ 1 & -1 & 3 & 1 \\ -1 & 1 & 1 & 3 \end{array}\right)X=0$
\\
$\displaystyle
\left(\begin{array}{rrrr}3 & 1 & 1 & -1 \\ 1 & 3 & -1 & 1 \\ 1 & -1 & 3 & 1 \\ -1 & 1 & 1 & 3 \end{array}\right|\left.\begin{array}{r}0 \\ 0 \\ 0 \\ 0 \end{array}\right)_{3S_2-S_1;3S_3-S_1;3S_4+S_1}
\Rightarrow
\left(\begin{array}{rrrr}3 & 1 & 1 & -1 \\ 0 & 8 & -6 & 6 \\ 0 & -6 & 8 & 6 \\ 0 & 4 & 4 & 8 \end{array}\right|\left.\begin{array}{r}0 \\ 0 \\ 0 \\ 0 \end{array}\right)
\Rightarrow
$
\\
$\displaystyle
\left(\begin{array}{rrrr}3 & 1 & 1 & -1 \\ 0 & 8 & -6 & 6 \\ 0 & -6 & 8 & 6 \\ 0 & 4 & 4 & 8 \end{array}\right|\left.\begin{array}{r}0 \\ 0 \\ 0 \\ 0 \end{array}\right)_{4S_3+3S_2;2S_4-S_2}
\Rightarrow
\left(\begin{array}{rrrr}3 & 1 & 1 & -1 \\ 0 & 8 & -6 & 6 \\ 0 & 0 & 14 & 42 \\ 0 & 0 & 14 & 10 \end{array}\right|\left.\begin{array}{r}0 \\ 0 \\ 0 \\ 0 \end{array}\right)_{S_4-S_3}
\Rightarrow
\left(\begin{array}{rrrr}3 & 1 & 1 & -1 \\ 0 & 8 & -6 & 6 \\ 0 & 0 & 14 & 42 \\ 0 & 0 & 0 & -32 \end{array}\right|\left.\begin{array}{r}0 \\ 0 \\ 0 \\ 0 \end{array}\right)
$
\\
$\displaystyle
\left\{
\begin{aligned}
3x_1+x_2+x_3-x_4=0 \\
8x_2-6x_3+6x_4=0 \\
14x_3+42x_4=0 \\
-32x_4=0
\end{aligned}
\right.
\Rightarrow
\left\{
\begin{aligned}
x_1=0 \\
x_2=0 \\
x_3=0 \\
x_4=0
\end{aligned}
\right.
$

$
\\
\displaystyle
\left(\begin{array}{r}x_1\\x_2\\x_3\\x_4\end{array}\right)=\left(\begin{array}{r}0\\0\\0\\0\end{array}\right)
$
\\
$\displaystyle
\vartheta_1=\left(\begin{array}{r}0\\0\\0\\0\end{array}\right)
$
\\
Ответ: $\displaystyle
\lambda_1=1; \vartheta_1=\left(\begin{array}{r}-1\\1\\0\\0\end{array}\right); \vartheta_2=\left(\begin{array}{r}-1\\0\\1\\0\end{array}\right); \vartheta_3=\left(\begin{array}{r}1\\0\\0\\1\end{array}\right);
\lambda_2=-3; \vartheta_4=\left(\begin{array}{r}0\\0\\0\\0\end{array}\right)$

%%%%%%%%%%%%%%%%%%%%%%%%%%%%%%%%%%%%%%%%%%%%%%%%%%%%
\end{solution}

%%%%%%%%%%%%%%%%%%%%%%%%%%%%%%%%%%%%%%%%%%%%%%%%%%%%
% Задача 4
\begin{problem}{1033.b}
Найти собственные значения матрицы\\
$\left( \begin{array}{rrrrrrr}0 & 1 & 0 & \ldots & 0 & 0 & 0\\ -1 & 0 & 1 & \ldots & 0 & 0 & 0 \\ 0 & -1 & 0 & \ldots & 0 & 0 & 0 \\ \vdots & \vdots & \vdots & \ddots & \vdots & \vdots & \vdots \\ 0 & 0 & 0 & \ldots & 0 & 1 & 0 \\ 0 & 0 & 0 & \ldots & -1 & 0 & 1 \\ 0 & 0 & 0 & \ldots & 0 & -1 & 0 \end{array} \right)$

\end{problem}
\begin{solution}
%%%%%%%%%%%%%%%%%%%%%%%%%%%%%%%%%%%%%%%%%%%%%%%%%%%%
%% Ваше решение задачи здесь

$\displaystyle
det(A-\lambda E)
=
\left| \begin{array}{rrrrrrr}-\lambda & 1 & 0 & \ldots & 0 & -\lambda & 0\\ -1 & 0 & 1 & \ldots & 0 & 0 & 0 \\ 0 & -1 & -\lambda & \ldots & 0 & 0 & 0 \\ \vdots & \vdots & \vdots & \ddots & \vdots & \vdots & \vdots \\ 0 & 0 & 0 & \ldots & -\lambda & 1 & 0 \\ 0 & 0 & 0 & \ldots & -1 & -\lambda & 1 \\ 0 & 0 & 0 & \ldots & 0 & -1 & -\lambda \end{array} \right|
=
-\lambda\left| \begin{array}{rrrrrrr} 0 & 1 & 0 & \ldots & 0 & 0 & 0 \\ -1 & -\lambda & 1 & \ldots & 0 & 0 & 0 \\ \vdots & \vdots & \vdots & \ddots & \vdots & \vdots & \vdots \\ 0 & 0 & 0 & \ldots & -\lambda & 1 & 0 \\ 0 & 0 & 0 & \ldots & -1 & -\lambda & 1 \\ 0 & 0 & 0 & \ldots & 0 & -1 & -\lambda \end{array} \right|
+
1(-1)^{1+1}\left| \begin{array}{rrrrrrr}-1 & 1 & 0 & \ldots & 0 & 0 & 0 \\ 0 & -\lambda & 1 & \ldots & 0 & 0 & 0 \\ \vdots & \vdots & \vdots & \ddots & \vdots & \vdots & \vdots \\ 0 & 0 & 0 & \ldots & -\lambda & 1 & 0 \\ 0 & 0 & 0 & \ldots & -1 & -\lambda & 1 \\ 0 & 0 & 0 & \ldots & 0 & -1 & -\lambda \end{array} \right|
=-\lambda\bigtriangleup_{n-1}+\bigtriangleup_{n-2}
$
\\
Получаем рекуррентную формулу: $\displaystyle
\bigtriangleup_{n}=-\lambda\bigtriangleup_{n-1}+\bigtriangleup_{n-2}
\bigtriangleup_2=-\lambda\bigtriangleup_1+\bigtriangleup_0
$
\\
$\displaystyle
\bigtriangleup_2=\lambda^2+1;\bigtriangleup_1=-\lambda$
\\
$\displaystyle
\bigtriangleup_0=1$
\\
$\displaystyle
G(z)=\frac{c_0+c_1z-\alpha c_0z}{1-\alpha z-\beta z^2}; Возьмем \alpha=-\lambda; \beta=1$
\\
$\displaystyle
G(z)=\frac{1-\lambda z+\lambda z}{1+\lambda z-z^2}$
\\
$\displaystyle
z_{1|2}=\frac{-\lambda\pm\sqrt{\lambda^2+4}}{-2}=\frac{\lambda\mp\sqrt{\lambda^2+4}}{2}$
\\
Пусть $z_{1|2}=\cos{\theta}+i\sin{\theta}$
\\
$\displaystyle
\frac{\lambda\mp\sqrt{\lambda^2+4}}{2}=\cos{\theta}+i\sin{\theta}\Rightarrow\lambda^2+4=4(\cos{2\theta}+i\sin{2\theta})-4\lambda(\cos{\theta}+i\sin{\theta})+\lambda^2\Rightarrow$
\\
$\displaystyle
1=\cos{2\theta}-\lambda\cos{\theta}\Rightarrow\lambda=\frac{\cos{2\theta}-1}{\cos{\theta}}$
\\
$\displaystyle
\frac{1}{1+\lambda z - z^2} = \frac{A}{z_1-z}+\frac{B}{z_2-z}=\frac{A(z_2-z)+B(z_1-z)}{(z_1-z)(z_2-z)}\Rightarrow$
\\
$\displaystyle
\left\{
\begin{aligned}
A(z_2-z_1)=1 \\
B(z_1-z_2)=1 
\end{aligned}
\right.
\Rightarrow
\left\{
\begin{aligned}
A=\frac{i}{2\sin{\theta}} \\
B=-\frac{i}{2\sin{\theta}} 
\end{aligned}
\right.
$
\\
$\displaystyle
\bigtriangleup_n=\frac{A}{z_1^{n+1}}+\frac{B}{z_2^{n+1}}=\frac{i}{2\sin{\theta}}\frac{1}{(\cos{\theta}+i\sin{\theta})^{n+1}}-\frac{i}{2\sin{\theta}}\frac{1}{(\cos{\theta}-i\sin{\theta})^{n+1}}
=
\frac{i}{2\sin{\theta}}\bigl(\frac{1}{(\cos{\theta}+i\sin{\theta})^{n+1}}-\frac{1}{(\cos{\theta}-i\sin{\theta})^{n+1}}\bigr)
=
\frac{i}{2\sin{\theta}}\frac{-2i\sin{\theta}(n+1)}{\cos^2{\theta}(n+1)+\sin^2{\theta}(n+1)}
=
\frac{\sin{\theta}(n+1)}{\sin{\theta}(\cos^2{\theta}(n+1)+\sin^2{\theta})}
=
\frac{\sin{\theta}(n+1)}{\sin{\theta}}
\Rightarrow
\bigtriangleup_n=\frac{\sin{\theta}(n+1)}{\sin{\theta}}
$
\\
$\displaystyle
\bigtriangleup_nX=0$
\\
$\displaystyle
fracsin(\theta)(n+1)\sin{\theta}=0$
\\
$\displaystyle
\sin{\theta}(n+1)=0$
\\
$\displaystyle
\theta(n+1)=\frac{\pi}{2}+2\pi k, k\in Z$
\\
$\displaystyle
\theta=\frac{\pi+4\pi k}{2(n+1)}, k\in Z$
\\
В итоге получаем, что $\displaystyle
\theta=\frac{\cos{\frac{\pi+4\pi k}{(n+1)}}-1}{\cos{\frac{\pi+4\pi k}{2(n+1)}}}$
\\
Ответ: $\displaystyle
\theta=\frac{\cos{\frac{\pi+4\pi k}{(n+1)}}-1}{\cos{\frac{\pi+4\pi k}{2(n+1)}}}
$

%%%%%%%%%%%%%%%%%%%%%%%%%%%%%%%%%%%%%%%%%%%%%%%%%%%%
\end{solution}

%%%%%%%%%%%%%%%%%%%%%%%%%%%%%%%%%%%%%%%%%%%%%%%%%%%%
% Задача 5
\begin{problem}{1034}
Найти собственные значения матрицы\\
$\left( \begin{array}{rrrrrr}-1 & 1 & 0 & \ldots & 0 & 0 \\ 1 & 0 & 1 & \ldots & 0 & 0 \\ 0 & 1 & 0 & \ldots & 0 & 0 \\ \vdots & \vdots & \vdots & \ddots & \vdots & \vdots \\ 0 & 0 & 0 & \ldots & 0 & 1 \\ 0 & 0 & 0 & \ldots & 1 & 0 \end{array} \right)$

\end{problem}
\begin{solution}
%%%%%%%%%%%%%%%%%%%%%%%%%%%%%%%%%%%%%%%%%%%%%%%%%%%%
%% Ваше решение задачи здесь

$
\left( 
\begin{array}{rrrrrr}
-1-\lambda & 1 & 0 & \ldots & 0 & 0 \\
1 & -\lambda & 1 & \ldots & 0 & 0 \\
0 & 1 & -\lambda & \ldots & 0 & 0 \\
\vdots & \vdots & \vdots & \ddots & \vdots & \vdots \\
0 & 0 & 0 & \ldots & -\lambda & 1 \\
0 & 0 & 0 & \ldots & 1 & -\lambda
\end{array}
\right)
=
$
$
(-1-\lambda)\left(\begin{array}{rrrrrr}-\lambda & 1 & \ldots & 0 & 0 \\1 & -\lambda & \ldots & 0 & 0 \\
\vdots & \vdots & \ddots & \vdots & \vdots \\0 & 0 & \ldots & -\lambda & 1 \\0 & 0 & \ldots & 1 & -\lambda
\end{array}\right)
-
$
\\
$
\left(\begin{array}{rrrrrr}-\lambda & 1 & \ldots & 0 & 0 \\0 & -\lambda & \ldots & 0 & 0 \\\vdots & \vdots & \ddots & \vdots & \vdots \\0 & 0 & \ldots & -\lambda & 1 \\0 & 0 & \ldots & 1 & -\lambda\end{array}\right)
=
-
\left(\begin{array}{rrrrrr}-\lambda & 1 & \ldots & 0 & 0 \\1 & -\lambda & \ldots & 0 & 0 \\
\vdots & \vdots & \ddots & \vdots & \vdots \\0 & 0 & \ldots & -\lambda & 1 \\0 & 0 & \ldots & 1 & -\lambda
\end{array}\right)
-
\left(\begin{array}{rrrrrr}-\lambda & 1 & \ldots & 0 & 0 \\1 & -\lambda & \ldots & 0 & 0 \\
\vdots & \vdots & \ddots & \vdots & \vdots \\0 & 0 & \ldots & -\lambda & 1 \\0 & 0 & \ldots & 1 & -\lambda
\end{array}\right)
$
\\
$\displaystyle
\left(\begin{array}{rrrrrr}-\lambda & 1 & \ldots & 0 & 0 \\1 & -\lambda & \ldots & 0 & 0 \\
\vdots & \vdots & \ddots & \vdots & \vdots \\0 & 0 & \ldots & -\lambda & 1 \\0 & 0 & \ldots & 1 & -\lambda
\end{array}\right) = \bigtriangleup_n$
\\
$\displaystyle
\bigtriangleup_n=-\lambda\bigtriangleup_{n-1}-\bigtriangleup_{n-2}$
\\
$\displaystyle G(z) = \frac{c_0 + c_1z - \alpha c_0z}{1 - \alpha z - \beta z^2}
\\
\alpha = -\lambda; \beta = -1
\\
c_0 = 1; c_1=-\lambda; c_2 = \lambda^2 - 1
$
\\
$\displaystyle
G(z) = \frac{1}{1 + \lambda z + z^2}
$
\\
$\displaystyle
z_{1|2} = \frac{\lambda \pm \sqrt{\lambda^2 - 4}}{2}$
\\
Пусть 
$\displaystyle
z_{1|2}=\cos{\theta}\pm i\sin{\theta}$
\\
$\displaystyle
\frac{\lambda \pm \sqrt{\lambda^2 - 4}}{2}=\cos{\theta}\pm i\sin{\theta}
\\
\lambda^2 - 4 = 4(cos2\theta + isin2\theta) + 4\lambda(cos\theta + isin\theta) + \lambda^2\\
-1 = \cos2\theta + \lambda \cos\theta\\
\lambda = \frac{-1 - cos2\theta}{cos\theta}\\
\frac{1}{1 + \lambda z + z^2} = \frac{A(z_2 - z) + B(z_1 - z)}{(z_1 - z)(z_2 - z)}
\\
\left\{
\begin{aligned}
A(z_2-z_1) = 1\\
B(z_1-z_2) = 1
\end{aligned}
\right.
;
\left\{
\begin{aligned}
A = \frac{i}{2sin\theta}\\
B = -\frac{i}{2sin\theta}
\end{aligned}
\right.
\\
\bigtriangleup_{n-1} = \frac{A}{z_1^n} + \frac{B}{z_2^n}
=
\frac{i}{2\sin{\theta}}\bigl(\frac{1}{(\cos{\theta}+i\sin{\theta})^n} - \frac{1}{(\cos{\theta}-i\sin{\theta})^n}\bigr)
=
\frac{\sin{\theta n}}{\sin{\theta}(\cos^2{\theta n} + \sin^2{\theta n})}
=
\frac{\sin{\theta n}}{\sin{\theta}}\\
$
$
\displaystyle
\bigtriangleup_{n-2} = \frac{sin\theta(n-1)}{sin\theta}
$
\\
$\displaystyle
-\frac{sin\theta n}{sin\theta} - \frac{sin\theta(n-1)}{sin\theta} = 0
\Rightarrow
-\sin{\theta n} = \sin({\theta n - \theta})
\Rightarrow
\theta = -\pi
$
\\
$\displaystyle
\lambda = \frac{-1 - cos2\theta}{cos\theta} = 2\\
$ Ответ: 2

%%%%%%%%%%%%%%%%%%%%%%%%%%%%%%%%%%%%%%%%%%%%%%%%%%%%
\end{solution}

%%%%%%%%%%%%%%%%%%%%%%%%%%%%%%%%%%%%%%%%%%%%%%%%%%%%
% Задача 6
\begin{problem}{1035}
Найти собственные значения матрицы\\
$\left( \begin{array}{rrrrr}0 & x & x & \ldots & x \\ y & 0 & x & \ldots & x \\ y & y & 0 & \ldots & x \\ \vdots & \vdots & \vdots & \ddots & \vdots \\ y & y & y & \ldots & 0 \end{array} \right)$

\end{problem}
\begin{solution}
%%%%%%%%%%%%%%%%%%%%%%%%%%%%%%%%%%%%%%%%%%%%%%%%%%%%
%% Ваше решение задачи здесь

$\displaystyle
\left( 
\begin{array}{rrrrr}
-\lambda & x & x & \ldots & x \\
y & -\lambda & x & \ldots & x \\ 
y & y & -\lambda & \ldots & x \\
\vdots & \vdots & \vdots & \ddots & \vdots \\
y & y & y & \ldots & -\lambda 
\end{array} 
\right)_{S_{i+1} - S_i, i=\overline{1, n}}
\rightarrow
\left( 
\begin{array}{rrrrr}
-\lambda & x & x & \ldots & x \\
y + \lambda & -\lambda - x & 0 & \ldots & 0 \\ 
0 & y + \lambda & -\lambda - x & \ldots & 0 \\
\vdots & \vdots & \vdots & \ddots & \vdots \\
0 & 0 & 0 & \ldots & -\lambda - x
\end{array} 
\right)
$
\\
$\displaystyle
-\lambda
\left( 
\begin{array}{rrrrr}
-\lambda - x & 0 & \ldots & 0 \\ 
y + \lambda & -\lambda - x & \ldots & 0 \\
\vdots & \vdots & \ddots & \vdots \\
0 & 0 & \ldots & -\lambda - x 
\end{array} 
\right)
-x
\left( 
\begin{array}{rrrrr}
-\lambda - x & 0 & \ldots & 0 \\ 
0 & -\lambda - x & \ldots & 0 \\
\vdots & \vdots & \ddots & \vdots \\
0 & 0 & \ldots & -\lambda - x 
\end{array} 
\right)
\ldots\\
-\lambda(-\lambda - x)^{n-1} - x(y + \lambda)(-\lambda - x)^{n-2} + \ldots$
\\
При четном n
$\displaystyle
\\
det(A-\lambda E) = -\lambda(-\lambda - x)^{n-1} - x(y + \lambda)(\lambda - x)^{n-2}
\\
-\lambda(-\lambda - x)^{n-1} - x(y + \lambda)(- \lambda - x)^{n-2} = 0
\\
\lambda_1 = \sqrt{xy}
\\
\lambda_2 = -\sqrt{xy}
$
\\
При нечетном n
\\
$
det(A-\lambda E) = -\lambda(-\lambda - x)^{n-1}
\\
-\lambda(-\lambda - x)^{n-1} = 0
\\
\lambda_1 = 0
\\
\lambda_2 = -x
$
\\
Ответ: 
\\
$1) \lambda_1 = \sqrt{xy}; \lambda_2 = -\sqrt{xy}\\ 2)\lambda_1 = 0; \lambda_2 = -x$

%%%%%%%%%%%%%%%%%%%%%%%%%%%%%%%%%%%%%%%%%%%%%%%%%%%%
\end{solution}

%------------------------------------------------
\end{document}
