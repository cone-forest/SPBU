\documentclass[a4paper, 12pt]{article} %----------------------------------------------------------------------------------------
%	PACKAGES AND OTHER DOCUMENT CONFIGURATIONS
%----------------------------------------------------------------------------------------
\usepackage[a4paper, total={7in, 10in}]{geometry}
\setlength{\parskip}{0pt}
\setlength{\parindent}{0in}

\usepackage[T2A]{fontenc}% Внутренняя T2A кодировка TeX
\usepackage[utf8]{inputenc}% кодировка файла
\usepackage[russian]{babel}% поддержка переносов в русском языке
\usepackage{amsthm, amsmath, amssymb} % Mathematical typesetting
\usepackage{float} % Improved interface for floating objects
\usepackage{graphicx, multicol} % Enhanced support for graphics
\usepackage{xcolor} % Driver-independent color extensions
\usepackage{mdframed}

\usepackage[yyyymmdd]{datetime} % Uses YEAR-MONTH-DAY format for dates
\renewcommand{\dateseparator}{.} % Sets dateseparator to '.'

\usepackage{fancyhdr} % Headers and footers
\pagestyle{fancy} % All pages have headers and footers
\fancyhead{}\renewcommand{\headrulewidth}{0pt} % Blank out the default header
\fancyfoot[L]{} % Custom footer text
\fancyfoot[C]{} % Custom footer text
\fancyfoot[R]{\thepage} % Custom footer text

\newenvironment{problem}[2][Задача]
    { \begin{mdframed}[backgroundcolor=gray!10] \textbf{#1 #2.} \\}
    {  \end{mdframed}}

\newenvironment{solution}
    {\textit{Решение: }}
    {\noindent\rule{7in}{1.5pt}}

\begin{document}

%-------------------------------
%	TITLE SECTION
%-------------------------------

\fancyhead[C]{}
\hrule \medskip % Upper rule
\begin{minipage}{0.295\textwidth}
\raggedright\footnotesize
Михаил Цуканов \hfill\\
st117303 \hfill\\
st117303@student.spbu.ru
\end{minipage}
\begin{minipage}{0.4\textwidth}
\centering\large
Homework Assignment 2\\
\normalsize
Алгебра и геометрия, 1 семестр\\
\end{minipage}
\begin{minipage}{0.295\textwidth}
\raggedleft
\today\hfill\\
\end{minipage}
\medskip\hrule
\bigskip

%------------------------------------------------
%	CONTENTS
%------------------------------------------------

%%%%%%%%%%%%%%%%%%%%%%%%%%%%%%%%%%%%%%%%%%%%%%%%%%%%
% Задача 1
\begin{problem}{120}
Представить в тригонометрической форме: $2+\sqrt{3}+i$.
\end{problem}
\begin{solution}
%%%%%%%%%%%%%%%%%%%%%%%%%%%%%%%%%%%%%%%%%%%%%%%%%%%%
%% Ваше решение задачи здесь

$
2 + \sqrt3 + i = \\
2(1 + \frac{\sqrt3}{2} + \frac{1}{2}i) = \\
$
Рассмотрим выражение в скобках как вектор. \\
Это будет сумма векторов $(1, 0)$ и $(1, \frac{\pi}{3})$. \\
Получившийся треугольник равнобедренный. Найдем его основание и угол при нем. \\
(Это и будет ответом) \\
Длина (по т. косинусов): \\
$c = \sqrt{1 + 1 - 2\cos{\frac{2\pi}{3}}} = \sqrt3$ \\
Угол: \\
$\alpha = \frac{\pi - \frac{2\pi}{3}}{2} = \frac{\pi}{6}$ \\
$
2(1 + \frac{\sqrt3}{2} + \frac{1}{2}i) =
(2\sqrt3, \frac{\pi}{6}) =
2\sqrt3(\frac{\sqrt3}{2} + i\frac{1}{2})\\
$

%%%%%%%%%%%%%%%%%%%%%%%%%%%%%%%%%%%%%%%%%%%%%%%%%%%%
\end{solution}

%%%%%%%%%%%%%%%%%%%%%%%%%%%%%%%%%%%%%%%%%%%%%%%%%%%%
% Задача 2
\begin{problem}{137(c, d)}
Вычислить, пользуясь формулой Муавра \\
c)~$\left( 1- \displaystyle\frac{\sqrt{3}-i}{2}\right)^{24}$, \\
d)~$\left( \displaystyle\frac{(-1+i\sqrt{3})^{15}}{(1-i)^{20}}\right)+\left( \displaystyle\frac{(-1-i\sqrt{3})^{15}}{(1+i)^{20}}\right)$.
\end{problem}
\begin{solution}
%%%%%%%%%%%%%%%%%%%%%%%%%%%%%%%%%%%%%%%%%%%%%%%%%%%%
%% Ваше решение задачи здесь

c)~
$
\left(1 - \displaystyle\frac{\sqrt{3} - i}{2}\right)^{24} \\
$
Рассмотрим выражение в скобках как сумму векторов $(1, 0)$ и $(-\frac{\sqrt3}{2}, \frac{1}{2})$ \\
Оба вектора имеют длинну 1, и угол между ними известен - $\frac{5\pi}{6}$. \\
Используя теорема косинусов найдем длинну суммы - $\sqrt{2 + \sqrt3}$.
Геометрически найдем угол (аргумент) суммы - $\frac{5\pi}{12}$ \\
Найдем значение выражения: \\
$
\left(1 - \displaystyle\frac{\sqrt{3} - i}{2}\right)^{24} =
(2 + \sqrt3)^{12}(\cos{\frac{5\pi}{12}*24} + isin{\frac{5\pi}{12}*24}) =
(2 + \sqrt3)^{12} \\
$
\\
d)~
$
\left(
  \displaystyle\frac{(-1 + i\sqrt{3})^{15}}{(1 - i)^{20}}
\right) +
\left(
  \displaystyle\frac{(-1 - i\sqrt{3})^{15}}{(1+i)^{20}}
\right) = \\
\left(
  \displaystyle\frac{2^{15}(\frac{-1}{2} + i\frac{\sqrt{3}}{2})^{15}}{2^{10}(\frac{1}{\sqrt2} - i\frac{1}{\sqrt2})^{20}}
\right) +
\left(
  \displaystyle\frac{2^{15}(\frac{-1}{2} - i\frac{\sqrt{3}}{2})^{15}}{2^{10}(\frac{1}{\sqrt2} + i\frac{1}{\sqrt2})^{20}}
\right) = \\
\left(
  \displaystyle\frac{2^{5}(\cos{(15*\frac{2\pi}{3})} + i\sin{(15*\frac{2\pi}{3})})}{(\cos{(20*-\frac{\pi}{4})} - i\sin{(20*-\frac{\pi}{4})})}
\right) +
\left(
  \displaystyle\frac{2^{5}(\cos{(15*-\frac{2\pi}{3})} - i\sin{(15*-\frac{2\pi}{3})})}{(\cos{(20*\frac{\pi}{4})} + i\sin{(20*\frac{\pi}{4})})}
\right) = \\
2^{5}\left(
  \displaystyle\frac{1}{-1}
\right) +
2^{5}\left(
  \displaystyle\frac{1}{-1}
\right) = \\
-2^{5} - 2^{5} = -2^6
$


%%%%%%%%%%%%%%%%%%%%%%%%%%%%%%%%%%%%%%%%%%%%%%%%%%%%
\end{solution}

%%%%%%%%%%%%%%%%%%%%%%%%%%%%%%%%%%%%%%%%%%%%%%%%%%%%
% Задача 3
\begin{problem}{143(b, e)}
Извлечь корни b)~$\sqrt[3]{2+2i}$, e)~$\sqrt[6]{-27}$.
\end{problem}
\begin{solution}
%%%%%%%%%%%%%%%%%%%%%%%%%%%%%%%%%%%%%%%%%%%%%%%%%%%%
%% Ваше решение задачи здесь

$
b)~\sqrt[3]{2 + 2i} = \sqrt[6]{8}(\cos{(\frac{\pi}{12} + \frac{2}{3}\pi k)} + i\sin{(\frac{\pi}{12} + \frac{2}{3}\pi k)}), k \in \{0, 1, 2\}
\\
e)~\sqrt[6]{-27} = \sqrt3(\cos{(\frac{-\pi}{12} + \frac{1}{6}\pi k)} + i\sin{((\frac{-\pi}{12} + \frac{1}{6}\pi k))}), k \in \{0, 1, 2, 3, 4, 5\}
$

%%%%%%%%%%%%%%%%%%%%%%%%%%%%%%%%%%%%%%%%%%%%%%%%%%%%
\end{solution}

%%%%%%%%%%%%%%%%%%%%%%%%%%%%%%%%%%%%%%%%%%%%%%%%%%%%
% Задача 4
\begin{problem}{145(c)}
Вычислить $\sqrt[6]{\displaystyle\frac{1-i}{1+i\sqrt{3}}}$.
\end{problem}
\begin{solution}
%%%%%%%%%%%%%%%%%%%%%%%%%%%%%%%%%%%%%%%%%%%%%%%%%%%%
%% Ваше решение задачи здесь

$
\sqrt[6]{\displaystyle\frac{1-i}{1+i\sqrt{3}}} =
\sqrt[6]{\frac{\sqrt2(\cos{(-\frac{\pi}{4})} + i\sin{(-\frac{\pi}{4})})}{2(\cos{\frac{\pi}{3}} + i\sin{\frac{\pi}{3}})}} =
\frac{1}{\sqrt[12]{2}}(\cos{(-\frac{7\pi}{12} + \frac{\pi k}{6})} + i\sin{(-\frac{7\pi}{12} + \frac{\pi k}{6})}) \\
k \in \{0, 1, 2, 3, 4, 5, 6, 7, 8, 9, 10, 11\}
$

%%%%%%%%%%%%%%%%%%%%%%%%%%%%%%%%%%%%%%%%%%%%%%%%%%%%
\end{solution}

%%%%%%%%%%%%%%%%%%%%%%%%%%%%%%%%%%%%%%%%%%%%%%%%%%%%
% Задача 5
\begin{problem}{125}
Доказать, что всякое комплексное число $z$, отличное от $-1$, модуль которого $1$,
может быть представлено в форме $\displaystyle z=\frac{1+it}{1-it}$, где $t$~-- вещественное число.
\end{problem}
\begin{solution}
%%%%%%%%%%%%%%%%%%%%%%%%%%%%%%%%%%%%%%%%%%%%%%%%%%%%
%% Ваше решение задачи здесь

$
\frac{1 + it}{1 - it} = \frac{(1 + it)^2}{1 + t^2} = \frac{1 - t^2 + 2ti}{1 + t^2} = \\
\frac{1 - t^2}{1 + t^2} + i\frac{2t}{1 + t^2} = \\
$
Заметим, что длина упрощенного вектора всегда равна 1. \\
Значит $\forall t\ v(t)$ пренадлежит данной окружности и при этом x и y по модулю не больше 1\\
Докажем, что множество значений $v(t)$ "заполняет" все множество точек окружности. \\
Исследуем функции $y(t)$ и $x(t)$. Обе функции непрерывны как отношения элементарных непрерывных функций (знаменатель никогда не равен 0). \\
Найдем минимум и максимум x и y: \\
$
x'(t) = \frac{-2t(1+t^2) - 2t(1-t^2)}{(1+t^2)^2} = \frac{-4t}{(1+t^2)^2} = 0 \leftrightarrow t = 0 \leftrightarrow x = 1 \land y = 0
$ \\
$
y'(t) = \frac{2(1+t^2) - 4t^2}{(1+t^2)^2} = \frac{2-2t^2}{(1+t^2)^2} = 0 \leftrightarrow t = \pm 1 \leftrightarrow (x = 0 \land y = 1) \lor (x = 0 \land y = -1)
$ \\
Пока не понятно, достигает ли $x(t)$ -1. Определим это \\
$\frac{1 - t^2}{1 + t^2} = -1$ нет решений $\rightarrow$ Не достигает. \\
При этом функция непрерывна и стремится к -1 при $t \rightarrow + \infty $. \\
При этом она ограничена сверху и снизу. Значит она принимает все значения из множества $(-1; 1]$. \\
Аналогично для для $y(t)$ получаем, что она принимает все значения из множества $[-1; 1]$ \\
Значит множество пар $(x(t), y(t))$ заполняет множество точек окружности. чтд.

%%%%%%%%%%%%%%%%%%%%%%%%%%%%%%%%%%%%%%%%%%%%%%%%%%%%
\end{solution}

%%%%%%%%%%%%%%%%%%%%%%%%%%%%%%%%%%%%%%%%%%%%%%%%%%%%
% Задача 6
\begin{problem}{182}
Найти сумму всех корней $n$-й степени ($n>1$) из $1$.
\end{problem}
\begin{solution}
%%%%%%%%%%%%%%%%%%%%%%%%%%%%%%%%%%%%%%%%%%%%%%%%%%%%
%% Ваше решение задачи здесь

Любое множество корней $n$-й степени - правильный $n$-угольник. Каждый корень - радиус-вектор к соотв. вершине. \\
Значит сумма корней - сумма таких радиус-векторов. Если бы она не равнялась 0, то при умножении ее на $(1, \frac{2\pi}{n})$, ее значение должно было бы повернуться на $\frac{2\pi}{n}$. \\
При этом все вершины переходят в друг друга, а значит и сумма не должна поменяться. Значит сумма равна 0.

%%%%%%%%%%%%%%%%%%%%%%%%%%%%%%%%%%%%%%%%%%%%%%%%%%%%
\end{solution}

%%%%%%%%%%%%%%%%%%%%%%%%%%%%%%%%%%%%%%%%%%%%%%%%%%%%
% Задача 7
\begin{problem}{183}
Вычислить $1+2\varepsilon+3\varepsilon^2+...+n\varepsilon^{n-1}$, где $\varepsilon$ -- корень $n$-й степени из $1$.
\end{problem}
\begin{solution}
%%%%%%%%%%%%%%%%%%%%%%%%%%%%%%%%%%%%%%%%%%%%%%%%%%%%
%% Ваше решение задачи здесь

$
1 + 2\varepsilon + 3\varepsilon^2 +...+ n\varepsilon^{n-1} = \\
\frac{(1 + 2\varepsilon + 3\varepsilon^2 +...+ n\varepsilon^{n-1})(\varepsilon - 1)}{(\varepsilon - 1)} = \\
\frac{n - 1 -\varepsilon - \varepsilon^2 - ... - \varepsilon }{(\varepsilon - 1)} = \\
\frac{n - 1}{(\varepsilon - 1)} = \\
$

%%%%%%%%%%%%%%%%%%%%%%%%%%%%%%%%%%%%%%%%%%%%%%%%%%%%
\end{solution}

%%%%%%%%%%%%%%%%%%%%%%%%%%%%%%%%%%%%%%%%%%%%%%%%%%%%
% Задача 8
\begin{problem}{146}
Выразить $\cos 5x$ через $\cos x$ и $\sin x$
\end{problem}
\begin{solution}
%%%%%%%%%%%%%%%%%%%%%%%%%%%%%%%%%%%%%%%%%%%%%%%%%%%%
%% Ваше решение задачи здесь

$
\cos 5x = \cos{(2x + 3x)} = \cos2x * \cos3x - \sin2x * \sin3x = \\
(\cos^2x - \sin^2x)(4\cos^3x - 3\cos x) - 2\sin x \cos x(3\sin x - 4\sin^3 x) = \\
4\cos^5 x - 3\cos^3 x - 4\sin^2x\cos^3x - 3\cos x \sin^2 x + 6\sin^4x\cos x
$

%%%%%%%%%%%%%%%%%%%%%%%%%%%%%%%%%%%%%%%%%%%%%%%%%%%%
\end{solution}

%------------------------------------------------
\end{document}
