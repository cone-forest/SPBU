\documentclass[a4paper, 12pt]{article}
%----------------------------------------------------------------------------------------
%	PACKAGES AND OTHER DOCUMENT CONFIGURATIONS
%----------------------------------------------------------------------------------------
\usepackage[a4paper, total={7in, 10in}]{geometry}
\setlength{\parskip}{0pt}
\setlength{\parindent}{0in}

\usepackage[T2A]{fontenc}% Внутренняя T2A кодировка TeX
\usepackage[utf8]{inputenc}% кодировка файла
\usepackage[russian]{babel}% поддержка переносов в русском языке
\usepackage{amsthm, amsmath, amssymb} % Mathematical typesetting
\usepackage{float} % Improved interface for floating objects
\usepackage{graphicx, multicol} % Enhanced support for graphics
\usepackage{xcolor} % Driver-independent color extensions
\usepackage{mdframed}

\usepackage[yyyymmdd]{datetime} % Uses YEAR-MONTH-DAY format for dates
\renewcommand{\dateseparator}{.} % Sets dateseparator to '.'

\usepackage{fancyhdr} % Headers and footers
\pagestyle{fancy} % All pages have headers and footers
\fancyhead{}\renewcommand{\headrulewidth}{0pt} % Blank out the default header
\fancyfoot[L]{} % Custom footer text
\fancyfoot[C]{} % Custom footer text
\fancyfoot[R]{\thepage} % Custom footer text

\newenvironment{problem}[2][Задача]
    { \begin{mdframed}[backgroundcolor=gray!10] \textbf{#1 #2.} \\}
    {  \end{mdframed}}

\newenvironment{solution}
    {\textit{Решение: }}
    {\noindent\rule{7in}{1.5pt}}

\begin{document}

%-------------------------------
%	TITLE SECTION
%-------------------------------

\fancyhead[C]{}
\hrule \medskip % Upper rule
\begin{minipage}{0.295\textwidth}
\raggedright\footnotesize
Цуканов Михаил \hfill\\
st117303 \hfill\\
st117303@student.spbu.ru
\end{minipage}
\begin{minipage}{0.4\textwidth}
\centering\large
Homework Assignment 13\\
\normalsize
Алгебра и геометрия, 1 семестр\\
\end{minipage}
\begin{minipage}{0.295\textwidth}
\raggedleft
\today\hfill\\
\end{minipage}
\medskip\hrule
\bigskip

%------------------------------------------------
%	CONTENTS
%------------------------------------------------



%%%%%%%%%%%%%%%%%%%%%%%%%%%%%%%%%%%%%%%%%%%%%%%%%%%%
% Задача 1
\begin{problem}{31}
Даны вектора $\vec{a}=\{1,2,3\}$, $\vec{b}=\{2,-2,1\}$, $\vec{c}=\{4,0,3\}$, $\vec{d}=\{16,10,18\}$.
Найти вектор, являющийся проекцией вектора $\vec d$ на плоскость, определяемую векторами
$\vec a$ и $\vec b$ при направлении проектирования, параллельном вектору $\vec c$.
\end{problem}
\begin{solution}
%%%%%%%%%%%%%%%%%%%%%%%%%%%%%%%%%%%%%%%%%%%%%%%%%%%%
%% Ваше решение задачи здесь

Плоскость $(\overrightarrow{a}, \overrightarrow{b})$ определяется нормалью. \\
$
\overrightarrow{n} =
\overrightarrow{a} \times \overrightarrow{b} =
\left|
  \begin{array}{rrr}
    \overrightarrow{i} & \overrightarrow{j} & \overrightarrow{k} \\
    1 &  2 & 3 \\
    2 & -2 & 1
  \end{array}
\right|
= \{8, 5, -6\}
$
\\ Уравнение плоскости: $\overrightarrow{p} \cdot \overrightarrow{n} = 0$ \\
Расстояние от начала координат до плоскости равно 0, и вектор d начинается в начале координат.
Значит нужно спроектировать на плоскость только саму точку (16, 10, 18).
Проекция будет совпадать с точкой пересечения прямой, параллельной $\overrightarrow{c}$ и проходящей через $\overrightarrow{d}$
\\ Уравнение прямой: $\overrightarrow{p} = \overrightarrow{d} + \alpha * \overrightarrow{c}$

Значит уравнение точки пересечения:
$(\overrightarrow{d} + \alpha * \overrightarrow{c}) \cdot \overrightarrow{n} = 0$ \\
$\alpha = -\frac{\overrightarrow{d} \cdot \overrightarrow{n}}{\overrightarrow{c} \cdot \overrightarrow{n}}$ \\
$\alpha = -\frac{16 * 8 + 5 * 10 - 6 * 18}{4 * 8 + 0 * 5 - 3 * 6} = -5$ \\

Значит спроектированная точка:
$
\overrightarrow{p} =
\overrightarrow{d} + -5\overrightarrow{c} =
\{16 - 5 * 4, 10 - 5 * 0, 18 - 5 * 3\} = \{-4, 10, 3\}
$ \\
Ответ: $\{-4, 10, 3\}$


%%%%%%%%%%%%%%%%%%%%%%%%%%%%%%%%%%%%%%%%%%%%%%%%%%%%
\end{solution}

%%%%%%%%%%%%%%%%%%%%%%%%%%%%%%%%%%%%%%%%%%%%%%%%%%%%
% Задача 2
\begin{problem}{138}
Доказать, что при любом расположении точек $A$, $B$, $C$, $D$ на плоскости или в пространстве имеет место равенство
$(\overrightarrow{BC},\overrightarrow{AD})+(\overrightarrow{CA},\overrightarrow{BD})+(\overrightarrow{AB},\overrightarrow{CD})=0$.
\end{problem}
\begin{solution}
%%%%%%%%%%%%%%%%%%%%%%%%%%%%%%%%%%%%%%%%%%%%%%%%%%%%
%% Ваше решение задачи здесь

Расскроем векторы так, чтобы остались только отношения радиус векторов, если взять $A$ за системы центр координат. \\
$
(\overrightarrow{BC},\overrightarrow{AD}) +
(\overrightarrow{CA},\overrightarrow{BD}) +
(\overrightarrow{AB},\overrightarrow{CD}) = 0
<=> \\
(\overrightarrow{AC}+\overrightarrow{BA},\overrightarrow{AD}) +
(\overrightarrow{CA},\overrightarrow{BA} + \overrightarrow{AD}) +
(\overrightarrow{AB},\overrightarrow{CA} + \overrightarrow{AD}) = 0
<=> \\
$
// В силу ассоциативности скалярного произведения \\
$
(\overrightarrow{AC}, \overrightarrow{AD}) + (\overrightarrow{BA}, \overrightarrow{AD}) +
(\overrightarrow{CA}, \overrightarrow{BA}) + (\overrightarrow{CA}, \overrightarrow{AD}) +
(\overrightarrow{AB}, \overrightarrow{CA}) + (\overrightarrow{AB}, \overrightarrow{AD}) = 0 \\
$
Первое и четвертое, второе и пятое, третье и шестое слагаемые взаимно обратные. \\
Значит сумма каждой пары - 0, значит сумма пар - 0. ч.т.д.

%%%%%%%%%%%%%%%%%%%%%%%%%%%%%%%%%%%%%%%%%%%%%%%%%%%%
\end{solution}

%%%%%%%%%%%%%%%%%%%%%%%%%%%%%%%%%%%%%%%%%%%%%%%%%%%%
% Задача 3
\begin{problem}{142}
Даны два неколлинеарных вектора $\vec a$ и $\vec b$. Найти вектор $\vec x$, компланарный векторам $\vec a$ и $\vec b$ и
удовлетворяющий системе уравнений $(\vec{a},\vec{x})=1$, $(\vec{b},\vec{x})=0$.
\end{problem}
\begin{solution}
%%%%%%%%%%%%%%%%%%%%%%%%%%%%%%%%%%%%%%%%%%%%%%%%%%%%
%% Ваше решение задачи здесь

По условию $x$ ортоганален $b$ и компланарен $a$. То есть он ортоганален $b$ и $a \times b$. \\
То есть, если $\overrightarrow{l}=\overrightarrow{a} \times \overrightarrow{b} \times \overrightarrow{b}$, то $\overrightarrow{x} = \alpha * \overrightarrow{l}$. Найдем $\alpha$ \\
$l=ab^2\sin{\hat{(\overrightarrow{a}, \overrightarrow{b})}}$, при этом
$\alpha al\cos{\hat{(\overrightarrow{a}, \overrightarrow{l})}} = 1$. То есть
$\alpha a^2 b^2 \sin^2{\hat{(\overrightarrow{a}, \overrightarrow{b})}} = 1$. \\
Значит
$\alpha = \frac{1}{a^2 b^2 \sin^2{\hat{(\overrightarrow{a}, \overrightarrow{b})}}}$
\\
\\
Итого ответ: $\overrightarrow{x} = \frac{\overrightarrow{a} \times \overrightarrow{b} \times \overrightarrow{b}}{a^2 b^2 \sin^2{\hat{(\overrightarrow{a}, \overrightarrow{b})}}}$

%%%%%%%%%%%%%%%%%%%%%%%%%%%%%%%%%%%%%%%%%%%%%%%%%%%%
\end{solution}

%%%%%%%%%%%%%%%%%%%%%%%%%%%%%%%%%%%%%%%%%%%%%%%%%%%%
% Задача 4
\begin{problem}{145}
Даны два вектора $\vec a$ и $\vec n$. Найти вектор $\vec b$, являющийся ортогональной проекцией вектора $\vec a$
на плоскость, перпендикулярную вектору $\vec n$.
\end{problem}
\begin{solution}
%%%%%%%%%%%%%%%%%%%%%%%%%%%%%%%%%%%%%%%%%%%%%%%%%%%%
%% Ваше решение задачи здесь

Векторы $\overrightarrow{a}, \overrightarrow{n}, \overrightarrow{b}$ лежат в одной плоскости (по т. о трех перпендикулярах).
Значит если
$\overrightarrow{l}=\overrightarrow{a}\times\overrightarrow{n}\times\overrightarrow{n}$,
то $\overrightarrow{l}$ тоже лежит в этой плоскости
и при этом так же лежит на плоскости, перп. $\overrightarrow{n}$. \\
Это значит, что $\overrightarrow{l}$ коллинеарен $\overrightarrow{b}$. То есть $\overrightarrow{b} = \alpha * \overrightarrow{l}$.
При этом $b=a\cos{\hat{((\overrightarrow{n}, 0), \overrightarrow{a})}}=
\frac{\overrightarrow{a} \cdot \overrightarrow{l}}{l}$ \\
Значит $\alpha = \frac{\overrightarrow{a} \cdot \overrightarrow{l}}{l^2}$
\\
Тогда ответ: $\overrightarrow{b}=\frac{\overrightarrow{a} \cdot (\overrightarrow{a}\times\overrightarrow{n}\times\overrightarrow{n})}{|\overrightarrow{a}\times\overrightarrow{n}\times\overrightarrow{n}|^2} * \overrightarrow{a}\times\overrightarrow{n}\times\overrightarrow{n}$

%%%%%%%%%%%%%%%%%%%%%%%%%%%%%%%%%%%%%%%%%%%%%%%%%%%%
\end{solution}

%%%%%%%%%%%%%%%%%%%%%%%%%%%%%%%%%%%%%%%%%%%%%%%%%%%%
% Задача 5
\begin{problem}{151}
Даны два вектора $\vec{a}=\{8,4,1\}$ и $\vec{b}=\{2,-2,1\}$. Найти вектор $\vec c$, компланарный векторам $\vec a$ и $\vec b$,
перпендикулярный к вектору $\vec a$, равный ему по длине и образующий с вектором $\vec b$ тупой угол.
\end{problem}
\begin{solution}
%%%%%%%%%%%%%%%%%%%%%%%%%%%%%%%%%%%%%%%%%%%%%%%%%%%%
%% Ваше решение задачи здесь

Нужно найти вектор, перпендикулярный $\vec{a} \times \vec{b}$ и $\vec{a}$. \\
$\vec{l} = (\vec{b} \times \vec{a}) \times \vec{a}$ подходит под оба условия. \\
$l = a^2b\sin{\hat{(\overrightarrow{b}, \overrightarrow{a})}}$, при этом требуется вектор с длиной $a$. \\
значит возьмем вектор $\vec{p}=\frac{1}{ab\sin{\hat{(\overrightarrow{b}, \overrightarrow{a})}}}\vec{l}$ \\
Чтобы угол между $\vec{p}$ и $\vec{b}$ был тупым, нужно, чтобы они находились в разных полупространствах относительно плоскости, заданной $\vec{b} \times \vec{a}$ и $\vec{a}$.
Это значит, что если $\vec{c} = \vec{b} \times \vec{a}$, то тройки $(\vec{c}, \vec{a}, \vec{p})$ и $(\vec{c}, \vec{a}, \vec{b})$ должны быть разными (одна левой, другая - правой). \\
При этом тройка $(\vec{c}, \vec{a}, \vec{b})$ левая,
значит $(\vec{c}, \vec{a}, \vec{p})$ - правая.
То есть $\vec{l}=(\vec{b} \times \vec{a}) \times \vec{a})$
(а не $\vec{a} \times (\vec{b} \times \vec{a})$) \\

Итого ответ: $\frac{\vec{b} \times \vec{a} \times \vec{a}}{ab\sin{\hat{(\overrightarrow{b}, \overrightarrow{a})}}}$

%%%%%%%%%%%%%%%%%%%%%%%%%%%%%%%%%%%%%%%%%%%%%%%%%%%%
\end{solution}

%%%%%%%%%%%%%%%%%%%%%%%%%%%%%%%%%%%%%%%%%%%%%%%%%%%%
% Задача 6
\begin{problem}{184}
Даны три вектора $\vec{a}=\{8,4,1\}$, $\vec{b}=\{2,-2,1\}$, $\vec{c}=\{4,0,3\}$.
Найти вектор $\vec d$ длины 1, перпендикулярный к векторам $\vec a$ и $\vec b$ и направленный так, чтобы упорядоченные
тройки векторов $\vec a$, $\vec b$, $\vec c$ и  $\vec a$, $\vec b$, $\vec d$ имели одинаковую ориентацию.
\end{problem}
\begin{solution}
%%%%%%%%%%%%%%%%%%%%%%%%%%%%%%%%%%%%%%%%%%%%%%%%%%%%
%% Ваше решение задачи здесь

Определим ориентацию тройки $(\vec{a}, \vec{b}, \vec{c})$ \\
$
\left|
  \begin{array}{rrr}
    8 & 4 & 1 \\
    2 & -2 & 1 \\
    4 & 0 & 3
  \end{array}
\right|
= -48
$. \\
Значит тройка левая. \\
Пусть
$
\vec{l} = \vec{b} \times \vec{a}, \vec{d} = \frac{\vec{l}}{l}
$.
Тогда тройка $(\vec{a}, \vec{b}, \vec{d})$ - левая. \\
Найдем векторы $\vec{l}$ и $\vec{d}$ \\
$
\vec{l} =
\left|
  \begin{array}{rrr}
    i & j & k \\
    2 & -2 & 1 \\
    8 & 4 & 1
  \end{array}
\right|
=
\{-6, 6, 24\} = 6 * \{-1, 1, 4\} \\
l = 6 * \sqrt{1 + 1 + 16} = 18 * \sqrt{2}
$. \\
Значит $\vec{d} = \{-\frac{\sqrt{2}}{6}, \frac{\sqrt{2}}{6}, \frac{\sqrt{2}}{3}\}$ \\
Ответ: $\{-\frac{\sqrt{2}}{6}, \frac{\sqrt{2}}{6}, \frac{\sqrt{2}}{3}\}$

%%%%%%%%%%%%%%%%%%%%%%%%%%%%%%%%%%%%%%%%%%%%%%%%%%%%
\end{solution}

%%%%%%%%%%%%%%%%%%%%%%%%%%%%%%%%%%%%%%%%%%%%%%%%%%%%
% Задача 7
\begin{problem}{191}
Вычислить объем параллелепипеда, зная длины $|\overrightarrow{OA}|=a$, $|\overrightarrow{OB}|=b$, $|\overrightarrow{OC}|=c$
трёх его ребер, выходящих из одной его вершины $O$, и углы $\angle BOC=\alpha$, $\angle COB=\beta$, $\angle AOB=\gamma$ между ними.
\end{problem}
\begin{solution}
%%%%%%%%%%%%%%%%%%%%%%%%%%%%%%%%%%%%%%%%%%%%%%%%%%%%
%% Ваше решение задачи здесь

Объем параллелепипеда равен произведению площади основания на длину соотв. высоты. \\
Рассмотрим основание, ... векторы $\vec{OA}$ и $\vec{OB}$. Его площадь равна $ab\sin{\gamma}$ \\
Теперь рассчитаем длину высоты. Она будет равна координате $z$ вектора $\vec{OC}$. \\
Рассмотрим $\vec{OC}$ как повернутый вектор $\{0, 0, c\}$ \\
$\vec{OC} =
\{0, 0, c\} *
\left(
  \begin{array}{rrr}
    \cos{\alpha} & 0 & -\sin{\alpha} \\
    0 & 1 & 0 \\
    \sin{\alpha} & 0 & \cos{\alpha}
  \end{array}
\right)
*
\left(
  \begin{array}{rrr}
    1 & 0 & 0 \\
    0 & \cos{\beta} & \sin{\beta} \\
    0 & -\sin{\beta} & \cos{\beta}
  \end{array}
\right)
=
\{c\sin{\alpha}, -c\sin{\beta}\cos{\alpha}, c\cos{\alpha}\cos{\beta}\}
$ \\
Значит высота будет равна $c\cos{\alpha}\cos{\beta}$. \\
Значит объем параллелепипеда равен $abc\cos{\alpha}\cos{\beta}\sin{\gamma}$ \\
Итого ответ: $abc\cos{\alpha}\cos{\beta}\sin{\gamma}$

%%%%%%%%%%%%%%%%%%%%%%%%%%%%%%%%%%%%%%%%%%%%%%%%%%%%
\end{solution}

%%%%%%%%%%%%%%%%%%%%%%%%%%%%%%%%%%%%%%%%%%%%%%%%%%%%
% Задача 8
\begin{problem}{192}
Три вектора $\vec a$, $\vec b$, $\vec c$ связаны соотношениями $\vec{a}=[\vec{b},\vec{c}]$,
$\vec{b}=[\vec{c},\vec{a}]$, $\vec{c}=[\vec{a},\vec{b}]$. Найти длины этих векторов.
\end{problem}
\begin{solution}
%%%%%%%%%%%%%%%%%%%%%%%%%%%%%%%%%%%%%%%%%%%%%%%%%%%%
%% Ваше решение задачи здесь

При таком условии базис $(\vec{a}, \vec{b}, \vec{c})$ - ортоганальный.
Значит синус угла между любыми двумя векторами равен 1. Значит длина каждого из векторов равна произведению длин оставшихся. \\
$
a = bc; b = ac; c = ab \\
\frac{bc}{a} = \frac{ac}{b} = \frac{ab}{c} = 1 \\
b^2c^2 = a^2c^2 = a^2b^2 \\
$ тогда \\
$
a = b = c
$ при этом
$
Итого получился ортонормированный базис.
a = bc; b = ac; c = ad \\
$ Значит
$
a = b = c = 1
$ \\

%%%%%%%%%%%%%%%%%%%%%%%%%%%%%%%%%%%%%%%%%%%%%%%%%%%%
\end{solution}

%%%%%%%%%%%%%%%%%%%%%%%%%%%%%%%%%%%%%%%%%%%%%%%%%%%%
% Задача 9
\begin{problem}{197}
Даны три некомпланарных вектора $\overrightarrow{OA}=\vec{a}$, $\overrightarrow{OB}=\vec{b}$, $\overrightarrow{OC}=\vec{c}$,
отложенных от одной точки $O$. Найти вектор $\overrightarrow{OD}=\vec{d}$, отложенный от той же точки $O$ и образующий
с векторами $\overrightarrow{OA}$, $\overrightarrow{OB}$ и $\overrightarrow{OC}$ равные между собой острые углы.
\end{problem}
\begin{solution}
%%%%%%%%%%%%%%%%%%%%%%%%%%%%%%%%%%%%%%%%%%%%%%%%%%%%
%% Ваше решение задачи здесь

Так как от длины векторов не меняется угол, введем векторы
$
\vec{x} = \frac{\vec{a}}{a};
\vec{y} = \frac{\vec{b}}{b};
\vec{z} = \frac{\vec{c}}{c};
$, длина которых будет равна 1.
Возьмем точку $M$ - центроид треугольника ($\overrightarrow{OM} = \frac{\vec{x} + \vec{y} + \vec{z}}{3}$),
образованного векторами $\vec{x}, \vec{y}, \vec{z}$ и докажем, что
вектор $\overrightarrow{OM}$ образует с векторами $\vec{x}, \vec{y}, \vec{z}$ три равных угла. \\
Введем точки
$
X = \vec{x};
Y = \vec{y};
Z = \vec{z};
$ \\
Рассмотрим треугольники $(O, X, M), (O, Y, M), (O, Z, M)$.
Стороны каждого из них соотв. равны. Значит равны и треугольники. Значит вектор $\vec{OM}$ образует с векторами $\vec{x}, \vec{y}, \vec{z}$ равные углы чтд. \\
Итого ответ:
$
  \frac{\vec{a}}{3a} +
  \frac{\vec{b}}{3b} +
  \frac{\vec{c}}{3c}
}
$

%%%%%%%%%%%%%%%%%%%%%%%%%%%%%%%%%%%%%%%%%%%%%%%%%%%%
\end{solution}

%------------------------------------------------
\end{document}
