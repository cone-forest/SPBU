\documentclass[a4paper, 12pt]{article}
%----------------------------------------------------------------------------------------
%	PACKAGES AND OTHER DOCUMENT CONFIGURATIONS
%----------------------------------------------------------------------------------------
\usepackage[a4paper, total={7in, 10in}]{geometry}
\setlength{\parskip}{0pt}
\setlength{\parindent}{0in}

\usepackage[T2A]{fontenc}% Внутренняя T2A кодировка TeX
\usepackage[utf8]{inputenc}% кодировка файла
\usepackage[russian]{babel}% поддержка переносов в русском языке
\usepackage{amsthm, amsmath, amssymb} % Mathematical typesetting
\usepackage{float} % Improved interface for floating objects
\usepackage{graphicx, multicol} % Enhanced support for graphics
\usepackage{xcolor} % Driver-independent color extensions
\usepackage{mdframed}

\usepackage[yyyymmdd]{datetime} % Uses YEAR-MONTH-DAY format for dates
\renewcommand{\dateseparator}{.} % Sets dateseparator to '.'

\usepackage{fancyhdr} % Headers and footers
\pagestyle{fancy} % All pages have headers and footers
\fancyhead{}\renewcommand{\headrulewidth}{0pt} % Blank out the default header
\fancyfoot[L]{} % Custom footer text
\fancyfoot[C]{} % Custom footer text
\fancyfoot[R]{\thepage} % Custom footer text

\newenvironment{problem}[2][Задача]
    { \begin{mdframed}[backgroundcolor=gray!10] \textbf{#1 #2.} \\}
    {  \end{mdframed}}

\newenvironment{solution}
    {\textit{Решение: }}
    {\noindent\rule{7in}{1.5pt}}

\begin{document}

%-------------------------------
%	TITLE SECTION
%-------------------------------

\fancyhead[C]{}
\hrule \medskip % Upper rule
\begin{minipage}{0.295\textwidth}
\raggedright\footnotesize
Цуканов Михаил \hfill\\
st117303 \hfill\\
st117303@student.spbu.ru
\end{minipage}
\begin{minipage}{0.4\textwidth}
\centering\large
Homework Assignment 17\\
\normalsize
Алгебра и геометрия, 1 семестр\\
\end{minipage}
\begin{minipage}{0.295\textwidth}
\raggedleft
\today\hfill\\
\end{minipage}
\medskip\hrule
\bigskip

%------------------------------------------------
%	CONTENTS
%------------------------------------------------



%%%%%%%%%%%%%%%%%%%%%%%%%%%%%%%%%%%%%%%%%%%%%%%%%%%%
% Задача 1
\begin{problem}{405}
Две параллельные прямые $2x-5y+6=0$ и $2x-5y-7=0$ делят плоскость на три области:
полосу заключенную между этими прямыми и две области вне этой полосы.
Установить каким областям принадлежат точки $A(2,1)$, $B(3,2)$, $C(1,1)$, $D(2,8)$, $E(7,1)$, $F(-4,6)$.
\end{problem}
\begin{solution}
%%%%%%%%%%%%%%%%%%%%%%%%%%%%%%%%%%%%%%%%%%%%%%%%%%%%
%% Ваше решение задачи здесь

Нарисовав данные прямые и точки получим результаты: \\
$A(2,1)$ - между \\
$B(3,2)$ - между \\
$C(1,1)$ - между \\
$D(2,8)$ - сверху \\
$E(7,1)$ - снизу \\
$F(-4,6)$ - сверху

%%%%%%%%%%%%%%%%%%%%%%%%%%%%%%%%%%%%%%%%%%%%%%%%%%%%
\end{solution}

%%%%%%%%%%%%%%%%%%%%%%%%%%%%%%%%%%%%%%%%%%%%%%%%%%%%
% Задача 2
\begin{problem}{406}
Даны две точки $A(-3,1)$ и $B(5,4)$ и прямая ${x-2y+1=0}$. Установить, пересекает ли данная прямая отрезок $AB$ или его продолжение за точку $A$ или точку $B$.
\end{problem}
\begin{solution}
%%%%%%%%%%%%%%%%%%%%%%%%%%%%%%%%%%%%%%%%%%%%%%%%%%%%
%% Ваше решение задачи здесь

$y = \frac{x + 1}{2}$ \\
$y(-3) = -1$ \\
$y(5) = 3$ \\
Границы отрезка AB находятся выше соотв точек прямой. Значит они не пересекаются.

%%%%%%%%%%%%%%%%%%%%%%%%%%%%%%%%%%%%%%%%%%%%%%%%%%%%
\end{solution}

%%%%%%%%%%%%%%%%%%%%%%%%%%%%%%%%%%%%%%%%%%%%%%%%%%%%
% Задача 3
\begin{problem}{411}
Дан треугольник $ABC$: $A(3,1)$, $B(-2,4)$, $C(1,0)$ и прямая ${x-7y+5=0}$. Установить, пересекает ли прямая стороны треугольника или их продолжение.
\end{problem}
\begin{solution}
%%%%%%%%%%%%%%%%%%%%%%%%%%%%%%%%%%%%%%%%%%%%%%%%%%%%
%% Ваше решение задачи здесь

Рассмотрим значения прямой у точек треугольника. \\
$y = \frac{x + 5}{7}$ \\
$y(3) = \frac{8}{7} > 1$ \\
$y(-2) = \frac{3}{7} < 4$ \\
$y(1) = \frac{6}{7} > 0$ \\
Значит прямая пересекает стороны AB и BC и продолжение AC.

%%%%%%%%%%%%%%%%%%%%%%%%%%%%%%%%%%%%%%%%%%%%%%%%%%%%
\end{solution}

%%%%%%%%%%%%%%%%%%%%%%%%%%%%%%%%%%%%%%%%%%%%%%%%%%%%
% Задача 4
\begin{problem}{442}
Зная уравнение стороны треугольника $x+7y-6=0$ и уравнения биссектрис $x+y-2=0$, $x-3y-6=0$,
выходящих из концов этой стороны, найти координаты вершины, противолежащей данной стороне.
\end{problem}
\begin{solution}
%%%%%%%%%%%%%%%%%%%%%%%%%%%%%%%%%%%%%%%%%%%%%%%%%%%%
%% Ваше решение задачи здесь

Выразим $x$ через $y$ и константы. Подставим эти выражения в уравнения других прямых и получим точки пересечения биссектрисс и стороны - вершины треугольника. \\
$X_1(\frac{4}{3}, \frac{2}{3})$ \\
$X_2(6, 0);$
Найдем направляющие векторы (как перпендикуляры к нормалям, выраженные в уравнениях): \\
Сторона: $\vec{p_1}=\{7, -1\}$;
Биссектрисса 1: $\vec{p_2}=\{1, -1\}$;
Биссектрисса 2: $\vec{p_3}=\{1, 3\}$
\\
Найдем синусы и косинусы углов между биссектрисами и стороной через скалярное произведение направляющих векторов, затем используем их в матрице повороте. \\
$\cos{\widehat{\vec{p_1}, \vec{p_2}}} = \frac{7+1}{\sqrt{50}\sqrt2} = \frac{4}{5};
\sin{\widehat{\vec{p_1}, \vec{p_2}}} = \frac{3}{5}$ \\
$\cos{\widehat{\vec{p_1}, \vec{p_3}}} = \frac{7-3}{\sqrt{50}\sqrt{10}} = \frac{2\sqrt5}{25};
\sin{\widehat{\vec{p_1}, \vec{p_3}}} = \frac{121}{125}$ \\
Умножим на матрицы поворота: \\
$
\overrightarrow{p_2'} =
\vec{p_2} *
\left(
  \begin{array}{rr}
    \frac{4}{5} & -\frac{3}{5} \\
    \frac{3}{5} & \frac{4}{5} \\
  \end{array}
\right)
=
\{\frac{1}{5}, -\frac{8}{5}\}
$ \\
$
\overrightarrow{p_3'} =
\vec{p_3} *
\left(
  \begin{array}{rr}
    \frac{2\sqrt5}{25} & -\frac{121}{125} \\
    \frac{121}{125} & \frac{2\sqrt5}{25} \\
  \end{array}
\right)
= \{\frac{10\sqrt5 + 363}{125}, \frac{30\sqrt5 - 121}{125}\}
$ \\
Осталось вычислить пересечение получившихся сторон. \\
$
\left\{
  \begin{array}{rr}
    3\alpha + 20 = 
  \end{array}
\right.
$

%%%%%%%%%%%%%%%%%%%%%%%%%%%%%%%%%%%%%%%%%%%%%%%%%%%%
\end{solution}

%%%%%%%%%%%%%%%%%%%%%%%%%%%%%%%%%%%%%%%%%%%%%%%%%%%%
% Задача 5
\begin{problem}{559}
Даны две точки $A(-3,1,5)$ и $B(5,4,2)$ и плоскость $ 2x - 4y + z + 14 = 0$.
Установить, пересекает ли данная плоскость отрезок $AB$, его продолжение за точкой $A$ или за точкой $B$?
\end{problem}
\begin{solution}
%%%%%%%%%%%%%%%%%%%%%%%%%%%%%%%%%%%%%%%%%%%%%%%%%%%%
%% Ваше решение задачи здесь

Свободный член уравнения плоскости равен 14,
значит проекция точки на плоскости на нормаль равна $-14 * n$, где $n$ - длина нормали.
Нормаль: $\vec{n} = \{2, -4, 1\}$. \\
$\vec{n} \cdot \vec{a} = -5$ \\
$\vec{n} \cdot \vec{b} = -4$ \\
Значит обе точки ближе к нулю координат, чем плоскость. При этом точка B ближе к нулю координат, чем A.
То есть плоскость пересекает продолжение отрезка $AB$ за точкой А.

%%%%%%%%%%%%%%%%%%%%%%%%%%%%%%%%%%%%%%%%%%%%%%%%%%%%
\end{solution}

%%%%%%%%%%%%%%%%%%%%%%%%%%%%%%%%%%%%%%%%%%%%%%%%%%%%
% Задача 6
\begin{problem}{608}
Составить уравнение биссекторных плоскостей двугранных углов между плоскостями
$7x+y-6=0$ и $3x+5y-4z+1=0$.
\end{problem}
\begin{solution}
%%%%%%%%%%%%%%%%%%%%%%%%%%%%%%%%%%%%%%%%%%%%%%%%%%%%
%% Ваше решение задачи здесь

Биссекторная плоскость будет выражаться через нормаль и точку на ней. \\
Ее нормаль - среднее арифметическое нормализованных нормалей данных плоскостей: \\
$
\overrightarrow{n_1} = \{7, 1, 0\}; n_1 = \sqrt{50} = 5\sqrt2;
\overrightarrow{n_2} = \{3, 5, -4\}; n_2 = \sqrt{50} = 5\sqrt2;
$ \\
$
\vec{n} = \frac{\overrightarrow{n_1}}{2n_1} + \frac{\overrightarrow{n_2}}{2n_2} =
\frac{\{10, 6, -4\}}{10\sqrt2} = \{\frac{\sqrt2}{2}, \frac{3\sqrt2}{10}, -\frac{\sqrt2}{5}\}
$ \\
На этом этапе можно умножить нормаль на константу, тк направление не изменится. \\
$\vec{n} = \{5, 3, -2\}$ \\
Ее точка - любая точка прямой-пересечения данных плоскостей: \\
$
\left\{
  \begin{array}{r}
    7x + y - 6 = 0 \\
    3x + 5y - 4z + 1 = 0
  \end{array}
\right.
\left\{
  \begin{array}{r}
    y = 6 - 7x \\
    3x + 30 - 35x - 4z + 1 = 0
  \end{array}
\right.
\left\{
  \begin{array}{r}
    y = 6 - 7x \\
    31 = 32x + 4z
  \end{array}
\right.
\left\{
  \begin{array}{r}
    y = 6 - 7\frac{31 - 4z}{32} \\
    x = \frac{31 - 4z}{32} \\
    z = z
  \end{array}
\right.
\\
\left\{
  \begin{array}{r}
    x = -\frac{25}{32} + \frac{7t}{8} \\
    y = \frac{31}{32} - \frac{t}{8} \\
    z = t
  \end{array}
\right.
$
Значит если $t = 0$, то точка на плоскости: $X(-\frac{25}{32}, \frac{31}{32}, 0)$ \\

Итого уравнение биссекторной плоскости: $\vec{p} = \{-\frac{25}{32}, \frac{31}{32}, 0\} + t\{5, 3, -2\}$


%%%%%%%%%%%%%%%%%%%%%%%%%%%%%%%%%%%%%%%%%%%%%%%%%%%%
\end{solution}


%------------------------------------------------
\end{document}
