\documentclass[a4paper, 12pt]{article}
%----------------------------------------------------------------------------------------
%	PACKAGES AND OTHER DOCUMENT CONFIGURATIONS
%----------------------------------------------------------------------------------------
\usepackage[a4paper, total={7in, 10in}]{geometry}
\setlength{\parskip}{0pt}
\setlength{\parindent}{0in}

\usepackage[T2A]{fontenc}% Внутренняя T2A кодировка TeX
\usepackage[utf8]{inputenc}% кодировка файла
\usepackage[russian]{babel}% поддержка переносов в русском языке
\usepackage{amsthm, amsmath, amssymb} % Mathematical typesetting
\usepackage{float} % Improved interface for floating objects
\usepackage{graphicx, multicol} % Enhanced support for graphics
\usepackage{xcolor} % Driver-independent color extensions
\usepackage{mdframed}

\usepackage[yyyymmdd]{datetime} % Uses YEAR-MONTH-DAY format for dates
\renewcommand{\dateseparator}{.} % Sets dateseparator to '.'

\usepackage{fancyhdr} % Headers and footers
\pagestyle{fancy} % All pages have headers and footers
\fancyhead{}\renewcommand{\headrulewidth}{0pt} % Blank out the default header
\fancyfoot[L]{} % Custom footer text
\fancyfoot[C]{} % Custom footer text
\fancyfoot[R]{\thepage} % Custom footer text

\newenvironment{problem}[2][Задача]
    { \begin{mdframed}[backgroundcolor=gray!10] \textbf{#1 #2.} \\}
    {  \end{mdframed}}

\newenvironment{solution}
    {\textit{Решение: }}
    {\noindent\rule{7in}{1.5pt}}

\begin{document}

%-------------------------------
%	TITLE SECTION
%-------------------------------

\fancyhead[C]{}
\hrule \medskip % Upper rule
\begin{minipage}{0.295\textwidth}
\raggedright\footnotesize
Цуканов Михаил \hfill\\
st117303 \hfill\\
st117303@student.spbu.ru
\end{minipage}
\begin{minipage}{0.4\textwidth}
\centering\large
Homework Assignment 14\\
\normalsize
Алгебра и геометрия, 1 семестр\\
\end{minipage}
\begin{minipage}{0.295\textwidth}
\raggedleft
\today\hfill\\
\end{minipage}
\medskip\hrule
\bigskip

%------------------------------------------------
%	CONTENTS
%------------------------------------------------



%%%%%%%%%%%%%%%%%%%%%%%%%%%%%%%%%%%%%%%%%%%%%%%%%%%%
% Задача 1
\begin{problem}{92}
Даны две смежные вершины параллелограмма $ABCD$: $A(-4,-7)$ и $B(2,6)$ и точка пересечения его диагоналей $M(3,1)$.
Найти две другие вершины параллелограмма. Система координат аффинная.
\end{problem}
\begin{solution}
%%%%%%%%%%%%%%%%%%%%%%%%%%%%%%%%%%%%%%%%%%%%%%%%%%%%
%% Ваше решение задачи здесь

Переведем параллелограмм в систему координат, в которой начало в точке A. \\
Значит новые координаты точек: $A(0, 0), B(6, 13), M(7, 8)$ \\
$2*\vec{AM} = \vec{AC}$ => $C(14, 16)$ \\
$\vec{AM} = \vec{AB} + \frac{1}{2}\vec{BD}$ <=> $\vec{AD} = 2 *(\vec{AM} - \frac{1}{2}\vec{AB})$
=> $D(14 - 6, 16 - 13) <=> D(8, 3)$
Теперь переведем полученные значения в изначальную систему координат. \\
$C(10, 9), D(4, -4)$
Ответ: $C(10, 9), D(4, -4)$

%%%%%%%%%%%%%%%%%%%%%%%%%%%%%%%%%%%%%%%%%%%%%%%%%%%%
\end{solution}

%%%%%%%%%%%%%%%%%%%%%%%%%%%%%%%%%%%%%%%%%%%%%%%%%%%%
% Задача 2
\begin{problem}{111}
Даны две точки $A(8,-6,7)$ и $B(-20,15,10)$. Установить, пересекает ли прямая $AB$ какую-нибудь из осей координат.
\end{problem}
\begin{solution}
%%%%%%%%%%%%%%%%%%%%%%%%%%%%%%%%%%%%%%%%%%%%%%%%%%%%
%% Ваше решение задачи здесь

Определим уравнение прямой. \\
$l = A + \alpha(B - A)$ \\
Тогда нужно решить 3 системы уравнений относительно $\alpha$. \\
$
\left(
  \begin{array}{rrr}
    8 -28\alpha & \beta \\
    -6 + 21\alpha & 0 \\
    7 + 3\alpha & 0
  \end{array}
\right)
\alpha = -\frac{7}{3}$ и $\alpha = \frac{2}{7}
$
Нет решений
$
\\
\left(
  \begin{array}{rrr}
    8 - 28\alpha & 0 \\
    -6 + 21\alpha & \beta \\
    7 + 3\alpha & 0
  \end{array}
\right)
\alpha = -\frac{7}{3}$ и $\alpha = \frac{2}{7}
$
Нет решений
$
\\
\left(
  \begin{array}{rrr}
    8 -28\alpha & 0 \\
    -6 + 21\alpha & 0 \\
    7 + 3\alpha & \beta
  \end{array}
\right)
\beta = \frac{55}{7}$ и $\alpha = \frac{2}{7}
$
\\
Итого ответ: Прямая пересекает ось z.

%%%%%%%%%%%%%%%%%%%%%%%%%%%%%%%%%%%%%%%%%%%%%%%%%%%%
\end{solution}

%%%%%%%%%%%%%%%%%%%%%%%%%%%%%%%%%%%%%%%%%%%%%%%%%%%%
% Задача 3
\begin{problem}{120}
Относительно полярной системы координат даны точки $A(2,\frac{\pi}{3})$, $B(\sqrt{2},\frac{3\pi}{4})$,
$C(5,\frac{\pi}{2})$, $D(3,\frac{\pi}{6})$. Найти координаты этих точек в соответствующей прямоугольной системе координат.
\end{problem}
\begin{solution}
%%%%%%%%%%%%%%%%%%%%%%%%%%%%%%%%%%%%%%%%%%%%%%%%%%%%
%% Ваше решение задачи здесь

$X(l, \phi) = X(l\cos{\phi}, l\sin{\phi})$ \\
$A(2,\frac{\pi}{3}) = A(1, \sqrt{3})$ \\
$B(\sqrt{2},\frac{3\pi}{4}) = B(-1, 1)$ \\
$C(5,\frac{\pi}{2}) = C(0, 5)$ \\
$D(3,\frac{\pi}{6}) = D(\frac{3\sqrt{3}}{2}, \frac{3}{2})$

%%%%%%%%%%%%%%%%%%%%%%%%%%%%%%%%%%%%%%%%%%%%%%%%%%%%
\end{solution}

%%%%%%%%%%%%%%%%%%%%%%%%%%%%%%%%%%%%%%%%%%%%%%%%%%%%
% Задача 4
\begin{problem}{122}
Зная полярные координаты точки $\rho = 10$, $\varphi = \frac{\pi}{6}$. Найти её прямоугольные координаты, если полюс
находится в точке $(2,3)$, а полярная ось параллельна оси $Ox$.
\end{problem}
\begin{solution}
%%%%%%%%%%%%%%%%%%%%%%%%%%%%%%%%%%%%%%%%%%%%%%%%%%%%
%% Ваше решение задачи здесь

Данная полярная система координат сдвинута относительно начала декартовой системы координат на вектор $\{2, 3\}$.
Значит его нужно будет прибавить к полученным значениям. \\
$A(10, \frac{\pi}{6} = A(5\sqrt{3}, 5) = A(2 + 5\sqrt{3}, 8)$ \\
Значение в 3х системах координат. В начальной в промежуточной декартовой с началом в точке (2, 3) и конечная - декартова с началом в точке (0, 0) \\
Итого ответ: $A(2 + 5\sqrt{3}, 8)$

%%%%%%%%%%%%%%%%%%%%%%%%%%%%%%%%%%%%%%%%%%%%%%%%%%%%
\end{solution}

%%%%%%%%%%%%%%%%%%%%%%%%%%%%%%%%%%%%%%%%%%%%%%%%%%%%
% Задача 5
\begin{problem}{128}
Найти цилиндрические координаты точек по их прямоугольным координатам $A(3, -4, 5)$, $B(1, -1, 1)$, $C(-6, 0, 8)$.
\end{problem}
\begin{solution}
%%%%%%%%%%%%%%%%%%%%%%%%%%%%%%%%%%%%%%%%%%%%%%%%%%%%
%% Ваше решение задачи здесь

$M(x, y, z) = M(z, r, \phi) = M(z, \sqrt{x^2 + y^2}, \phi)$ \\
При этом угол $\phi$ будет определятся косинусом и синусом угла ($\frac{x}{r}, \frac{y}{r}$) \\
$A(3, -4, 5) = A(5, 5, \arcsin(-\frac{4}{5}))$ \\
$B(1, -1, 1) = B(1, \sqrt{2}, \arcsin(-\frac{\sqrt{2}}{2}) = B(1, \sqrt{2}, -\frac{\pi}{4})$ \\
$C(-6, 0, 8) = C(8, 6, \arccos(-1)) = C(8, 6, \pi)$

%%%%%%%%%%%%%%%%%%%%%%%%%%%%%%%%%%%%%%%%%%%%%%%%%%%%
\end{solution}

%%%%%%%%%%%%%%%%%%%%%%%%%%%%%%%%%%%%%%%%%%%%%%%%%%%%
% Задача 6
\begin{problem}{665}
Найти формулы перехода от одной аффинной системы координат $Oxy$ с координатным углом $\omega$
к другой аффинной системе координат $Ox'y'$, если одноименные оси этих систем взаимно перпендикулярны, а разноименные образуют острые углы.
Длины базисных векторов равны 1.
\end{problem}
\begin{solution}
%%%%%%%%%%%%%%%%%%%%%%%%%%%%%%%%%%%%%%%%%%%%%%%%%%%%
%% Ваше решение задачи здесь

$x$ перпендикулярен $x'$ и
$y$ перпендикулярен $y'$,
а значит одноименные оси координат не будут оказывать друг на друга влияния. \\
При этом разноименные оси образуют острый угол, а значит будут влиять друг на друга (при этом проекции положительные). \\
Этот острый угол будет равен $\frac{\pi}{2} - \omega$ \\
Спроектировав на оси $Ox' Oy'$ вектор $\{1, 0\}$, заданный в $Oxy$ системе координат, получим $\{0, cos(\frac{\pi}{2} - \omega)\}$ \\
Спроектировав на оси $Ox' Oy'$ вектор $\{0, 1\}$, заданный в $Oxy$ системе координат, получим $\{\cos(\frac{\pi}{2} - \omega), 0\}$ \\
Упростим $\cos(\frac{\pi}{2} - \omega) = -\sin{\omega}$ \\
Любой вектор в системе координат $Oxy$ будет линейной комбинацией векторов $\{0, 1\}$ и $\{1, 0\}$. А значит и любой вектор в системе координат $Oxy$ будет равен $\{-y\sin{\omega}, -x\sin{\omega}\}$. \\
Такое преобразование можно представить в виде матрицы. \\
$
\left(
  \begin{array}{rr}
    0 & -\sin{\omega} \\
    -\sin{\omega} & 0
  \end{array}
\right)
$
\\
Домножив вектор в системе координат $Oxy$ на эту матрицу получим вектор в системе координат $Ox'y'$.
(На обратную матрицу - обратно)

%%%%%%%%%%%%%%%%%%%%%%%%%%%%%%%%%%%%%%%%%%%%%%%%%%%%
\end{solution}


%------------------------------------------------
\end{document}
