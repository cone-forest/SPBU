\documentclass[a4paper, 12pt]{article}
%----------------------------------------------------------------------------------------
%	PACKAGES AND OTHER DOCUMENT CONFIGURATIONS
%----------------------------------------------------------------------------------------
\usepackage[a4paper, total={7in, 10in}]{geometry}
\setlength{\parskip}{0pt}
\setlength{\parindent}{0in}

\usepackage[T2A]{fontenc}% Внутренняя T2A кодировка TeX
\usepackage[utf8]{inputenc}% кодировка файла
\usepackage[russian]{babel}% поддержка переносов в русском языке
\usepackage{amsthm, amsmath, amssymb} % Mathematical typesetting
\usepackage{float} % Improved interface for floating objects
\usepackage{graphicx, multicol} % Enhanced support for graphics
\usepackage{xcolor} % Driver-independent color extensions
\usepackage{mdframed}

\usepackage[yyyymmdd]{datetime} % Uses YEAR-MONTH-DAY format for dates
\renewcommand{\dateseparator}{.} % Sets dateseparator to '.'

\usepackage{fancyhdr} % Headers and footers
\pagestyle{fancy} % All pages have headers and footers
\fancyhead{}\renewcommand{\headrulewidth}{0pt} % Blank out the default header
\fancyfoot[L]{} % Custom footer text
\fancyfoot[C]{} % Custom footer text
\fancyfoot[R]{\thepage} % Custom footer text

\newenvironment{problem}[2][Задача]
    { \begin{mdframed}[backgroundcolor=gray!10] \textbf{#1 #2.} \\}
    {  \end{mdframed}}

\newenvironment{solution}
    {\textit{Решение: }}
    {\noindent\rule{7in}{1.5pt}}

\begin{document}

%-------------------------------
%	TITLE SECTION
%-------------------------------

\fancyhead[C]{}
\hrule \medskip % Upper rule
\begin{minipage}{0.295\textwidth}
\raggedright\footnotesize
Михаил Цуканов \hfill\\
st117303 \hfill\\
st117303@student.spbu.ru
\end{minipage}
\begin{minipage}{0.4\textwidth}
\centering\large
Homework Assignment 18\\
\normalsize
Алгебра и геометрия, 1 семестр\\
\end{minipage}
\begin{minipage}{0.295\textwidth}
\raggedleft
\today\hfill\\
\end{minipage}
\medskip\hrule
\bigskip

%------------------------------------------------
%	CONTENTS
%------------------------------------------------



%%%%%%%%%%%%%%%%%%%%%%%%%%%%%%%%%%%%%%%%%%%%%%%%%%%%
% Задача 1
\begin{problem}{527(b)}
Преобразовать квадратичную форму к сумме квадратов методом Лагранжа: $x_1^2 - 4 x_1 x_2 + 2 x_1 x_3 + 4 x_2^2 + x_3^2$
\end{problem}
\begin{solution}
%%%%%%%%%%%%%%%%%%%%%%%%%%%%%%%%%%%%%%%%%%%%%%%%%%%%
%% Ваше решение задачи здесь

Для удобства сделаем следующие замены: $x_1=a,x_2=b,x_3=c$
\\
$a^2-4ab+2ac+4b^2+c^2=(a^2-2a(2b-c)+(2b-c)^2)-(2b-c)^2+4b^2+c^2=(a+(2b-c))^2-(2b-c)^2+4b^2+c^2=(a-(2b-c))^2+4bc$
\\
Пусть $y_1=a-2b+c,b=y_2+y_3,c=y_2-y_3 \rightarrow$ получим каноническую форму:
\\
$y_1^2+4y_2^2-4y_3^2$
\\
$X=
\left(
\begin{array}{ccc}
1 & 1 &3 \\
0 & 1 & 1 \\
0 & 1 & -1 \\
\end{array}
\right)*Y
$

%%%%%%%%%%%%%%%%%%%%%%%%%%%%%%%%%%%%%%%%%%%%%%%%%%%%
\end{solution}

%%%%%%%%%%%%%%%%%%%%%%%%%%%%%%%%%%%%%%%%%%%%%%%%%%%%
% Задача 2
\begin{problem}{527(d)}
Преобразовать квадратичную форму к сумме квадратов методом Лагранжа: $x_1^2 - 2 x_1 x_2 + 2 x_1 x_3 -2 x_1 x_4 + x_2^2 + 2 x_2 x_3 - 4 x_2 x_4 + x_3^2 - 2 x_4^2$
\end{problem}
\begin{solution}
%%%%%%%%%%%%%%%%%%%%%%%%%%%%%%%%%%%%%%%%%%%%%%%%%%%%
%% Ваше решение задачи здесь

Для удобства сделаем следующие замены: $x_1=a,x_2=b,x_3=c,x_4=d$
\\
$a^2-2ab+2ac-2ad+b^2+2bc-4bd+c^2-2d^2=a^2-2a(b-c+d)+(b-c+d)^2-b^2+2bc-2bd-c^2+2cd-d^2+b^2+2bc-4bd+c^2-2d^2=(a-b+c-d)^2+4bc-6bd+2cd-3d^2$
\\
Заменим: $y_1=a-b+c-d,y_2=b,y_3=c,y_4=d (a=y_1+y_2-y_3+y_4)$ =>
\\
$y_1^2+4y_2y_3-6y_2y_4+2y_3y_4-3y_4^2=y_1^2+2y_2(2y_3-3y_4)+y_4(2y_3-3y_4)$
\\
Заменим: $y_1=z_1,y_2=z_2+z_3,y_3=z_2-z_3,y_4=z_4$ =>
\\
$z_1^2+4z_2^2-4z_3^2-6z_2z_4-6z_3z_4+2z_2z_4-2z_3z_4-3z_4^2=z_1^2+(2z_2-z_4)^2-4(z_3+z_4)^2$
\\
Заменим: $z_1=k_1,z_2=0,5k_2+0,5k_4,z_3=k_3-k_4,z_4=k_4$, получим каноническую форму:
\\
$k_1^2+k_2^2-4k_3^2$
\\
$X=
\left(
\begin{array}{cccc}
1 & 1 & -1 & 1\\
0 & 1 & 0 & 0\\
0 & 0 & 1 & 0\\
0 & 0 & 0 & 1\\
\end{array}
\right)
*
\left(
\begin{array}{cccc}
1 & 0 & 0 & 0\\
0 & 1 & 1 & 0\\
0 & 1 & -1 & 0\\
0 & 0 & 0 & 1\\
\end{array}
\right)
*
\left(
\begin{array}{cccc}
1 & 0 & 0 & 0\\
0 & 0,5 & 0 & 0,5\\
0 & 0 & 1 & -1\\
0 & 0 & 0 & 1\\
\end{array}
\right)
*
K
=
\\
=
\left(
\begin{array}{cccc}
1 & 0 & 2 & -1\\
0 & 0,5 & 1 & -0,5\\
0 & 0,5 & -1 & 1,5\\
0 & 0 & 0 & 1\\
\end{array}
\right)
*K
$

%%%%%%%%%%%%%%%%%%%%%%%%%%%%%%%%%%%%%%%%%%%%%%%%%%%%
\end{solution}


%------------------------------------------------
\end{document}
