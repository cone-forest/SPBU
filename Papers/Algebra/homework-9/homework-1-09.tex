\documentclass[a4paper, 12pt]{article}
%----------------------------------------------------------------------------------------
%	PACKAGES AND OTHER DOCUMENT CONFIGURATIONS
%----------------------------------------------------------------------------------------
\usepackage[a4paper, total={7in, 10in}]{geometry}
\setlength{\parskip}{0pt}
\setlength{\parindent}{0in}

\usepackage[T2A]{fontenc}% Внутренняя T2A кодировка TeX
\usepackage[utf8]{inputenc}% кодировка файла
\usepackage[russian]{babel}% поддержка переносов в русском языке
\usepackage{amsthm, amsmath, amssymb} % Mathematical typesetting
\usepackage{float} % Improved interface for floating objects
\usepackage{graphicx, multicol} % Enhanced support for graphics
\usepackage{xcolor} % Driver-independent color extensions
\usepackage{mdframed}

\usepackage[yyyymmdd]{datetime} % Uses YEAR-MONTH-DAY format for dates
\renewcommand{\dateseparator}{.} % Sets dateseparator to '.'

\usepackage{fancyhdr} % Headers and footers
\pagestyle{fancy} % All pages have headers and footers
\fancyhead{}\renewcommand{\headrulewidth}{0pt} % Blank out the default header
\fancyfoot[L]{} % Custom footer text
\fancyfoot[C]{} % Custom footer text
\fancyfoot[R]{\thepage} % Custom footer text

\newenvironment{problem}[2][Задача]
    { \begin{mdframed}[backgroundcolor=gray!10] \textbf{#1 #2.} \\}
    {  \end{mdframed}}

\newenvironment{solution}
    {\textit{Решение: }}
    {\noindent\rule{7in}{1.5pt}}

\begin{document}

%-------------------------------
%	TITLE SECTION
%-------------------------------

\fancyhead[C]{}
\hrule \medskip % Upper rule
\begin{minipage}{0.295\textwidth}
\raggedright\footnotesize
Цуканов Михаил \hfill\\
st117303 \hfill\\
st117303@student.spbu.ru
\end{minipage}
\begin{minipage}{0.4\textwidth}
\centering\large
Homework Assignment 9\\
\normalsize
Алгебра и геометрия, 1 семестр\\
\end{minipage}
\begin{minipage}{0.295\textwidth}
\raggedleft
\today\hfill\\
\end{minipage}
\medskip\hrule
\bigskip

%------------------------------------------------
%	CONTENTS
%------------------------------------------------



%%%%%%%%%%%%%%%%%%%%%%%%%%%%%%%%%%%%%%%%%%%%%%%%%%%%
% Задача 1
\begin{problem}{410(d)}
Обратить матрицу:
$\left(\begin{array}{rrrr}1 & 3 & -5 & 7\\0 & 1 & 2 & -3\\ 0 & 0 & 1 & 2 \\ 0 & 0 & 0 & 1\end{array}\right)$

\end{problem}
\begin{solution}
%%%%%%%%%%%%%%%%%%%%%%%%%%%%%%%%%%%%%%%%%%%%%%%%%%%%
%% Ваше решение задачи здесь

Воспользуемся методом Гаусса: \\
$
\left(
\begin{array}{rrrrrrrr}
1 & 3 & -5 & 7 & 1 & 0 & 0 & 0 \\
0 & 1 & 2 & -3 & 0 & 1 & 0 & 0 \\
0 & 0 & 1 & 2  & 0 & 0 & 1 & 0 \\
0 & 0 & 0 & 1  & 0 & 0 & 0 & 1
\end{array}
\right)
=>
\left(
\begin{array}{rrrrrrrr}
1 & 3 & -5 & 0 & 1 & 0 & 0 & -7 \\
0 & 1 & 2 & 0 & 0 & 1 & 0 & 3 \\
0 & 0 & 1 & 0  & 0 & 0 & 1 & -2 \\
0 & 0 & 0 & 1  & 0 & 0 & 0 & 1
\end{array}
\right)
=> \\
\left(
\begin{array}{rrrrrrrr}
1 & 3 & 0 & 0 & 1 & 0 & 5 & -17 \\
0 & 1 & 0 & 0 & 0 & 1 & -2 & 7 \\
0 & 0 & 1 & 0  & 0 & 0 & 1 & -2 \\
0 & 0 & 0 & 1  & 0 & 0 & 0 & 1
\end{array}
\right)
=>
\left(
\begin{array}{rrrrrrrr}
1 & 0 & 0 & 0 & 1 & -3 & 11 & -38 \\
0 & 1 & 0 & 0 & 0 & 1 & -2 & 7 \\
0 & 0 & 1 & 0 & 0 & 0 & 1 & -2 \\
0 & 0 & 0 & 1 & 0 & 0 & 0 & 1
\end{array}
\right)
=> \\
\left(
\begin{array}{rrrr}
1 & 3 & -5 & 7 \\
0 & 1 & 2 & -3 \\
0 & 0 & 1 & 2 \\
0 & 0 & 0 & 1
\end{array}
\right)^{-1}
=
\left(
\begin{array}{rrrr}
1 & -3 & 11 & -38 \\
0 & 1 & -2 & 7 \\
0 & 0 & 1 & -2 \\
0 & 0 & 0 & 1
\end{array}
\right)
$
\\
\\
\\
\\

%%%%%%%%%%%%%%%%%%%%%%%%%%%%%%%%%%%%%%%%%%%%%%%%%%%%
\end{solution}

%%%%%%%%%%%%%%%%%%%%%%%%%%%%%%%%%%%%%%%%%%%%%%%%%%%%
% Задача 2
\begin{problem}{410(f)}
Обратить матрицу:
$\left(\begin{array}{rrrr}1 & 1 & 1 & 1\\1 & 1 & -1 & -1\\ 1 & -1 & 1 & -1 \\ 1 & -1 & -1 & 1\end{array}\right)$

\end{problem}
\begin{solution}
%%%%%%%%%%%%%%%%%%%%%%%%%%%%%%%%%%%%%%%%%%%%%%%%%%%%
%% Ваше решение задачи здесь

Воспользуемся методом Гаусса: \\
$
\\
\left(
\begin{array}{rrrrrrrr}
1 & 1 & 1 & 1   & 1 & 0 & 0 & 0 \\
1 & 1 & -1 & -1 & 0 & 1 & 0 & 0 \\
1 & -1 & 1 & -1 & 0 & 0 & 1 & 0 \\
1 & -1 & -1 & 1 & 0 & 0 & 0 & 1
\end{array}
\right)
=>
\left(
\begin{array}{rrrrrrrr}
1 & 1 & 1 & 1 & 1 & 0 & 0 & 0 \\
2 & 2 & 0 & 0 & 1 & 1 & 0 & 0 \\
2 & 0 & 2 & 0 & 1 & 0 & 1 & 0 \\
2 & 0 & 0 & 2 & 1 & 0 & 0 & 1
\end{array}
\right)
=> \\
\left(
\begin{array}{rrrrrrrr}
-4 & 0 & 0 & 0 & -1 & -1 & -1 & -1 \\
 2 & 2 & 0 & 0 & 1 & 1 & 0 &  0 \\
 2 & 0 & 2 & 0 & 1 & 0 & 1 &  0 \\
 2 & 0 & 0 & 2 & 1 & 0 & 0 &  1
\end{array}
\right) =>
\left(
\begin{array}{rrrrrrrr}
4 & 0 & 0 & 0 & 1 & 1 & 1 & 1 \\
0 & 4 & 0 & 0 & 1 & 1 & -1 & -1 \\
0 & 0 & 4 & 0 & 1 & -1 & 1 & -1 \\
0 & 0 & 0 & 4 & 1 & -1 & -1 & 1
\end{array}
\right)
=> \\
\left(
\begin{array}{rrrrrrrr}
1 & 0 & 0 & 0 & \frac{1}{4} & \frac{1}{4} & \frac{1}{4} & \frac{3}{4} \\
0 & 1 & 0 & 0 & \frac{1}{4} & \frac{1}{4} & -\frac{1}{4} & -\frac{1}{4} \\
0 & 0 & 1 & 0 & \frac{1}{4} & -\frac{1}{4} & \frac{1}{4} & -\frac{1}{4} \\
0 & 0 & 0 & 1 & \frac{1}{4} & -\frac{1}{4} & -\frac{1}{4} & \frac{1}{4}
\end{array}
\right)
\\
\left(
\begin{array}{rrrr}
1 & 1 & 1 & 1 \\
1 & 1 & -1 & -1 \\
1 & -1 & 1 & -1 \\
1 & -1 & -1 & 1
\end{array}
\right)^{-1}
=
\left(
  \begin{array}{rrrr}
    \frac{1}{4} & \frac{1}{4} & \frac{1}{4} & \frac{3}{4} \\
    \frac{1}{4} & \frac{1}{4} & -\frac{1}{4} & -\frac{1}{4} \\\
    \frac{1}{4} & -\frac{1}{4} & \frac{1}{4} & -\frac{1}{4} \\\
    \frac{1}{4} & -\frac{1}{4} & -\frac{1}{4} & \frac{1}{4}
  \end{array}
\right)
$

%%%%%%%%%%%%%%%%%%%%%%%%%%%%%%%%%%%%%%%%%%%%%%%%%%%%
\end{solution}

%%%%%%%%%%%%%%%%%%%%%%%%%%%%%%%%%%%%%%%%%%%%%%%%%%%%
% Задача 3
\begin{problem}{416}
$\left( \begin{array}{ccccc}2 & -1 & 0 & \ldots & 0 \\ -1 & 2 & -1 & \ldots & 0 \\ 0 & -1 & 2 & \ldots & 0 \\ \vdots & \vdots & \vdots & \ddots & \vdots \\ 0 & 0 & 0 & \ldots & 2 \end{array} \right)^{-1}$

\end{problem}
\begin{solution}
%%%%%%%%%%%%%%%%%%%%%%%%%%%%%%%%%%%%%%%%%%%%%%%%%%%%
%% Ваше решение задачи здесь

Воспользуемся методом Гаусса: \\
$
\left(
\begin{array}{ccccc}
2 & -1 & 0 & \ldots & 0 \\
-1 & 2 & -1 & \ldots & 0 \\
0 & -1 & 2 & \ldots & 0 \\
\vdots & \vdots & \vdots & \ddots & \vdots \\
0 & 0 & 0 & \ldots & 2
\end{array}
\right)^{-1}
$
\\ Будем умножать $i$ строчку на $i$ и прибавлять к ней $i - 1$ю \\
$
\left(
\begin{array}{cccccccccc}
2 & -1 & 0 & \ldots & 0  & 1 & 0 & 0 & \ldots & 0\\
-1 & 2 & -1 & \ldots & 0 & 0 & 1 & 0 & \ldots & 0 \\
0 & -1 & 2 & \ldots & 0  & 0 & 0 & 1 & \ldots & 0\\
\vdots & \vdots & \vdots & \ddots & \vdots & \vdots & \vdots & \vdots & \ddots & \vdots \\
0 & 0 & 0 & \ldots & 2 & 0 & 0 & 0 & \ldots & 1
\end{array}
\right)
=>
\left(
\begin{array}{cccccccccc}
2 & -1 & 0 & \ldots & 0  & 1 & 0 & 0 & \ldots & 0\\
-2 & 4 & -2 & \ldots & 0 & 0 & 2 & 0 & \ldots & 0 \\
0 & -3 & 6 & \ldots & 0  & 0 & 0 & 3 & \ldots & 0\\
\vdots & \vdots & \vdots & \ddots & \vdots & \vdots & \vdots & \vdots & \ddots & \vdots \\
0 & 0 & 0 & \ldots & n + 1 & 0 & 0 & 0 & \ldots & n
\end{array}
\right)
=>
\left(
\begin{array}{cccccccccc}
2 & -1 & 0 & \ldots & 0  & 1 & 0 & 0 & \ldots & 0\\
0 & 3 & -2 & \ldots & 0 & 0 & 2 & 0 & \ldots & 0 \\
0 & 0 & 4 & \ldots & 0  & 0 & 0 & 3 & \ldots & 0\\
\vdots & \vdots & \vdots & \ddots & \vdots & \vdots & \vdots & \vdots & \ddots & \vdots \\
0 & 0 & 0 & \ldots & n+1 & 0 & 0 & 0 & \ldots & n
\end{array}
\right)
$
\\
Осталось обнулить диагональ с отрицательными числами. \\
Рассмотрим матрицу по блокам 3х3. \\
$
\left(
  \begin{array}{rrr}
    i - 1 & -i + 2 & 0 \\
    0     & i      & -i + 1 \\
    0     & 0      & i + 1
  \end{array}
\right)
=
\left(
  \begin{array}{rrr}
    i - 1 & -i + 2 & 0 \\
    0     & i      & -i + 1 \\
    0     & 0      & \frac{i(i + 1)}{i}
  \end{array}
\right)
$ \\
Для того, чтобы обнулить элемент в позиции [2][3] нужно домножить его строчку на элемент в позиции [3][3], поделить на него самого и вычесть из нее следующую строчку. \\
То есть нужно умножить 2 строчку на $\frac{i(i + 1)}{i(-i + 1)}$ и вычесть $\frac{i(i + 1)}{i}$\\
Тогда получим: \\
$
\left(
  \begin{array}{rrr}
    i - 1 & -i + 2 & 0 \\
    0     & \frac{i(i + 1)}{-i + 1}      & 0 \\
    0     & 0      & \frac{i(i + 1)}{i}
  \end{array}
\right)
$
\\
Повторив аналогичные действия для элемента в позиции [1][2] получим: \\
$
\left(
  \begin{array}{rrr}
    \frac{i(i + 1)}{i - 2} & 0 & 0 \\
    0     & \frac{i(i + 1)}{-i + 1}      & 0 \\
    0     & 0      & \frac{i(i + 1)}{i}
  \end{array}
\right)
$ \\
Тогда по индукции получим следующее: (при этом домножим соответственный строчки на -1)\\
$
\left(
\begin{array}{cccccccccc}
\frac{n(n+1)}{1} & 0 & 0 & \ldots & 0  & n(n + 1) & 0 & 0 & \ldots & 0 \\
0 & \frac{n(n+1)}{2} & 0 & \ldots & 0  & 0 & n(n + 1) & 0 & \ldots & 0 \\
0 & 0 & \frac{n(n+1)}{3} & \ldots & 0  & 0 & 0 & n(n + 1) & \ldots & 0 \\
\vdots & \vdots & \vdots & \ddots & \vdots & \vdots & \vdots & \vdots & \ddots & \vdots \\
0 & 0 & 0 & \ldots & n+1 & 0 & 0 & 0 & \ldots & n
\end{array}
\right)
=> \\
\left(
\begin{array}{cccccccccc}
1 & 0 & 0 & \ldots & 0  & 1 & 0 & 0 & \ldots & 0 \\
0 & \frac{1}{2} & 0 & \ldots & 0  & 0 & 1 & 0 & \ldots & 0 \\
0 & 0 & \frac{1}{3} & \ldots & 0  & 0 & 0 & 1 & \ldots & 0 \\
\vdots & \vdots & \vdots & \ddots & \vdots & \vdots & \vdots & \vdots & \ddots & \vdots \\
0 & 0 & 0 & \ldots & \frac{1}{n} & 0 & 0 & 0 & \ldots & \frac{1}{n+1}
\end{array}
\right)
=> \\
\left(
\begin{array}{cccccccccc}
1 & 0 & 0 & \ldots & 0  & 1 & 0 & 0 & \ldots & 0 \\
0 & 1 & 0 & \ldots & 0  & 0 & 2 & 0 & \ldots & 0 \\
0 & 0 & 1 & \ldots & 0  & 0 & 0 & 3 & \ldots & 0 \\
\vdots & \vdots & \vdots & \ddots & \vdots & \vdots & \vdots & \vdots & \ddots & \vdots \\
0 & 0 & 0 & \ldots & 1 & 0 & 0 & 0 & \ldots & n+1
\end{array}
\right)
=> \\
$
$
\left(
\begin{array}{ccccc}
2 & -1 & 0 & \ldots & 0 \\
-1 & 2 & -1 & \ldots & 0 \\
0 & -1 & 2 & \ldots & 0 \\
\vdots & \vdots & \vdots & \ddots & \vdots \\
0 & 0 & 0 & \ldots & 2
\end{array}
\right)^{-1}
=
\left(
  \begin{array}{ccccc}
    1 & 0 & 0 & \ldots & 0 \\
    0 & 2 & 0 & \ldots & 0 \\
    0 & 0 & 3 & \ldots & 0 \\
    \vdots & \vdots & \vdots & \ddots & \vdots \\
    0 & 0 & 0 & \ldots & n + 1
  \end{array}
\right)
$

%%%%%%%%%%%%%%%%%%%%%%%%%%%%%%%%%%%%%%%%%%%%%%%%%%%%
\end{solution}

%%%%%%%%%%%%%%%%%%%%%%%%%%%%%%%%%%%%%%%%%%%%%%%%%%%%
% Задача 4
\begin{problem}{442(b)}
Найти ранг матрицы:
$\left(\begin{array}{rrrr}2 & 1 & 11 & 2\\1 & 0 & 4 & -1\\ 11 & 4 & 56 & 5 \\ 2 & -1 & 5 & -6\end{array}\right)$


\end{problem}
\begin{solution}
%%%%%%%%%%%%%%%%%%%%%%%%%%%%%%%%%%%%%%%%%%%%%%%%%%%%
%% Ваше решение задачи здесь

Поменяем строчки местами: \\
$
\left(
  \begin{array}{rrrr}
    1 & 0 & 4 & -1 \\
    2 & 1 & 11 & 2 \\
    2 & -1 & 5 & -6 \\
    11 & 4 & 56 & 5 \\
  \end{array}
\right)
=>
\left(
  \begin{array}{rrrr}
    1 & 0 & 4 & -1 \\
    0 & 1 & 3 & 4 \\
    0 & -1 & -3 & -4 \\
    0 & 4 & 12 & 16 \\
  \end{array}
\right)
=>
\left(
  \begin{array}{rrrr}
    1 & 0 & 4 & -1 \\
    0 & 1 & 3 & 4 \\
    0 & 0 & 0 & 0 \\
    0 & 0 & 0 & 0 \\
  \end{array}
\right)
$
\\
Значит ранг матрицы - 2

%%%%%%%%%%%%%%%%%%%%%%%%%%%%%%%%%%%%%%%%%%%%%%%%%%%%
\end{solution}

%%%%%%%%%%%%%%%%%%%%%%%%%%%%%%%%%%%%%%%%%%%%%%%%%%%%
% Задача 5
\begin{problem}{442(c)}
Найти ранг матрицы:
$\left(\begin{array}{rrrrr}1 & 0 & 0 & 1 & 4\\0 & 1 & 0 & 2 & 5\\ 0 & 0 & 1 & 3 & 6 \\ 1 & 2 & 3 & 14 & 32\\ 4 & 5 & 6 & 32 & 77\end{array}\right)$

\end{problem}
\begin{solution}
%%%%%%%%%%%%%%%%%%%%%%%%%%%%%%%%%%%%%%%%%%%%%%%%%%%%
%% Ваше решение задачи здесь

$
\left(
  \begin{array}{rrrrr}
    1 & 0 & 0 & 1 & 4 \\
    0 & 1 & 0 & 2 & 5 \\
    0 & 0 & 1 & 3 & 6 \\
    1 & 2 & 3 & 14 & 32 \\
    4 & 5 & 6 & 32 & 77 \\
  \end{array}
\right)
=>
\left(
  \begin{array}{rrrrr}
    1 & 0 & 0 & 1 & 4 \\
    0 & 1 & 0 & 2 & 5 \\
    0 & 0 & 1 & 3 & 6 \\
    0 & 2 & 3 & 13 & 28 \\
    0 & 5 & 6 & 28 & 61 \\
  \end{array}
\right)
=>
\left(
  \begin{array}{rrrrr}
    1 & 0 & 0 & 1 & 4 \\
    0 & 1 & 0 & 2 & 5 \\
    0 & 0 & 1 & 3 & 6 \\
    0 & 0 & 3 & 9 & 18 \\
    0 & 0 & 6 & 18 & 36 \\
  \end{array}
\right)
=> \\
\left(
  \begin{array}{rrrrr}
    1 & 0 & 0 & 1 & 4 \\
    0 & 1 & 0 & 2 & 5 \\
    0 & 0 & 1 & 3 & 6 \\
    0 & 0 & 0 & 0 & 0 \\
    0 & 0 & 0 & 0 & 0 \\
  \end{array}
\right)
$
Значит ранг матрицы - 3

%%%%%%%%%%%%%%%%%%%%%%%%%%%%%%%%%%%%%%%%%%%%%%%%%%%%
\end{solution}

%------------------------------------------------
\end{document}
