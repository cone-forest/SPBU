\documentclass[a4paper, 12pt]{article}
%----------------------------------------------------------------------------------------
%	PACKAGES AND OTHER DOCUMENT CONFIGURATIONS
%----------------------------------------------------------------------------------------
\usepackage[a4paper, total={7in, 10in}]{geometry}
\setlength{\parskip}{0pt}
\setlength{\parindent}{0in}

\usepackage[T2A]{fontenc}% Внутренняя T2A кодировка TeX
\usepackage[utf8]{inputenc}% кодировка файла
\usepackage[russian]{babel}% поддержка переносов в русском языке
\usepackage{amsthm, amsmath, amssymb} % Mathematical typesetting
\usepackage{float} % Improved interface for floating objects
\usepackage{graphicx, multicol} % Enhanced support for graphics
\usepackage{xcolor} % Driver-independent color extensions
\usepackage{mdframed}

\usepackage[yyyymmdd]{datetime} % Uses YEAR-MONTH-DAY format for dates
\renewcommand{\dateseparator}{.} % Sets dateseparator to '.'

\usepackage{fancyhdr} % Headers and footers
\pagestyle{fancy} % All pages have headers and footers
\fancyhead{}\renewcommand{\headrulewidth}{0pt} % Blank out the default header
\fancyfoot[L]{} % Custom footer text
\fancyfoot[C]{} % Custom footer text
\fancyfoot[R]{\thepage} % Custom footer text

\newenvironment{problem}[2][Задача]
    { \begin{mdframed}[backgroundcolor=gray!10] \textbf{#1 #2.} \\}
    {  \end{mdframed}}

\newenvironment{solution}
    {\textit{Решение: }}
    {\noindent\rule{7in}{1.5pt}}

\begin{document}

%-------------------------------
%	TITLE SECTION
%-------------------------------

\fancyhead[C]{}
\hrule \medskip % Upper rule
\begin{minipage}{0.295\textwidth}
\raggedright\footnotesize
Михаил Цуканов \hfill\\
st117303 \hfill\\
st117303@student.spbu.ru
\end{minipage}
\begin{minipage}{0.4\textwidth}
\centering\large
Homework Assignment 6\\
\normalsize
Алгебра и геометрия, 1 семестр\\
\end{minipage}
\begin{minipage}{0.295\textwidth}
\raggedleft
\today\hfill\\
\end{minipage}
\medskip\hrule
\bigskip

%------------------------------------------------
%	CONTENTS
%------------------------------------------------



%%%%%%%%%%%%%%%%%%%%%%%%%%%%%%%%%%%%%%%%%%%%%%%%%%%%
% Задача 1
\begin{problem}{220(e)}
Умножить матрицы:
e)
$\left(
\begin{array}{rrr}
1 & 2 & 1 \\
0 & 1 & 2 \\
3 & 1 & 1 \\
\end{array}
\right)
\cdot
\left(
\begin{array}{rrr}
2 & 3 & 1 \\
-1 & 1 & 0 \\
1 & 2 & -1 \\
\end{array}
\right)
\cdot
\left(
\begin{array}{rrr}
1 & 2 & 1 \\
0 & 1 & 2 \\
3 & 1 & 1 \\
\end{array}
\right)
$

\end{problem}
\begin{solution}
%%%%%%%%%%%%%%%%%%%%%%%%%%%%%%%%%%%%%%%%%%%%%%%%%%%%
%% Ваше решение задачи здесь

Перемножим матрицы по определению \\
$\left(
\begin{array}{rrr}
1 & 2 & 1 \\
0 & 1 & 2 \\
3 & 1 & 1 \\
\end{array}
\right)
\cdot
\left(
\begin{array}{rrr}
2 & 3 & 1 \\
-1 & 1 & 0 \\
1 & 2 & -1 \\
\end{array}
\right)
=
\left(
\begin{array}{rrr}
1 & 9 & 0 \\
1 & 5 & -2 \\
6 & 12 & 2 \\
\end{array}
\right) \\
\left(
\begin{array}{rrr}
1 & 9 & 0 \\
1 & 5 & -2 \\
6 & 12 & 2 \\
\end{array}
\right)
\cdot
\left(
\begin{array}{rrr}
1 & 2 & 1 \\
0 & 1 & 2 \\
3 & 1 & 1 \\
\end{array}
\right)
=
\left(
\begin{array}{rrr}
1 & 11 & 19 \\
-5 & 5 & 9 \\
12 & 26 & 32 \\
\end{array}
\right)
$

%%%%%%%%%%%%%%%%%%%%%%%%%%%%%%%%%%%%%%%%%%%%%%%%%%%%
\end{solution}

%%%%%%%%%%%%%%%%%%%%%%%%%%%%%%%%%%%%%%%%%%%%%%%%%%%%
% Задача 2
\begin{problem}{220(f)}
$\left(
\begin{array}{rrr}
a & b & c \\
c & b & a \\
1 & 1 & 1 \\
\end{array}
\right)
\cdot
\left(
\begin{array}{rrr}
1 & a & c \\
1 & b & b \\
1 & c & a \\
\end{array}
\right)
$

\end{problem}
\begin{solution}
%%%%%%%%%%%%%%%%%%%%%%%%%%%%%%%%%%%%%%%%%%%%%%%%%%%%
%% Ваше решение задачи здесь

Перемножим матрицы по отпределению \\
$\left(
\begin{array}{rrr}
a & b & c \\
c & b & a \\
1 & 1 & 1 \\
\end{array}
\right)
\cdot
\left(
\begin{array}{rrr}
1 & a & c \\
1 & b & b \\
1 & c & a \\
\end{array}
\right)
=
\left(
\begin{array}{rrr}
a + b + c & a^2 + b^2 + c^2 & b^2 + 2ac \\
a + b + c  & b^2 + 2ac & a^2 + b^2 + c^2 \\
3 & a + b + c & a + b + c \\
\end{array}
\right)
$

%%%%%%%%%%%%%%%%%%%%%%%%%%%%%%%%%%%%%%%%%%%%%%%%%%%%
\end{solution}

%%%%%%%%%%%%%%%%%%%%%%%%%%%%%%%%%%%%%%%%%%%%%%%%%%%%
% Задача 3
\begin{problem}{221}
Выполнить действия: \\
\noindent
\begin{tabular}{ll}
a)
$
\left(
\begin{array}{rrr}
2 & 1 & 1\\
3 & 1 & 0\\
0 & 1 & 2
\end{array}
\right)^2
$
\\
\\
b)
$
\left(
\begin{array}{rr}
2 & 1 \\
1 & 3
\end{array}
\right)^3
$
\\
\\
c)
$\left(
\begin{array}{rr}
3 & 2 \\
-4 & -2 \\
\end{array}
\right)^5
$
\\
\\
d)
$
\left(
\begin{array}{rr}
1 & 1 \\
0 & 1 \\
\end{array}
\right)^n
$
\\
\end{tabular}

\end{problem}
\begin{solution}
%%%%%%%%%%%%%%%%%%%%%%%%%%%%%%%%%%%%%%%%%%%%%%%%%%%%
%% Ваше решение задачи здесь

a)
$
\left(
\begin{array}{rrr}
2 & 1 & 1\\
3 & 1 & 0\\
0 & 1 & 2
\end{array}
\right)^2
=
\left(
\begin{array}{rrr}
7 & 4 & 4\\
9 & 4 & 3\\
3 & 3 & 4
\end{array}
\right)
\\
$
\\
\\
b)
$
\left(
\begin{array}{rr}
2 & 1 \\
1 & 3
\end{array}
\right)^3
=
\left(
\begin{array}{rr}
15 & 20 \\
20 & 35
\end{array}
\right)
$
\\
\\
c)
$
\left(
\begin{array}{rr}
3 & 2 \\
-4 & -2 \\
\end{array}
\right)^5
=
\left(
  \begin{array}{rr}
    3 & 2 \\
    -4 & -2 \\
  \end{array}
\right)
\cdot
\left(
  \begin{array}{rr}
    1 & 2 \\
    -4 & -4 \\
  \end{array}
\right)^2
=
\left(
  \begin{array}{rr}
    3 & 2 \\
    -4 & -2 \\
  \end{array}
\right)
\cdot
\left(
  \begin{array}{rr}
    -7 & -6 \\
    12 & 8 \\
  \end{array}
\right)
=
\left(
  \begin{array}{rr}
    3 & -2 \\
    4 & 8 \\
  \end{array}
\right)
$

d)
ММИ: \\
База: \\
$
\left(
  \begin{array}{rr}
    1 & 1 \\
    0 & 1 \\
  \end{array}
\right)^2
=
\left(
  \begin{array}{rr}
    1 & 2 \\
    0 & 1 \\
  \end{array}
\right)
$ - Истина \\
Пусть \\
$
\left(
  \begin{array}{rr}
    1 & 1 \\
    0 & 1 \\
  \end{array}
\right)^n
=
\left(
  \begin{array}{rr}
    1 & n \\
    0 & 1 \\
  \end{array}
\right)
$
\\
Проверим для k = n + 1: \\
$
\left(
  \begin{array}{rr}
    1 & 1 \\
    0 & 1 \\
  \end{array}
\right)^{n + 1}
=
\left(
  \begin{array}{rr}
    1 & n \\
    0 & 1 \\
  \end{array}
\right)
\cdot
\left(
  \begin{array}{rr}
    1 & 1 \\
    0 & 1 \\
  \end{array}
\right)
=
\left(
  \begin{array}{rr}
    1 & n + 1 \\
    0 & 1 \\
  \end{array}
\right)
$
\\
Вывод: \\
$
\left(
  \begin{array}{rr}
    1 & 1 \\
    0 & 1 \\
  \end{array}
\right)^n
=
\left(
  \begin{array}{rr}
    1 & n \\
    0 & 1 \\
  \end{array}
\right)
$ \\

%%%%%%%%%%%%%%%%%%%%%%%%%%%%%%%%%%%%%%%%%%%%%%%%%%%%
\end{solution}

%%%%%%%%%%%%%%%%%%%%%%%%%%%%%%%%%%%%%%%%%%%%%%%%%%%%
% Задача 4
\begin{problem}{224}
Вычислить $A\cdot A'$, где $A=\left( \begin{array}{cccc}3 & 2 & 1& 2\\4 & 1 & 1 & 3 \end{array} \right)$
\end{problem}
\begin{solution}
%%%%%%%%%%%%%%%%%%%%%%%%%%%%%%%%%%%%%%%%%%%%%%%%%%%%
%% Ваше решение задачи здесь

$A=\left( \begin{array}{cccc}3 & 2 & 1& 2\\4 & 1 & 1 & 3 \end{array} \right)$ \\
$
A \cdot A'
=
\left(
  \begin{array}{rr}
    18 & 21 \\
    21 & 27 \\
  \end{array}
\right)
$

%%%%%%%%%%%%%%%%%%%%%%%%%%%%%%%%%%%%%%%%%%%%%%%%%%%%
\end{solution}

%%%%%%%%%%%%%%%%%%%%%%%%%%%%%%%%%%%%%%%%%%%%%%%%%%%%
% Задача 5
\begin{problem}{274}
Вычислить определитель $\left| \begin{array}{rrr}246 & 427 & 327 \\ 1014 & 543 & 443 \\ -342 & 721 & 621 \end{array} \right|$
\end{problem}
\begin{solution}
%%%%%%%%%%%%%%%%%%%%%%%%%%%%%%%%%%%%%%%%%%%%%%%%%%%%
%% Ваше решение задачи здесь

$
\left|
\begin{array}{rrr}
  246 & 427 & 327 \\
  1014 & 543 & 443 \\
  -342 & 721 & 621
\end{array}
\right|
=
246*543*621 - 427*443*342 + 327*1014*721 + 342*543*327 - 246*721*443 - 621*1014*427 =
-29400000
$

%%%%%%%%%%%%%%%%%%%%%%%%%%%%%%%%%%%%%%%%%%%%%%%%%%%%
\end{solution}

%%%%%%%%%%%%%%%%%%%%%%%%%%%%%%%%%%%%%%%%%%%%%%%%%%%%
% Задача 6
\begin{problem}{632(c)}
Вычислить определитель:
c) $\left| \begin{array}{rrr}a & a & a \\ -a & a & x\\-a & -a & x \end{array} \right|$
\end{problem}
\begin{solution}
%%%%%%%%%%%%%%%%%%%%%%%%%%%%%%%%%%%%%%%%%%%%%%%%%%%%
%% Ваше решение задачи здесь

Воспользуемся методом Гауса: \\
$
\left|
\begin{array}{rrr}
  a & a & a \\
  -a & a & x \\
  -a & -a & x
\end{array}
\right|
=
\left|
\begin{array}{rrr}
  a & a & a \\
  0 & 2a & a + x \\
  0 & 0 & a + x
\end{array}
\right|
=
2a^2(a + x)
$

%%%%%%%%%%%%%%%%%%%%%%%%%%%%%%%%%%%%%%%%%%%%%%%%%%%%
\end{solution}

%%%%%%%%%%%%%%%%%%%%%%%%%%%%%%%%%%%%%%%%%%%%%%%%%%%%
% Задача 7
\begin{problem}{232(d)}
Вычислить определитель:
d) $\left| \begin{array}{rrr}1 & 1 & 1 \\ 1 & 2 & 3 \\ 1 & 3 & 6 \end{array} \right|$
\end{problem}
\begin{solution}
%%%%%%%%%%%%%%%%%%%%%%%%%%%%%%%%%%%%%%%%%%%%%%%%%%%%
%% Ваше решение задачи здесь

Воспользуемся методом Гауса: \\
$
\left|
\begin{array}{rrr}
  1 & 1 & 1 \\
  1 & 2 & 3 \\
  1 & 3 & 6
\end{array}
\right|
=
\left|
\begin{array}{rrr}
  1 & 1 & 1 \\
  0 & 1 & 2 \\
  0 & 2 & 5
\end{array}
\right|
=
\left|
\begin{array}{rrr}
  1 & 1 & 1 \\
  0 & 1 & 2 \\
  0 & 0 & 1
\end{array}
\right|
= 1 * 1 * 1 = 1
$


%%%%%%%%%%%%%%%%%%%%%%%%%%%%%%%%%%%%%%%%%%%%%%%%%%%%
\end{solution}

%%%%%%%%%%%%%%%%%%%%%%%%%%%%%%%%%%%%%%%%%%%%%%%%%%%%
% Задача 8
\begin{problem}{232(f)}
Вычислить определитель:
f) $\left| \begin{array}{rrr}1 & 1 & 1 \\ 1 & \omega & \omega\\1 & \omega^2 & \omega \end{array} \right|$,
где $\omega=\cos\frac{2\pi}{3}+i\sin\frac{2\pi}{3}$
\end{problem}
\begin{solution}
%%%%%%%%%%%%%%%%%%%%%%%%%%%%%%%%%%%%%%%%%%%%%%%%%%%%
%% Ваше решение задачи здесь

Воспользуемся методом Гауса: \\
$
\left|
\begin{array}{rrr}
  1 & 1 & 1 \\
  1 & \omega & \omega \\
  1 & \omega^2 & \omega
\end{array}
\right|
=
\left|
\begin{array}{rrr}
  1 & 1 & 1 \\
  0 & \omega - 1 & \omega - 1 \\
  0 & \omega^2 - 1 & \omega - 1
\end{array}
\right|
$
= Константа $\omega + 1$
$
=
\left|
\begin{array}{rrr}
  1 & 1 & 1 \\
  0 & \omega - 1 & \omega - 1 \\
  0 & 0 & -\omega(\omega - 1)
\end{array}
\right|
= -\omega(\omega - 1)^2
$ \\
Вычислим значение, подставив $\omega$ \\
$
-\omega(\omega - 1)^2 = -\omega^3 + 2\omega^2 - \omega = (1, \pi) + (2, \frac{4\pi}{3}) + (1, \frac{5\pi}{3}) \\
$
Представим в векторном виде и сложим: \\
Векторы $(1, \pi), (1, \frac{5\pi}{3})$ и их сумма образуют равнобедренный треугольник \\ с углом при вершине $\frac{\pi}{3}$ (равносторонний треугольник) \\
Значит их сумма - $(1, \frac{4\pi}{3})$, а общая сумма (и определитель) - $(3, \frac{4\pi}{3})$.

%%%%%%%%%%%%%%%%%%%%%%%%%%%%%%%%%%%%%%%%%%%%%%%%%%%%
\end{solution}

%%%%%%%%%%%%%%%%%%%%%%%%%%%%%%%%%%%%%%%%%%%%%%%%%%%%
% Задача 9
\begin{problem}{284}
$\left| \begin{array}{rrr}x & y & x+y \\ y & x+y & x \\ x+y & x & y \end{array} \right|$
\end{problem}
\begin{solution}
%%%%%%%%%%%%%%%%%%%%%%%%%%%%%%%%%%%%%%%%%%%%%%%%%%%%
%% Ваше решение задачи здесь

Разложим определитель по 1 строчке: \\
$
\left|
\begin{array}{rrr}
  x & y & x+y \\
  y & x+y & x \\
  x+y & x & y
\end{array}
\right|
=
x
\left|
\begin{array}{rrr}
  x+y & x \\
  x & y
\end{array}
\right|
-
y
\left|
\begin{array}{rrr}
  y & x \\
  x+y & y
\end{array}
\right|
+
(x + y)
\left|
\begin{array}{rrr}
  y & x+y \\
  x+y & x
\end{array}
\right|
=
\\
\\
x(xy + y^2 -x^2) - y(y^2 - x^2 - xy) + (x + y)(xy - x^2 - 2xy - y^2) =$ Раскроем скобки $= \\
-2(x^3 + y^3)
$

%%%%%%%%%%%%%%%%%%%%%%%%%%%%%%%%%%%%%%%%%%%%%%%%%%%%
\end{solution}

%------------------------------------------------
\end{document}
