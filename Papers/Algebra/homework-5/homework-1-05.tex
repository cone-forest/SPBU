\documentclass[a4paper, 12pt]{article}
%----------------------------------------------------------------------------------------
%	PACKAGES AND OTHER DOCUMENT CONFIGURATIONS
%----------------------------------------------------------------------------------------
\usepackage[a4paper, total={7in, 10in}]{geometry}
\setlength{\parskip}{0pt}
\setlength{\parindent}{0in}

\usepackage[T2A]{fontenc}% Внутренняя T2A кодировка TeX
\usepackage[utf8]{inputenc}% кодировка файла
\usepackage[russian]{babel}% поддержка переносов в русском языке
\usepackage{amsthm, amsmath, amssymb} % Mathematical typesetting
\usepackage{float} % Improved interface for floating objects
\usepackage{graphicx, multicol} % Enhanced support for graphics
\usepackage{xcolor} % Driver-independent color extensions
\usepackage{mdframed}
\usepackage{polynom}

\usepackage[yyyymmdd]{datetime} % Uses YEAR-MONTH-DAY format for dates
\renewcommand{\dateseparator}{.} % Sets dateseparator to '.'

\usepackage{fancyhdr} % Headers and footers
\pagestyle{fancy} % All pages have headers and footers
\fancyhead{}\renewcommand{\headrulewidth}{0pt} % Blank out the default header
\fancyfoot[L]{} % Custom footer text
\fancyfoot[C]{} % Custom footer text
\fancyfoot[R]{\thepage} % Custom footer text

\newenvironment{problem}[2][Задача]
    { \begin{mdframed}[backgroundcolor=gray!10] \textbf{#1 #2.} \\}
    {  \end{mdframed}}

\newenvironment{solution}
    {\textit{Решение: }}
    {\noindent\rule{7in}{1.5pt}}

\begin{document}

%-------------------------------
%	TITLE SECTION
%-------------------------------

\fancyhead[C]{}
\hrule \medskip % Upper rule
\begin{minipage}{0.295\textwidth}
\raggedright\footnotesize
Цуканов Михаил \hfill\\
st117303 \hfill\\
st117303@student.spbu.ru
\end{minipage}
\begin{minipage}{0.4\textwidth}
\centering\large
Homework Assignment 5\\
\normalsize
Алгебра и геометрия, 1 семестр\\
\end{minipage}
\begin{minipage}{0.295\textwidth}
\raggedleft
\today\hfill\\
\end{minipage}
\medskip\hrule
\bigskip

%------------------------------------------------
%	CONTENTS
%------------------------------------------------



%%%%%%%%%%%%%%%%%%%%%%%%%%%%%%%%%%%%%%%%%%%%%%%%%%%%
% Задача 1
\begin{problem}{615}
Сумма двух корней уравнения $2x^3-x^2-7x+\lambda=0$ равна 1. Определить $\lambda$.
\end{problem}
\begin{solution}
%%%%%%%%%%%%%%%%%%%%%%%%%%%%%%%%%%%%%%%%%%%%%%%%%%%%
%% Ваше решение задачи здесь

$
2x^3-x^2-7x+\lambda=0 \\
x^3-\frac{1}{2}x^2-3\frac{1}{2}x+\frac{\lambda}{2}=0 \\
\\
$
По т. Виета: \\
$
x_0 + x_1 + x_2 = -\frac{1}{2} \\
x_2 = -1\frac{1}{2} \\
\\
(x_0 + x_1)x_2 + x_0*x_1 = -3\frac{1}{2} \\
-1\frac{1}{2} + x_0*x_1 = -3\frac{1}{2} \\
x_0*x_1 = -2 \\
\\
$
тогда \\
$
\lambda = x_0*x_1*x_2 = -2 * -1\frac{1}{2} = 3
$

%%%%%%%%%%%%%%%%%%%%%%%%%%%%%%%%%%%%%%%%%%%%%%%%%%%%
\end{solution}

%%%%%%%%%%%%%%%%%%%%%%%%%%%%%%%%%%%%%%%%%%%%%%%%%%%%
% Задача 2
\begin{problem}{618}
Решить уравнение $x^n+a_1 x^{n-1}+a_2 x^{n-2}+...+a_n = 0$, зная коэффициенты $a_1$ и $a_2$, и зная, что корни его образуют арифметическую прогрессию.
\end{problem}
\begin{solution}
%%%%%%%%%%%%%%%%%%%%%%%%%%%%%%%%%%%%%%%%%%%%%%%%%%%%
%% Ваше решение задачи здесь

По теореме Виета мы знаем, что коэффициент при $\displaystyle x^{n-1}$ равен сумме всех корней, а так как корни образуют арифметическую прогрессию, то этот коэффициент с минусом равен формуле суммы арифметической прогресии, то есть $\displaystyle -a_1 = \frac{x_1 + x_n}{2}n$
\\
Разность прогресси по формуле равна $\displaystyle\frac{x_1 + x_n}{n + 1}$
\\
$\displaystyle -a_1 = \frac{x_1 + x_n}{2}n \Rightarrow -\frac{2a_1}{n} = x_1 + x_n \Rightarrow -\frac{2a_1}{n(n + 1)} = \frac{x_1 + x_n}{n + 1}$
\\
$\displaystyle d = -\frac{2a_1}{n(n + 1)}$
\\
$\displaystyle S = \frac{2x_1 + d(n - 1)}{2}n = -a_1$
\\
$\displaystyle - \frac{2a_1}{n} = 2x_1 + d(n - 1) \Rightarrow -\frac{2a_1}{n} - d(n - 1) = 2x_1 \Rightarrow x_1 = \frac{-\frac{2a_1}{n} - d(n - 1)}{2}$
\\
$\displaystyle x_1 = \frac{-\frac{2a_1}{n} + \frac{2a_1(n - 1)}{n(n + 1)}}{2} = \frac{-2a_1(n + 1) + 2a_1(n - 1)}{2n(n + 1)} = \frac{-2a_1}{n(n + 1)}$
\\
$\displaystyle x_n = x_1 + (n - 1)d = \frac{-2a_1}{n(n + 1)} + (n - 1)\biggl(-\frac{2a_1}{n(n + 1)}\biggr) = \frac{-2a_1 - 2a_1(n - 1)}{n(n + 1)} = -\frac{2a_1}{n + 1}$
\\
Ответ: $\displaystyle -\frac{2a_1}{n + 1}$

%%%%%%%%%%%%%%%%%%%%%%%%%%%%%%%%%%%%%%%%%%%%%%%%%%%%
\end{solution}

%%%%%%%%%%%%%%%%%%%%%%%%%%%%%%%%%%%%%%%%%%%%%%%%%%%%
% Задача 3
\begin{problem}{621}
Составить уравнение $6$-й степени, имеющее корни
$\alpha$, $\frac{1}{\alpha}$, $1-\alpha$, $\frac{1}{1-\alpha}$, $1-\frac{1}{\alpha}$, $\frac{1}{1-\frac{1}{\alpha}}$.
\end{problem}
\begin{solution}
%%%%%%%%%%%%%%%%%%%%%%%%%%%%%%%%%%%%%%%%%%%%%%%%%%%%
%% Ваше решение задачи здесь

$
(x-\alpha)
(x-\frac{1}{\alpha})
(x-1+\alpha)
(x-\frac{1}{1-\alpha})
(x-1+\frac{1}{\alpha})
(x-\frac{1}{1-\frac{1}{\alpha}}) = \\
x^6+ax^5+bx^4+cx^3+dx^2+ex^1+1 = \\
x^6 + 3x^5 +\frac{-\alpha^6+5\alpha^5-8\alpha^4+10\alpha^3-15\alpha^2+11\alpha-3}{(\alpha^2-\alpha)^2}x^4 + \frac{2\alpha^6-6\alpha^5+5\alpha^4+5\alpha^2-6\alpha+2}{(\alpha^2-\alpha)^2}x^3 + \frac{-\alpha^6+5\alpha^5-8\alpha^4+10\alpha^3-15\alpha^2+11\alpha-3}{(\alpha^2-\alpha)^2}x^2 + 3x + 1
$

%%%%%%%%%%%%%%%%%%%%%%%%%%%%%%%%%%%%%%%%%%%%%%%%%%%%
\end{solution}

%%%%%%%%%%%%%%%%%%%%%%%%%%%%%%%%%%%%%%%%%%%%%%%%%%%%
% Задача 4
\begin{problem}{552(a)}
Пользуясь схемой Горнера, разложить $\displaystyle\frac{x^3-x+1}{(x-2)^5}$ на простейшие дроби.
\end{problem}
\begin{solution}
%%%%%%%%%%%%%%%%%%%%%%%%%%%%%%%%%%%%%%%%%%%%%%%%%%%%
%% Ваше решение задачи здесь

\polyhornerscheme[showbase=bottom, showmiddlerow=false, x=2]{x^3-x+1} \\
\polyhornerscheme[showbase=bottom, showmiddlerow=false, x=2]{x^2+2x+3} \\
\polyhornerscheme[showbase=bottom, showmiddlerow=false, x=2]{x^1+4} \\

Получили разложение $x^3-x+1$ на $x-2$. Поделим его на $(x-2)^5$. \\
Итого \\
$
\displaystyle\frac{x^3-x+1}{(x-2)^5} =
\frac{1}{(x-2)^2} + \frac{6}{(x-2)^3} + \frac{11}{(x-2)^4} + \frac{7}{(x-2)^5} \\
$

%%%%%%%%%%%%%%%%%%%%%%%%%%%%%%%%%%%%%%%%%%%%%%%%%%%%
\end{solution}

%%%%%%%%%%%%%%%%%%%%%%%%%%%%%%%%%%%%%%%%%%%%%%%%%%%%
% Задача 5
\begin{problem}{626(b)}
Разложить на простейшие дроби над полем $R$:
$\frac{x^2}{x^4-16}$
\end{problem}
\begin{solution}
%%%%%%%%%%%%%%%%%%%%%%%%%%%%%%%%%%%%%%%%%%%%%%%%%%%%
%% Ваше решение задачи здесь

$
\frac{x^2}{x^4-16} = \frac{x^2}{(x^2+4)(x - 2)(x + 2)} =
\frac{ax + b}{x^2+4} + \frac{c}{x - 2} + \frac{d}{x + 2} =
\frac{(a+c+d)x^3 + (b+2c-2d)x^2 + (-4a+4c+4d)x + (-4b+8c-8d)}{x^4-16} \\
\left\{
  \begin{array}{rrrr}
    a + c + d = 0 \\
    b + 2c -2d = 1 \\
    -a + c + d = 0 \\
    -b + 2c - 2d = 0
  \end{array}
\right.
\left\{
  \begin{array}{rrrr}
    a = 0 \\
    b = \frac{1}{2} \\
    c = \frac{1}{8} \\
    d = -\frac{1}{8}
  \end{array}
\right. \\
\frac{x^2}{x^4-16} =
\frac{\frac{1}{2}}{x^2+4} + \frac{\frac{1}{8}}{x - 2} + \frac{-\frac{1}{8}}{x + 2}
$

%%%%%%%%%%%%%%%%%%%%%%%%%%%%%%%%%%%%%%%%%%%%%%%%%%%%
\end{solution}

%%%%%%%%%%%%%%%%%%%%%%%%%%%%%%%%%%%%%%%%%%%%%%%%%%%%
% Задача 6
\begin{problem}{627(b)}
Разложить на простейшие дроби над полем $R$:
$\frac{2x-1}{x(x+1)^2(x^2+x+1)^2}$.
\end{problem}
\begin{solution}
%%%%%%%%%%%%%%%%%%%%%%%%%%%%%%%%%%%%%%%%%%%%%%%%%%%%
%% Ваше решение задачи здесь

$\displaystyle \frac{2x-1}{x(x+1)^2(x^2+x+1)^2} = \frac{a}{x} + \frac{b}{x + 1} + \frac{c}{(x + 1)^2} + \frac{dx + e}{x^2 + x + 1} + \frac{fx + g}{(x^2 + x + 1)^2}$
\\
$\displaystyle 2x-1 = a(x+1)^2(x^2+x+1)^2 + bx(x+1)(x^2+x+1)^2 + cx(x^2+x+1)^2 + (dx+e)x(x+1)^2(x^2+x+1)+(fx+g)x(x+1)^2$
\\
$\displaystyle 2x-1 = x^6(a+b+d)+x^5(4a+3b+c+3d+e)+x^4(8a+5b+2c+4d+f)+x^3(10a+5b+3c+3d+4e+2f+g)+x^2(8a+3b+2c+d+3e+f+2g)+x(4a+b+c+e+g)+a$
\\
$
\left\{
\begin{aligned}
    a + b + d &= 0,\\
    4a + 3b + c + 3d + e &= 0,\\
    8a + 5b + 2c + 4d + f &= 0,\\
    10a + 5b + 3c + 3d + 4e + 2f + g &= 0,\\
    8a + 3b + 2c + d + 3e + f + 2g &= 0,\\
    4a + b + c + e + g &= 2,\\
    a &= -1
\end{aligned}
\right.
$
\\
$
\left\{
\begin{aligned}
    a &= -1,\\
    b &= 7,\\
    c &= 3,\\
    d &= -6,\\
    e &= -2,\\
    f &= -3,\\
    g &= -2
\end{aligned}
\right.
$
\\
$\displaystyle \frac{2x-1}{x(x+1)^2(x^2+x+1)^2} = -\frac{1}{x} + \frac{7}{x + 1} + \frac{3}{(x+1)^2} - \frac{6x + 2}{x^2 + x + 1} - \frac{3x + 2}{(x^2 + x + 1)^2}$

%%%%%%%%%%%%%%%%%%%%%%%%%%%%%%%%%%%%%%%%%%%%%%%%%%%%
\end{solution}

%%%%%%%%%%%%%%%%%%%%%%%%%%%%%%%%%%%%%%%%%%%%%%%%%%%%
% Задача 7
\begin{problem}{624(d)}
Разложить на простейшие дроби над полем $C$: $\frac{x^2}{x^4-1}$.
\end{problem}
\begin{solution}
%%%%%%%%%%%%%%%%%%%%%%%%%%%%%%%%%%%%%%%%%%%%%%%%%%%%
%% Ваше решение задачи здесь

$\displaystyle \frac{x^2}{x^4-1} = \frac{a}{x - 1} + \frac{b}{x + 1} + \frac{c}{x - i} + \frac{d}{x + i}$
\\
$\displaystyle x^4-1 = (x^2 - 1)(x^2 + 1) = (x - 1)(x + 1)(x - i)(x + i)$
\\
$\displaystyle x^2 = a(x - i)(x + i)(x + 1) + b(x - i)(x + i)(x - 1) + c(x - 1)(x + 1)(x + i) + d(x - 1)(x + 1)(x - i)$
\\
$\displaystyle\text{при x = 1: $4a = 1\rightarrow a = \frac{1}{4}$}$
\\
$\displaystyle\text{при x = -1: $-4b = 1\rightarrow b = -\frac{1}{4}$}$
\\
$\displaystyle\text{при x = i: $-4ic = -1\rightarrow c = -\frac{i}{4}$}$
\\
$\displaystyle\text{при x = -i: $4id = -1\rightarrow d = \frac{i}{4}$}$
\\
$\displaystyle \frac{x^2}{x^4-1} = \frac{1}{4(x - 1)} - \frac{1}{4(x + 1)} - \frac{i}{4(x - i)} + \frac{i}{4(x + i)}$
\\
Ответ: $\displaystyle \frac{1}{4(x - 1)} - \frac{1}{4(x + 1)} - \frac{i}{4(x - i)} + \frac{i}{4(x + i)}$

%%%%%%%%%%%%%%%%%%%%%%%%%%%%%%%%%%%%%%%%%%%%%%%%%%%%
\end{solution}

%%%%%%%%%%%%%%%%%%%%%%%%%%%%%%%%%%%%%%%%%%%%%%%%%%%%
% Задача 8
\begin{problem}{625(c)}
Разложить на простейшие дроби над полем $C$: $\frac{5x^2+6x-23}{(x-1)^3(x+1)^2(x-2)}$.
\end{problem}
\begin{solution}
%%%%%%%%%%%%%%%%%%%%%%%%%%%%%%%%%%%%%%%%%%%%%%%%%%%%
%% Ваше решение задачи здесь

$\displaystyle \frac{5x^2+6x-23}{(x-1)^3(x+1)^2(x-2)} = \frac{a}{x-1} + \frac{b}{(x-1)^2} + \frac{c}{(x-1)^3}+\frac{d}{x+1} + \frac{e}{(x+1)^2} + \frac{f}{x-2}$
\\
$\displaystyle 5x^2+6x-23 = a(x-1)^2(x+1)^2(x-2) + b(x-1)(x+1)^2(x-2) + c(x+1)^2(x-2) + d(x-1)^3(x+1)(x-2) + e(x-1)^3(x-2) + f(x-1)^3(x+1)^2$
\\
$\displaystyle \text{при x = 1: } -4c=-12 \rightarrow c = 3$
\\
$\displaystyle \text{при x = -1: } 24e=-24 \rightarrow e = -1$
\\
$\displaystyle \text{при x = 2: } 9f=9 \rightarrow f = 1$
\\
$
\left\{
\begin{aligned}
    a + d &= -1,\\
    2a - b + 4d &= -2,\\
    2a + b - 4d &= 6,\\
    4a - 3b + 2d &= 12,\\
    a + b - 5d &= 7,\\
    2a - 2b - 2d &= 14
\end{aligned}
\right.
$
\\
$
\left\{
\begin{aligned}
    b &= 2a + 4d + 2,\\
    2a + b - 4d &= 6
\end{aligned}
\right.
$
\\
$\displaystyle 2a + 2a + 4d + 2 - 4d = 6 \rightarrow 4a = 4 \rightarrow a = 1$
\\
$\displaystyle d = -2$
\\
$\displaystyle b = 2 - 8 + 2 = -4$
\\
$\displaystyle \frac{5x^2+6x-23}{(x-1)^3(x+1)^2(x-2)} = \frac{1}{x - 1} - \frac{4}{(x-1)^2} + \frac{3}{(x-1)^3} - \frac{2}{x+1} - \frac{1}{(x+2)^2} + \frac{1}{x-2}$

%%%%%%%%%%%%%%%%%%%%%%%%%%%%%%%%%%%%%%%%%%%%%%%%%%%%
\end{solution}


%------------------------------------------------
\end{document}
