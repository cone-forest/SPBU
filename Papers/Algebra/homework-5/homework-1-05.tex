\documentclass[a4paper, 12pt]{article}
%----------------------------------------------------------------------------------------
%	PACKAGES AND OTHER DOCUMENT CONFIGURATIONS
%----------------------------------------------------------------------------------------
\usepackage[a4paper, total={7in, 10in}]{geometry}
\setlength{\parskip}{0pt}
\setlength{\parindent}{0in}

\usepackage[T2A]{fontenc}% Внутренняя T2A кодировка TeX
\usepackage[utf8]{inputenc}% кодировка файла
\usepackage[russian]{babel}% поддержка переносов в русском языке
\usepackage{amsthm, amsmath, amssymb} % Mathematical typesetting
\usepackage{float} % Improved interface for floating objects
\usepackage{graphicx, multicol} % Enhanced support for graphics
\usepackage{xcolor} % Driver-independent color extensions
\usepackage{mdframed}
\usepackage{polynom}

\usepackage[yyyymmdd]{datetime} % Uses YEAR-MONTH-DAY format for dates
\renewcommand{\dateseparator}{.} % Sets dateseparator to '.'

\usepackage{fancyhdr} % Headers and footers
\pagestyle{fancy} % All pages have headers and footers
\fancyhead{}\renewcommand{\headrulewidth}{0pt} % Blank out the default header
\fancyfoot[L]{} % Custom footer text
\fancyfoot[C]{} % Custom footer text
\fancyfoot[R]{\thepage} % Custom footer text

\newenvironment{problem}[2][Задача]
    { \begin{mdframed}[backgroundcolor=gray!10] \textbf{#1 #2.} \\}
    {  \end{mdframed}}

\newenvironment{solution}
    {\textit{Решение: }}
    {\noindent\rule{7in}{1.5pt}}

\begin{document}

%-------------------------------
%	TITLE SECTION
%-------------------------------

\fancyhead[C]{}
\hrule \medskip % Upper rule
\begin{minipage}{0.295\textwidth} 
\raggedright\footnotesize
Цуканов Михаил \hfill\\   
st117303 \hfill\\
st117303@student.spbu.ru
\end{minipage}
\begin{minipage}{0.4\textwidth} 
\centering\large 
Homework Assignment 5\\ 
\normalsize 
Алгебра и геометрия, 1 семестр\\ 
\end{minipage}
\begin{minipage}{0.295\textwidth} 
\raggedleft
\today\hfill\\
\end{minipage}
\medskip\hrule 
\bigskip

%------------------------------------------------
%	CONTENTS
%------------------------------------------------



%%%%%%%%%%%%%%%%%%%%%%%%%%%%%%%%%%%%%%%%%%%%%%%%%%%%
% Задача 1
\begin{problem}{615}
Сумма двух корней уравнения $2x^3-x^2-7x+\lambda=0$ равна 1. Определить $\lambda$.
\end{problem}
\begin{solution}
%%%%%%%%%%%%%%%%%%%%%%%%%%%%%%%%%%%%%%%%%%%%%%%%%%%%
%% Ваше решение задачи здесь

$
2x^3-x^2-7x+\lambda=0 \\
x^3-\frac{1}{2}x^2-3\frac{1}{2}x+\frac{\lambda}{2}=0 \\
\\
$
По т. Виета: \\
$
x_0 + x_1 + x_2 = -\frac{1}{2} \\
x_2 = -1\frac{1}{2} \\
\\
(x_0 + x_1)x_2 + x_0*x_1 = -3\frac{1}{2} \\
-1\frac{1}{2} + x_0*x_1 = -3\frac{1}{2} \\
x_0*x_1 = -2 \\
\\
$
тогда \\
$
\lambda = x_0*x_1*x_2 = -2 * -1\frac{1}{2} = 3
$

%%%%%%%%%%%%%%%%%%%%%%%%%%%%%%%%%%%%%%%%%%%%%%%%%%%%
\end{solution} 

%%%%%%%%%%%%%%%%%%%%%%%%%%%%%%%%%%%%%%%%%%%%%%%%%%%%
% Задача 2
\begin{problem}{618}
Решить уравнение $x^n+a_1 x^{n-1}+a_2 x^{n-2}+...+a_n = 0$, зная коэффициенты $a_1$ и $a_2$, и зная, что корни его образуют арифметическую прогрессию.
\end{problem}
\begin{solution}
%%%%%%%%%%%%%%%%%%%%%%%%%%%%%%%%%%%%%%%%%%%%%%%%%%%%
%% Ваше решение задачи здесь

Воспользуемся теоремой Виета: \\
$
\sum_{i=1}^{n}x_i = -a_1 \\
\sum_{i=1, j=1, \neq{i}{j}}^{n}x_i*x_j = a_2 \\
$
Тогда  \\
$-a_1$ - сумма арифметической прогрессии \\
$a_2$ - сумма таблицы умножения этой прогрессии без главной диагонали \\
$
-a_1 = x_0*n + d\frac{n(n-1)}{2} \\
a_2 =
\sum_{i=0, j=0, \neq{i}{j}}^{n - 1}(x_0 + jd)*(x_0 + id) =
\sum_{i=0, j=0, \neq{i}{j}}^{n - 1}(x_0^2 + dx_0(i + j) + ijd^2) = \\
n(n - 1)x_0^2 + dx_0\sum_{i=0, j=0, \neq{i}{j}}^{n - 1}(i + j) + d^2\sum_{i=1, j=1, \neq{i}{j}}^{n - 1}(ij)\\
$
Осталось узнать суммы таблиц суммы и умножения без главных диагоналей и решить получившуюся систему. \\
Сумма таблицы сложения без главной диагонали \\
$n * n(n + 1) - \frac{n(n+1)}{2} = (n-\frac{1}{2})n(n+1)$
Сумма таблицы умножения без главной диагонали \\
$
\sum_{i = 1}^{n}\frac{n(n+1)}{2}i - \sum_{i = 1}^{n}i^2 =
\frac{n^2(n+1)^2}{4} - \frac{2n(n+1)(2n+1)}{3} = 
$

%%%%%%%%%%%%%%%%%%%%%%%%%%%%%%%%%%%%%%%%%%%%%%%%%%%%
\end{solution} 

%%%%%%%%%%%%%%%%%%%%%%%%%%%%%%%%%%%%%%%%%%%%%%%%%%%%
% Задача 3
\begin{problem}{621}
Составить уравнение $6$-й степени, имеющее корни 
$\alpha$, $\frac{1}{\alpha}$, $1-\alpha$, $\frac{1}{1-\alpha}$, $1-\frac{1}{\alpha}$, $\frac{1}{1-\frac{1}{\alpha}}$.
\end{problem}
\begin{solution}
%%%%%%%%%%%%%%%%%%%%%%%%%%%%%%%%%%%%%%%%%%%%%%%%%%%%
%% Ваше решение задачи здесь



%%%%%%%%%%%%%%%%%%%%%%%%%%%%%%%%%%%%%%%%%%%%%%%%%%%%
\end{solution} 

%%%%%%%%%%%%%%%%%%%%%%%%%%%%%%%%%%%%%%%%%%%%%%%%%%%%
% Задача 4
\begin{problem}{552(a)}
Пользуясь схемой Горнера, разложить $\displaystyle\frac{x^3-x+1}{(x-2)^5}$ на простейшие дроби.
\end{problem}
\begin{solution}
%%%%%%%%%%%%%%%%%%%%%%%%%%%%%%%%%%%%%%%%%%%%%%%%%%%%
%% Ваше решение задачи здесь

\polyhornerscheme[showbase=bottom, showmiddlerow=false, x=2]{x^3-x+1} \\
\polyhornerscheme[showbase=bottom, showmiddlerow=false, x=2]{x^2+2x+3} \\
\polyhornerscheme[showbase=bottom, showmiddlerow=false, x=2]{x^1+4} \\

Получили разложение $x^3-x+1$ на $x-2$. Поделим его на $(x-2)^5$. \\
Итого \\
$
\displaystyle\frac{x^3-x+1}{(x-2)^5} = 
\frac{1}{(x-2)^2} + \frac{6}{(x-2)^3} + \frac{11}{(x-2)^4} + \frac{7}{(x-2)^5} \\
$

%%%%%%%%%%%%%%%%%%%%%%%%%%%%%%%%%%%%%%%%%%%%%%%%%%%%
\end{solution} 

%%%%%%%%%%%%%%%%%%%%%%%%%%%%%%%%%%%%%%%%%%%%%%%%%%%%
% Задача 5
\begin{problem}{626(b)}
Разложить на простейшие дроби над полем $R$:
$\frac{x^2}{x^4-16}$
\end{problem}
\begin{solution}
%%%%%%%%%%%%%%%%%%%%%%%%%%%%%%%%%%%%%%%%%%%%%%%%%%%%
%% Ваше решение задачи здесь

$
\frac{x^2}{x^4-16} = \frac{x^2}{(x^2+4)(x - 2)(x + 2)} = 
\frac{ax + b}{x^2+4} + \frac{c}{x - 2} + \frac{d}{x + 2} = 
$

%%%%%%%%%%%%%%%%%%%%%%%%%%%%%%%%%%%%%%%%%%%%%%%%%%%%
\end{solution} 

%%%%%%%%%%%%%%%%%%%%%%%%%%%%%%%%%%%%%%%%%%%%%%%%%%%%
% Задача 6
\begin{problem}{627(b)}
Разложить на простейшие дроби над полем $R$:
$\frac{2x-1}{x(x+1)^2(x^2+x+1)^2}$.
\end{problem}
\begin{solution}
%%%%%%%%%%%%%%%%%%%%%%%%%%%%%%%%%%%%%%%%%%%%%%%%%%%%
%% Ваше решение задачи здесь



%%%%%%%%%%%%%%%%%%%%%%%%%%%%%%%%%%%%%%%%%%%%%%%%%%%%
\end{solution} 

%%%%%%%%%%%%%%%%%%%%%%%%%%%%%%%%%%%%%%%%%%%%%%%%%%%%
% Задача 7
\begin{problem}{624(d)}
Разложить на простейшие дроби над полем $C$: $\frac{x^2}{x^4-1}$.
\end{problem}
\begin{solution}
%%%%%%%%%%%%%%%%%%%%%%%%%%%%%%%%%%%%%%%%%%%%%%%%%%%%
%% Ваше решение задачи здесь



%%%%%%%%%%%%%%%%%%%%%%%%%%%%%%%%%%%%%%%%%%%%%%%%%%%%
\end{solution} 

%%%%%%%%%%%%%%%%%%%%%%%%%%%%%%%%%%%%%%%%%%%%%%%%%%%%
% Задача 8
\begin{problem}{625(c)}
Разложить на простейшие дроби над полем $C$: $\frac{5x^2+6x-23}{(x-1)^3(x+1)^2(x-2)}$.
\end{problem}
\begin{solution}
%%%%%%%%%%%%%%%%%%%%%%%%%%%%%%%%%%%%%%%%%%%%%%%%%%%%
%% Ваше решение задачи здесь



%%%%%%%%%%%%%%%%%%%%%%%%%%%%%%%%%%%%%%%%%%%%%%%%%%%%
\end{solution} 


%------------------------------------------------
\end{document}
