\documentclass[a4paper, 12pt]{article}
%----------------------------------------------------------------------------------------
%	PACKAGES AND OTHER DOCUMENT CONFIGURATIONS
%----------------------------------------------------------------------------------------
\usepackage[a4paper, total={7in, 10in}]{geometry}
\setlength{\parskip}{0pt}
\setlength{\parindent}{0in}

\usepackage[T2A]{fontenc}% Внутренняя T2A кодировка TeX
\usepackage[utf8]{inputenc}% кодировка файла
\usepackage[russian]{babel}% поддержка переносов в русском языке
\usepackage{amsthm, amsmath, amssymb} % Mathematical typesetting
\usepackage{float} % Improved interface for floating objects
\usepackage{graphicx, multicol} % Enhanced support for graphics
\usepackage{xcolor} % Driver-independent color extensions
\usepackage{mdframed}

\usepackage[yyyymmdd]{datetime} % Uses YEAR-MONTH-DAY format for dates
\renewcommand{\dateseparator}{.} % Sets dateseparator to '.'

\usepackage{fancyhdr} % Headers and footers
\pagestyle{fancy} % All pages have headers and footers
\fancyhead{}\renewcommand{\headrulewidth}{0pt} % Blank out the default header
\fancyfoot[L]{} % Custom footer text
\fancyfoot[C]{} % Custom footer text
\fancyfoot[R]{\thepage} % Custom footer text

\newenvironment{problem}[2][Задача]
    { \begin{mdframed}[backgroundcolor=gray!10] \textbf{#1 #2.} \\}
    {  \end{mdframed}}

\newenvironment{solution}
    {\textit{Решение: }}
    {\noindent\rule{7in}{1.5pt}}

\begin{document}

%-------------------------------
%	TITLE SECTION
%-------------------------------

\fancyhead[C]{}
\hrule \medskip % Upper rule
\begin{minipage}{0.295\textwidth}
\raggedright\footnotesize
Михаил Цуканов \hfill\\
st117303 \hfill\\
st117303@student.spbu.ru
\end{minipage}
\begin{minipage}{0.4\textwidth}
\centering\large
Homework Assignment 12\\
\normalsize
Алгебра и геометрия, 1 семестр\\
\end{minipage}
\begin{minipage}{0.295\textwidth}
\raggedleft
\today\hfill\\
\end{minipage}
\medskip\hrule
\bigskip

%------------------------------------------------
%	CONTENTS
%------------------------------------------------



%%%%%%%%%%%%%%%%%%%%%%%%%%%%%%%%%%%%%%%%%%%%%%%%%%%%
% Задача 1
\begin{problem}{10}
Дан четырёхугольник $ABCD$. Найти такую точку $M$, чтобы
$
\overrightarrow{MA} + \overrightarrow{MB} + \overrightarrow{MC} + \overrightarrow{MD} = \vec{0}
$
\end{problem}
\begin{solution}
%%%%%%%%%%%%%%%%%%%%%%%%%%%%%%%%%%%%%%%%%%%%%%%%%%%%
%% Ваше решение задачи здесь

$
\overrightarrow{MA} + \overrightarrow{MB} + \overrightarrow{MC} + \overrightarrow{MD} =
\overrightarrow{OA} + \overrightarrow{MO} +
\overrightarrow{OB} + \overrightarrow{MO} +
\overrightarrow{OC} + \overrightarrow{MO} +
\overrightarrow{OD} + \overrightarrow{MO} = \vec{0}
\\
4\overrightarrow{OM} =
 \overrightarrow{OA} +
 \overrightarrow{OB} +
 \overrightarrow{OC} +
 \overrightarrow{OD}
$
\\
Значит $M$ - центр массы четырехугольника $ABCD$ \\
То есть $M$ - середина средней линии (Такая точка единственная по свойству средней линии).

%%%%%%%%%%%%%%%%%%%%%%%%%%%%%%%%%%%%%%%%%%%%%%%%%%%%
\end{solution}

%%%%%%%%%%%%%%%%%%%%%%%%%%%%%%%%%%%%%%%%%%%%%%%%%%%%
% Задача 2
\begin{problem}{13}
Дан тетраэдр $ABCD$. Найти точку $M$ для которой
$
\overrightarrow{MA} +
\overrightarrow{MB} +
\overrightarrow{MC} +
\overrightarrow{MD} = \vec{0}
$.
\end{problem}
\begin{solution}
%%%%%%%%%%%%%%%%%%%%%%%%%%%%%%%%%%%%%%%%%%%%%%%%%%%%
%% Ваше решение задачи здесь

Аналогично задаче 10 раскроем каждый вектор как сумму радиус-векторов. \\
$
\overrightarrow{MA} + \overrightarrow{MB} + \overrightarrow{MC} + \overrightarrow{MD} =
\overrightarrow{OA} + \overrightarrow{MO} +
\overrightarrow{OB} + \overrightarrow{MO} +
\overrightarrow{OC} + \overrightarrow{MO} +
\overrightarrow{OD} + \overrightarrow{MO} = \vec{0}
$
\\
Значит $M$ - тоже центр масс. Для тетраэдра это - точка пересечения медиан.

%%%%%%%%%%%%%%%%%%%%%%%%%%%%%%%%%%%%%%%%%%%%%%%%%%%%
\end{solution}

%%%%%%%%%%%%%%%%%%%%%%%%%%%%%%%%%%%%%%%%%%%%%%%%%%%%
% Задача 3
\begin{problem}{24}
В трапеции $ABCD$ отношение основания $BC$ к основанию $AD$ равно $\lambda$. Принимая за базис
векторы $\overrightarrow{AD}$ и $\overrightarrow{AB}$, найти координаты векторов
$\overrightarrow{AB}$, $\overrightarrow{BC}$, $\overrightarrow{CD}$, $\overrightarrow{DA}$,
$\overrightarrow{AC}$ и $\overrightarrow{BD}$.
\end{problem}
\begin{solution}
%%%%%%%%%%%%%%%%%%%%%%%%%%%%%%%%%%%%%%%%%%%%%%%%%%%%
%% Ваше решение задачи здесь

$
\overrightarrow{AB} = \{0, 1\}, \\
\overrightarrow{BC} = \{\lambda, 0\}, \\
\overrightarrow{CD} = \overrightarrow{AD} - (\overrightarrow{AB} + \overrightarrow{BC}) = \{1, 0\} - (\{0, 1\} + \{\lambda, 0\}) = \{1 - \lambda, -1\}, \\
\overrightarrow{DA} = \{-1, 0\}, \\
$

%%%%%%%%%%%%%%%%%%%%%%%%%%%%%%%%%%%%%%%%%%%%%%%%%%%%
\end{solution}

%%%%%%%%%%%%%%%%%%%%%%%%%%%%%%%%%%%%%%%%%%%%%%%%%%%%
% Задача 4
\begin{problem}{17}
Доказать, что сумма векторов, идущих из центра правильного многоугольника к его вершинам, равна нулю.
\end{problem}
\begin{solution}
%%%%%%%%%%%%%%%%%%%%%%%%%%%%%%%%%%%%%%%%%%%%%%%%%%%%
%% Ваше решение задачи здесь

Пойдем от обратного.
Пусть
$
\sum_{i=3}^{n} \overrightarrow{OP}_i = \overrightarrow{a} \\
$
Повернем многоугольник на угол между двумя его вершинами ($\gamma = 180 - \alpha$, где $\alpha$ - угол многоугольника). При этом получившаяся фигура будет равна исходной в силу симметрии правильного многоугольника. То есть сумма не должна поменяться. \\
домножим сумму на матрицу поворота относительно оси $Z$ \\
$
\overrightarrow{a} = \overrightarrow{a} *
\left(
\begin{array}{rrr}
\cos{\gamma} & -\sin{\gamma} & 0 \\
\sin{\gamma} &  \cos{\gamma} & 0 \\
0 & 0 & 1
\end{array}
\right)
\\
\left(
  \begin{array}{r}
    x \\ y \\ z
  \end{array}
\right)
=
\left(
  \begin{array}{r}
    x\cos{\gamma} + y\sin{\gamma} \\
    -x\sin{\gamma} + y\cos{\gamma} \\
    z
  \end{array}
\right)
\\
\left\{
  \begin{array}{l}
    x(1 - \cos{\gamma}) = y\sin{\gamma}\\
    y(\cos{\gamma} - 1) = x\sin{\gamma}\\
  \end{array}
\right.
\left\{
  \begin{array}{l}
    \frac{x(1 - \cos{\gamma})}{\sin{\gamma}} = y\\
    -x(\cos{\gamma} - 1)^2 = x\sin{\gamma}^2\\
  \end{array}
\right.
\left\{
  \begin{array}{l}
    \frac{x(1 - \cos{\gamma})}{\sin{\gamma}} = y\\
    x((\cos{\gamma} - 1)^2 + \sin{\gamma}^2) = 0 \\
  \end{array}
\right.
\\
\left\{
  \begin{array}{l}
    \frac{x(1 - \cos{\gamma})}{\sin{\gamma}} = y\\
    x(2 - 2\cos{\gamma}) = 0 \\
  \end{array}
\right.
\\
\left\{
  \begin{array}{l}
    x = 0 \\
    y = 0
  \end{array}
\right.
\lor
\gamma = 0
$

По условию $\gamma > 0$ значит $x = 0 \land y = 0$ чтд.

%%%%%%%%%%%%%%%%%%%%%%%%%%%%%%%%%%%%%%%%%%%%%%%%%%%%
\end{solution}

%%%%%%%%%%%%%%%%%%%%%%%%%%%%%%%%%%%%%%%%%%%%%%%%%%%%
% Задача 5
\begin{problem}{29}
Показать, что каковы бы ни были три вектора $\vec a$, $\vec b$ и $\vec c$ и три числа $\alpha$, $\beta$, $\gamma$
векторы $\alpha\vec{a}-\beta\vec{b}$, $\gamma\vec{b}-\alpha\vec{c}$,
$\beta\vec{c}-\gamma\vec{a}$ компланарны.
\end{problem}
\begin{solution}
%%%%%%%%%%%%%%%%%%%%%%%%%%%%%%%%%%%%%%%%%%%%%%%%%%%%
%% Ваше решение задачи здесь

$
\\
\dot{\overrightarrow{a}} = \alpha\vec{a}-\beta\vec{b}
\\
\dot{\overrightarrow{b}} = \gamma\vec{b}-\alpha\vec{c}
\\
\dot{\overrightarrow{c}} = \beta\vec{c}-\gamma\vec{a}
$

3 вектора комланарны тогда и только тогда, когда они линейно зависимы. \\
$\dot{\overrightarrow{c}} = -\dot{\overrightarrow{a}} - \dot{\overrightarrow{b}}$ \\
Значит они компланарны.

%%%%%%%%%%%%%%%%%%%%%%%%%%%%%%%%%%%%%%%%%%%%%%%%%%%%
\end{solution}

%%%%%%%%%%%%%%%%%%%%%%%%%%%%%%%%%%%%%%%%%%%%%%%%%%%%
% Задача 6
\begin{problem}{31}
Даны вектора $\vec{a}=\{1,2,3\}$, $\vec{b}=\{2,-2,1\}$, $\vec{c}=\{4,0,3\}$, $\vec{d}=\{16,10,18\}$.
Найти вектор, являющийся проекцией вектора $\vec d$ на плоскость, определяемую векторами
$\vec a$ и $\vec b$ при направлении проектирования, параллельном вектору $\vec c$.
\end{problem}
\begin{solution}
%%%%%%%%%%%%%%%%%%%%%%%%%%%%%%%%%%%%%%%%%%%%%%%%%%%%
%% Ваше решение задачи здесь

Плоскость $(\overrightarrow{a}, \overrightarrow{b})$ определяется нормалью. \\
$
\overrightarrow{n} =
\overrightarrow{a} \times \overrightarrow{b} =
\left|
  \begin{array}{rrr}
    \overrightarrow{i} & \overrightarrow{j} & \overrightarrow{k} \\
    1 &  2 & 3 \\
    2 & -2 & 1
  \end{array}
\right|
= \{8, 5, -6\}
$
\\ Уравнение плоскости: $\overrightarrow{p} \cdot \overrightarrow{n} = 0$ \\
Расстояние от начала координат до плоскости равно 0, и вектор d начинается в начале координат.
Значит нужно спроектировать на плоскость только саму точку (16, 10, 18).
Проекция будет совпадать с точкой пересечения прямой, параллельной $\overrightarrow{c}$ и проходящей через $\overrightarrow{d}$
\\ Уравнение прямой: $\overrightarrow{p} = \overrightarrow{d} + \alpha * \overrightarrow{c}$

Значит уравнение точки пересечения:
$(\overrightarrow{d} + \alpha * \overrightarrow{c}) \cdot \overrightarrow{n} = 0$ \\
$\alpha = -\frac{\overrightarrow{d} \cdot \overrightarrow{n}}{\overrightarrow{c} \cdot \overrightarrow{n}}$ \\
$\alpha = -\frac{16 * 8 + 5 * 10 - 6 * 18}{4 * 8 + 0 * 5 - 3 * 6} = -5$ \\

Значит спроектированная точка:
$
\overrightarrow{p} =
\overrightarrow{d} + -5\overrightarrow{c} =
\{16 - 5 * 4, 10 - 5 * 0, 18 - 5 * 3\} = \{-4, 10, 3\}
$ \\
Ответ: $\{-4, 10, 3\}$

%%%%%%%%%%%%%%%%%%%%%%%%%%%%%%%%%%%%%%%%%%%%%%%%%%%%
\end{solution}

%------------------------------------------------
\end{document}
