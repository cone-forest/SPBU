\documentclass[a4paper, 12pt]{article}
%----------------------------------------------------------------------------------------
%	PACKAGES AND OTHER DOCUMENT CONFIGURATIONS
%----------------------------------------------------------------------------------------
\usepackage[a4paper, total={7in, 10in}]{geometry}
\setlength{\parskip}{0pt}
\setlength{\parindent}{0in}

\usepackage[T2A]{fontenc}% Внутренняя T2A кодировка TeX
\usepackage[utf8]{inputenc}% кодировка файла
\usepackage[russian]{babel}% поддержка переносов в русском языке
\usepackage{amsthm, amsmath, amssymb} % Mathematical typesetting
\usepackage{float} % Improved interface for floating objects
\usepackage{graphicx, multicol} % Enhanced support for graphics
\usepackage{xcolor} % Driver-independent color extensions
\usepackage{mdframed}

\usepackage[yyyymmdd]{datetime} % Uses YEAR-MONTH-DAY format for dates
\renewcommand{\dateseparator}{.} % Sets dateseparator to '.'

\usepackage{fancyhdr} % Headers and footers
\pagestyle{fancy} % All pages have headers and footers
\fancyhead{}\renewcommand{\headrulewidth}{0pt} % Blank out the default header
\fancyfoot[L]{} % Custom footer text
\fancyfoot[C]{} % Custom footer text
\fancyfoot[R]{\thepage} % Custom footer text

\newenvironment{problem}[2][Задача]
    { \begin{mdframed}[backgroundcolor=gray!10] \textbf{#1 #2.} \\}
    {  \end{mdframed}}

\newenvironment{solution}
    {\textit{Решение: }}
    {\noindent\rule{7in}{1.5pt}}

\begin{document}

%-------------------------------
%	TITLE SECTION
%-------------------------------

\fancyhead[C]{}
\hrule \medskip % Upper rule
\begin{minipage}{0.295\textwidth}
\raggedright\footnotesize
Михаил Цуканов \hfill\\
st117303 \hfill\\
st117303@student.spbu.ru
\end{minipage}
\begin{minipage}{0.4\textwidth}
\centering\large
Homework Assignment 19\\
\normalsize
Алгебра и геометрия, 1 семестр\\
\end{minipage}
\begin{minipage}{0.295\textwidth}
\raggedleft
\today\hfill\\
\end{minipage}
\medskip\hrule
\bigskip

%------------------------------------------------
%	CONTENTS
%------------------------------------------------



%%%%%%%%%%%%%%%%%%%%%%%%%%%%%%%%%%%%%%%%%%%%%%%%%%%%
% Задача 1
\begin{problem}{535(c)}
Преобразовать квадратичную форму к каноническому виду ортогональным преобразованием: $3 x_1^2 + 4 x_2^2 + 5 x_3^2 + 4 x_1 x_2 - 4 x_2 x_3$
\end{problem}
\begin{solution}
%%%%%%%%%%%%%%%%%%%%%%%%%%%%%%%%%%%%%%%%%%%%%%%%%%%%
%% Ваше решение задачи здесь

$
\left(
\begin{array}{ccc}
3 & 2 & 0 \\
2 & 4 & -2 \\ 
0 & -2 & 5 \\
\end{array}
\right)
$
=>
\\
\\
$
\left|
\begin{array}{ccc}
3-\lambda & 2 & 0 \\
2 & 4-\lambda & -2 \\ 
0 & -2 & 5-\lambda \\
\end{array}
\right|
=
-\lambda^3+12\lambda^2-39\lambda+28
=
-(\lambda-4)(\lambda-7)(\lambda-1)=>
\\
\\
\lambda_1=4,\lambda_2=7,\lambda_3=1
\\
\\
\\
\lambda=4:
\left(
\begin{array}{ccc}
-1 & 2 & 0 \\
2 & 0 & -2 \\ 
0 & -2 & 1 \\
\end{array}
\right)_{S_2+2S_1,S_1*(-1)}
\sim
\left(
\begin{array}{ccc}
1 & -2 & 0 \\
0 & 4 & -2 \\ 
0 & -2 & 1 \\
\end{array}
\right)_{S_2-2S_3}
\sim
\left(
\begin{array}{ccc}
1 & -2 & 0 \\
0 & -2 & 1 \\
\end{array}
\right)_{S_2-S_1}
\sim
\left(
\begin{array}{ccc}
1 & -2 & 0 \\
-1 & 0 & 1 \\
\end{array}
\right)_{} =>
\\
h_1=
\left(
\begin{array}{c}
2 \\
1 \\
2 
\end{array}
\right)
,
|h_1|=3
\\
\\
\lambda=7:
\left(
\begin{array}{ccc}
-4 & 2 & 0 \\
2 & -3 & -2 \\ 
0 & -2 & -2 \\
\end{array}
\right)_{S_2+S_3,S_3*(-1)}
\sim
\left(
\begin{array}{ccc}
-4 & 2 & 0 \\
2 & -1 & 0 \\ 
0 & 2 & 2 \\
\end{array}
\right)_{S_1-2S_2}
\sim
\left(
\begin{array}{ccc}
2 & -1 & 0 \\
0 & 1 & 1 \\
\end{array}
\right)_{}
=>
\\
h_2=
\left(
\begin{array}{c}
1 \\
2 \\
-2 
\end{array}
\right)
,
|h_2|=3
\\
\\
\lambda=1:
\left(
\begin{array}{ccc}
2 & 2 & 0 \\
2 & 3 & -2 \\ 
0 & -2 & 4 \\
\end{array}
\right)_{S_2-S_1}
\sim
\left(
\begin{array}{ccc}
1 & 1 & 0 \\
0 & 1 & -2 \\ 
0 & -1 & 2 \\
\end{array}
\right)_{S_2+S_3}
\sim
\left(
\begin{array}{ccc}
1 & 1 & 0 \\
0 & 1 & -2 \\ 
\end{array}
\right)_{}
=>
\\
h_3=
\left(
\begin{array}{c}
-2 \\
2 \\
1
\end{array}
\right)
,
|h_3|=3
$
\\
\\
$H=
\left(
\begin{array}{ccc}
\frac{2}{3} & \frac{1}{3} & -\frac{2}{3} \\
\frac{1}{3} & \frac{2}{3} & \frac{2}{3} \\ 
\frac{2}{3} & -\frac{2}{3} & \frac{1}{3} \\
\end{array}
\right)
\\
\\
4y_1^2+7y_2^2+y_3^2
$

%%%%%%%%%%%%%%%%%%%%%%%%%%%%%%%%%%%%%%%%%%%%%%%%%%%%
\end{solution}

%%%%%%%%%%%%%%%%%%%%%%%%%%%%%%%%%%%%%%%%%%%%%%%%%%%%
% Задача 2
\begin{problem}{535(d)}
Преобразовать квадратичную форму к каноническому виду ортогональным преобразованием: $2 x_1^2 + 5 x_2^2 + 5 x_3^2 + 4 x_1 x_2 - 4 x_1 x_3 - 8 x_2 x_3$
\end{problem}
\begin{solution}
%%%%%%%%%%%%%%%%%%%%%%%%%%%%%%%%%%%%%%%%%%%%%%%%%%%%
%% Ваше решение задачи здесь

$
\left(
\begin{array}{ccc}
2 & 2 & -2 \\
2 & 5 & -4 \\ 
-2 & -4 & 5 \\
\end{array}
\right)
$
=>
\\
\\
$
\left|
\begin{array}{ccc}
2-\lambda & 2 & -2 \\
2 & 5-\lambda & -4 \\ 
-2 & -4 & 5-\lambda \\
\end{array}
\right|
=
(1-\lambda)^2(\lambda-10)=0 =>
\\
\\
\lambda_1=1(kr 2),\lambda_2=10
\\
\\
\\
\lambda_1=1(kr 2):
\left(
\begin{array}{ccc}
1 & 2 & -2 \\
2 & 4 & -4 \\ 
-2 & -4 & 4 \\
\end{array}
\right)
\sim
\left(
\begin{array}{ccc}
1 & 2 & -2 \\
\end{array}
\right)
=>
h_1=
\left(
\begin{array}{c}
-2 \\
1 \\
0
\end{array}
\right),
h_2=
\left(
\begin{array}{c}
2 \\
0 \\
1
\end{array}
\right)
\\
$
Данные вектора не ортоганальны, преобразуем их в ортогональные:
\\
Пусть $b_1=h_1,b_2=h_2+ah_1,(b_1,b_2)=0 => (h_2,h_1)+a(h_1,h_1)=0 => a=0,8 =>
b_2=
\left(
\begin{array}{c}
0,4 \\
0,8 \\
1
\end{array}
\right)
\\
\\
\\
\lambda_2=10:
\left(
\begin{array}{ccc}
-8 & 2 & -2 \\
2 & -5 & -4 \\ 
-2 & -4 & -5 \\
\end{array}
\right)_{S_3+S_2}
\sim
\left(
\begin{array}{ccc}
-4 & 1 & -1 \\
2 & -5 & -4 \\ 
0 & -9 & -9 \\
\end{array}
\right)_{S_1+2S_2}
\sim
\left(
\begin{array}{ccc}
0 & -9 & -9 \\
2 & -5 & -4 \\ 
0 & 1 & 1 \\
\end{array}
\right)_{S_2+4S_3}
\sim
\left(
\begin{array}{ccc}
2 & -1 & 0 \\
0 & 1 & 1
\end{array}
\right)
=>
h_3=
\left(
\begin{array}{c}
1 \\
2 \\
-2
\end{array}
\right)
$
\\
\\
$|b_1|=\sqrt{5},|b_2|=\frac{3}{\sqrt{5}},|h_3|=3 =>
\\
\\
H=
\left(
\begin{array}{ccc}
-\frac{2}{\sqrt{5}} & \frac{2}{3\sqrt{5}} & \frac{1}{3} \\
\frac{1}{\sqrt{5}} & \frac{4}{3\sqrt{5}} & \frac{2}{3} \\ 
0 & \frac{\sqrt{5}}{3} & -\frac{2}{3} \\
\end{array}
\right)
\\
y_1^2+y_2^2+10y_3^2
$

%%%%%%%%%%%%%%%%%%%%%%%%%%%%%%%%%%%%%%%%%%%%%%%%%%%%
\end{solution}

%%%%%%%%%%%%%%%%%%%%%%%%%%%%%%%%%%%%%%%%%%%%%%%%%%%%
% Задача 3
\begin{problem}{535(e)}
Преобразовать квадратичную форму к каноническому виду ортогональным преобразованием: $x_1^2 - 2 x_2^2 - 2 x_3^2 - 4 x_1 x_2 + 4 x_1 x_3 + 8 x_2 x_3$
\end{problem}
\begin{solution}
%%%%%%%%%%%%%%%%%%%%%%%%%%%%%%%%%%%%%%%%%%%%%%%%%%%%
%% Ваше решение задачи здесь

$
\left(
\begin{array}{ccc}
1 & -2 & 2 \\
-2 & -2 & 4 \\ 
2 & 4 & -2 \\
\end{array}
\right)
$
=>
\\
\\
$
\left|
\begin{array}{ccc}
1-\lambda & -2 & 2 \\
-2 & -2-\lambda & 4 \\ 
2 & 4 & -2-\lambda \\
\end{array}
\right|
=
(2-\lambda)^2(\lambda+7)=0 =>
\\
\\
\lambda_1=2(kr 2),\lambda_2=-7
\\
\\
\\
\lambda_1=2(kr 2):
\left(
\begin{array}{ccc}
-1 & -2 & 2 \\
-2 & -4 & 4 \\ 
2 & 4 & -4 \\
\end{array}
\right)
\sim
\left(
\begin{array}{ccc}
1 & 2 & -2 \\
\end{array}
\right)
=>
h_1=
\left(
\begin{array}{c}
-2 \\
1 \\
0
\end{array}
\right),
h_2=
\left(
\begin{array}{c}
2 \\
0 \\
1
\end{array}
\right)
\\
$
Данные вектора не ортоганальны, преобразуем их в ортогональные:
\\
Пусть $b_1=h_1,b_2=h_2+ah_1,(b_1,b_2)=0 => (h_2,h_1)+a(h_1,h_1)=0 => a=0,8 =>
b_2=
\left(
\begin{array}{c}
0,4 \\
0,8 \\
1
\end{array}
\right)
\\
\\
\\
\lambda_2=-7:
\left(
\begin{array}{ccc}
8 & -2 & 2 \\
-2 & 5 & 4 \\ 
2 & 4 & 5 \\
\end{array}
\right)_{S_2+S_3}
\sim
\left(
\begin{array}{ccc}
4 & -1 & 1 \\
0 & 9 & 9 \\ 
2 & 4 & 5 \\
\end{array}
\right)_{S_1-2S_3,S_3-5/9S_2}
\sim
\left(
\begin{array}{ccc}
0 & -9 & -9 \\
0 & 1 & 1 \\ 
2 &-1 & 0 \\
\end{array}
\right)_{}
\sim
\left(
\begin{array}{ccc}
2 & -1 & 0 \\
0 & 1 & 1
\end{array}
\right)
=>
h_3=
\left(
\begin{array}{c}
1 \\
2 \\
-2
\end{array}
\right)
$
\\
\\
$|b_1|=\sqrt{5},|b_2|=\frac{3}{\sqrt{5}},|h_3|=3 =>
\\
\\
H=
\left(
\begin{array}{ccc}
-\frac{2}{\sqrt{5}} & \frac{2}{3\sqrt{5}} & \frac{1}{3} \\
\frac{1}{\sqrt{5}} & \frac{4}{3\sqrt{5}} & \frac{2}{3} \\ 
0 & \frac{\sqrt{5}}{3} & -\frac{2}{3} \\
\end{array}
\right)
\\
2y_1^2+2y_2^2-7y_3^2
$

%%%%%%%%%%%%%%%%%%%%%%%%%%%%%%%%%%%%%%%%%%%%%%%%%%%%
\end{solution}


%------------------------------------------------
\end{document}
