\documentclass[a4paper, 12pt]{article}
%----------------------------------------------------------------------------------------
%	PACKAGES AND OTHER DOCUMENT CONFIGURATIONS
%----------------------------------------------------------------------------------------
\usepackage[a4paper, total={7in, 10in}]{geometry}
\setlength{\parskip}{0pt}
\setlength{\parindent}{0in}

\usepackage[T2A]{fontenc}% Внутренняя T2A кодировка TeX
\usepackage[utf8]{inputenc}% кодировка файла
\usepackage[russian]{babel}% поддержка переносов в русском языке
\usepackage{amsthm, amsmath, amssymb} % Mathematical typesetting
\usepackage{float} % Improved interface for floating objects
\usepackage{graphicx, multicol} % Enhanced support for graphics
\usepackage{xcolor} % Driver-independent color extensions
\usepackage{mdframed}

\usepackage[yyyymmdd]{datetime} % Uses YEAR-MONTH-DAY format for dates
\renewcommand{\dateseparator}{.} % Sets dateseparator to '.'

\usepackage{fancyhdr} % Headers and footers
\pagestyle{fancy} % All pages have headers and footers
\fancyhead{}\renewcommand{\headrulewidth}{0pt} % Blank out the default header
\fancyfoot[L]{} % Custom footer text
\fancyfoot[C]{} % Custom footer text
\fancyfoot[R]{\thepage} % Custom footer text

\newenvironment{problem}[2][Задача]
    { \begin{mdframed}[backgroundcolor=gray!10] \textbf{#1 #2.} \\}
    {  \end{mdframed}}

\newenvironment{solution}
    {\textit{Решение: }}
    {\noindent\rule{7in}{1.5pt}}

\begin{document}

%-------------------------------
%	TITLE SECTION
%-------------------------------

\fancyhead[C]{}
\hrule \medskip % Upper rule
\begin{minipage}{0.295\textwidth}
\raggedright\footnotesize
Цуканов Михаил \hfill\\
st117303 \hfill\\
st117303@student.spbu.ru
\end{minipage}
\begin{minipage}{0.4\textwidth}
\centering\large
Homework Assignment 16\\
\normalsize
Алгебра и геометрия, 1 семестр\\
\end{minipage}
\begin{minipage}{0.295\textwidth}
\raggedleft
\today\hfill\\
\end{minipage}
\medskip\hrule
\bigskip

%------------------------------------------------
%	CONTENTS
%------------------------------------------------



%%%%%%%%%%%%%%%%%%%%%%%%%%%%%%%%%%%%%%%%%%%%%%%%%%%%
% Задача 1
\begin{problem}{514}
Составить уравнение плоскости, проходящей через точку $(-2,3,0)$ и через прямую $x=1$, $y=2+t$, $z=2-t$. Система координат аффинная.
\end{problem}
\begin{solution}
%%%%%%%%%%%%%%%%%%%%%%%%%%%%%%%%%%%%%%%%%%%%%%%%%%%%
%% Ваше решение задачи здесь

Определим оставшиеся 2 точки, достаточные для задания плоскости. \\
$x=1 \land y=2+t \land z=2-t => x=1 \land z = 4 - y$. \\
Выберем удобные точки на этой прямой. Например $(1, 0, 4)$ и $(1, 4, 0)$. \\
$\vec{a} = \{3, -3, 4\}$
$\vec{b} = \{3, 1, 0\}$
\\
$
\vec{a} \times \vec{b} =
\left|
  \begin{array}{rrr}
    i & j & k \\
    3 & -3 & 4 \\
    3 & 1 & 0
  \end{array}
\right|
= \{-4, 12, 12\}
$ \\
$\vec{n} = \{-1, 3, 3\}$ \\
Тогда уравнение плоскости: $-x + 3y + 3z - 11 = 0$

%%%%%%%%%%%%%%%%%%%%%%%%%%%%%%%%%%%%%%%%%%%%%%%%%%%%
\end{solution}

%%%%%%%%%%%%%%%%%%%%%%%%%%%%%%%%%%%%%%%%%%%%%%%%%%%%
% Задача 2
\begin{problem}{518}
Написать уравнения биссектрисы тупого угла между прямой
$
\left\{
\begin{array}{l}
x-2y-5 = 0 \\
y-4z+14= 0
\end{array}
\right.
$
и её ортогональной проекцией на плоскость ${x+y+1=0}$.
Система координат прямоугольная.
\end{problem}
\begin{solution}
%%%%%%%%%%%%%%%%%%%%%%%%%%%%%%%%%%%%%%%%%%%%%%%%%%%%
%% Ваше решение задачи здесь

Нормаль к плоскости: $\vec{n} = \{1, 1, 0\}$ \\
Направляющий вектор прямой: $\vec{l} = \{8, 4, 1\}$ \\
Точка пересечения прямой и плоскости: $M(1, -2, 3)$ \\
Найдем точку, принадлежащую ортогональной проекции прямой на данную плоскость. \\
Для этого возьмем точку прямой $M + \vec{l}$ и составим уравнение прямой, проходящей через эту точку и параллельую нормали к плоскости: \\
$\{1, 1, 0\}t + \{9, 2, 4\} = \vec{p} \land x + y + 1 = 0$ <=> \\
$
\left\{
  \begin{array}{rrrr}
    x = t + 9 \\
    y = t + 2 \\
    z = 4 \\
    x + y + 1 = 0
  \end{array}
\right.
<=> t = -6 <=>
\left\{
  \begin{array}{rrrr}
    x = 3 \\
    y = -4 \\
    z = 4 \\
  \end{array}
\right.
$ \\
Тогда направляющий вектор проекции будет равен: $\vec{k} = M - (3, -4, 4) = (-2, 2, -1)$ \\
Проверим, что угол тупой: $\vec{l} \cdot \vec{k} = -16 + 8 -1 < 0 \rightarrow \widehat{(\vec{l}, \vec{k})} > \frac{\pi}{2}$ \\
Тогда направляющий вектор искомой прямой будет равен среднему арифметическому нормализованных направляющих векторов данной прямой и ее проекции:
$\frac{\vec{k}}{6} + \frac{\vec{l}}{18} = \frac{3\vec{k} + \vec{l}}{18} = \{\frac{1}{9}, \frac{5}{9}, -\frac{1}{9}\}$ \\
Можем безболезненно умножить его на 9: $\{1, 5, -1\}$. \\
Тогда ответ: $\vec{p} = \{1, 5, -1\}t + \{1, -2, 3\}$

% Для того, чтобы получить уравнение искомой прямой, нужно повернуть направляющий вектор на $\frac{\alpha}{2}$, где $\alpha = \widehat{(\vec{n}, \vec{l})} + \frac{\pi}{2}$, а за точку на прямой взять точку пересечения - $\{1, -2, 3\}$.
%
% $\cos{\widehat{(\vec{n}, \vec{l})}} = {\frac{\vec{l} \cdot \vec{n}}{ln}} = {\frac{12}{9\sqrt2}}=\frac{2\sqrt2}{3}$
% $\sin{\widehat{(\vec{n}, \vec{l})}} = \frac{1}{3}$ \\
%
% $\cos\alpha = -\frac{1}{3}$
% $\sin\alpha = \frac{2\sqrt2}{3}$ \\
% \\
% $
% \vec{r} =
% \vec{l} \times \vec{n} =
% \left|
%   \begin{array}{rrr}
%     i & j & k \\
%     1 & 1 & 0 \\
%     8 & 4 & 1
%   \end{array}
% \right|
% = \{1, 1, -4\}
% $;
% $\vec{R} = \frac{\vec{r}}{r} = \{\frac{\sqrt2}{6}, \frac{\sqrt2}{6}, \frac{2\sqrt2}{3}\}$
%
% Повернем направляющий вектор $\vec{l}$ на $\alpha$ относительно вектора $\vec{R}$. \\
%
% Для этого можно использовать матрицу

%%%%%%%%%%%%%%%%%%%%%%%%%%%%%%%%%%%%%%%%%%%%%%%%%%%%
\end{solution}

%%%%%%%%%%%%%%%%%%%%%%%%%%%%%%%%%%%%%%%%%%%%%%%%%%%%
% Задача 3
\begin{problem}{508}
Написать уравнение плоскости, проходящей через точку $(1,2,3)$
параллельной прямой $x=y=z$ и отсекающей на осях $Ox$ и $Oy$ равные отрезки.
Система координат аффинная.
\end{problem}
\begin{solution}
%%%%%%%%%%%%%%%%%%%%%%%%%%%%%%%%%%%%%%%%%%%%%%%%%%%%
%% Ваше решение задачи здесь

Искомая плоскость должна отсекать на осях $Ox$ и $Oy$ равные отрезки. \\
Значит она параллельна прямой x = -y (вектору $\{1, -1, 0\}$) \\
По условию она параллельная прямой x = y = z (вектоу $\{1, 1, 1\}$) \\
И проходит через точку $(1, 2, 3)$ \\
$
\vec{n} = \{1, 1, 0\} \times \{1, 1, 1\} =
\left|
\begin{array}{rrr}
  i &  j & k \\
  1 & -1 & 0 \\
  1 &  1 & 1
\end{array}
\right|
=
\{-1, 1, 2\} \\
\text{Тогда уравнение плоскости:} \\
-x+y+2z = -1 + 2 + 6 \\
-x + y + 2z - 7 = 0
$

%%%%%%%%%%%%%%%%%%%%%%%%%%%%%%%%%%%%%%%%%%%%%%%%%%%%
\end{solution}

%%%%%%%%%%%%%%%%%%%%%%%%%%%%%%%%%%%%%%%%%%%%%%%%%%%%
% Задача 4
\begin{problem}{569}
Написать уравнение плоскости, проходящей через точки $(1,2,3)$ и $(4,5,7)$ и перпендикулярной к плоскости $x-y+2z-4=0$.
\end{problem}
\begin{solution}
%%%%%%%%%%%%%%%%%%%%%%%%%%%%%%%%%%%%%%%%%%%%%%%%%%%%
%% Ваше решение задачи здесь

Для задания плоскости необходимо 3 точки. Обозначим недостающую за $X(x, y, z)$ \\
Тогда нормаль к искомой плоскости будет равна \\
$
\{1 - x, 2 - y, 3 - z\} \times \{4 - x, 5 - y, 7 - z\} =
\left|
  \begin{array}{rrr}
    i & j & k \\
    1 - x & 2 - y & 3 - z \\
    4 - x & 5 - y & 7 - z
  \end{array}
\right|
= \\
\{(2 - x)(7 - z) - (3 - z)(5 - y), (3 - z)(4 - x) - (1 - x)(7 - z), (1 - x)(5 - y) - (2 - y)(4 - x)\} = \\
\{ \\
(14 - 7y - 2z + yz) - (15 - 5z -3y + yz), \\
(12 - 4z - 3x + xz) - (7 - 7x - z + xz), \\
(5 - 5x - y + xy) - (8 - 4y - 2x + yx) \\
\}
$ \\
Нормаль к даннй плоскости равна $\{1, -1, 2\}$ \\
Искомая плоскость перпендикулярна данной. Значит скалярное произведение их нормалей равно 0: \\
$
(-1 + 3z - 4y) - (5 - 3z+ 4x) + 2(-3 + 3y - 3x) = 0 \\
-10x + 2y + 6z = 12 \\
$
Значит в искомую плоскость попадает любая точка, удовлетворяющая уравнению $-5x +y + 3z = 6$. \\
Ответ: $-5x +y + 3z = 6$

%%%%%%%%%%%%%%%%%%%%%%%%%%%%%%%%%%%%%%%%%%%%%%%%%%%%
\end{solution}


%------------------------------------------------
\end{document}
