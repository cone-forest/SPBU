\documentclass[a4paper, 12pt]{article}
%----------------------------------------------------------------------------------------
%	PACKAGES AND OTHER DOCUMENT CONFIGURATIONS
%----------------------------------------------------------------------------------------
\usepackage[a4paper, total={7in, 10in}]{geometry}
\setlength{\parskip}{0pt}
\setlength{\parindent}{0in}

\usepackage[T2A]{fontenc}% Внутренняя T2A кодировка TeX
\usepackage[utf8]{inputenc}% кодировка файла
\usepackage[russian]{babel}% поддержка переносов в русском языке
\usepackage{amsthm, amsmath, amssymb} % Mathematical typesetting
\usepackage{float} % Improved interface for floating objects
\usepackage{graphicx, multicol} % Enhanced support for graphics
\usepackage{xcolor} % Driver-independent color extensions
\usepackage{mdframed}

\usepackage[yyyymmdd]{datetime} % Uses YEAR-MONTH-DAY format for dates
\renewcommand{\dateseparator}{.} % Sets dateseparator to '.'

\usepackage{fancyhdr} % Headers and footers
\pagestyle{fancy} % All pages have headers and footers
\fancyhead{}\renewcommand{\headrulewidth}{0pt} % Blank out the default header
\fancyfoot[L]{} % Custom footer text
\fancyfoot[C]{} % Custom footer text
\fancyfoot[R]{\thepage} % Custom footer text

\newenvironment{problem}[2][Задача]
    { \begin{mdframed}[backgroundcolor=gray!10] \textbf{#1 #2.} \\}
    {  \end{mdframed}}

\newenvironment{solution}
    {\textit{Решение: }}
    {\noindent\rule{7in}{1.5pt}}

\begin{document}

%-------------------------------
%	TITLE SECTION
%-------------------------------

\fancyhead[C]{}
\hrule \medskip % Upper rule
\begin{minipage}{0.295\textwidth} 
\raggedright\footnotesize
Цуканов Михаил \hfill\\   
st117303 \hfill\\
st117303@student.spbu.ru
\end{minipage}
\begin{minipage}{0.4\textwidth} 
\centering\large 
Homework Assignment 8\\ 
\normalsize 
Алгебра и геометрия, 1 семестр\\ 
\end{minipage}
\begin{minipage}{0.295\textwidth} 
\raggedleft
\today\hfill\\
\end{minipage}
\medskip\hrule 
\bigskip

%------------------------------------------------
%	CONTENTS
%------------------------------------------------



%%%%%%%%%%%%%%%%%%%%%%%%%%%%%%%%%%%%%%%%%%%%%%%%%%%%
% Задача 1
\begin{problem}{293}
Вычислить определитель порядка $2n$
$\left| \begin{array}{rrrrrrr}a & 0 & 0 & \ldots & 0 & 0 & b\\ 0 & a & 0 & \ldots & 0 & b & 0 \\ 0 & 0 & a & \ldots & b & 0 & 0 \\ \vdots & \vdots & \vdots & \ddots & \vdots & \vdots & \vdots \\ 0 & 0 & b & \ldots & a & 0 & 0 \\ 0 & b & 0 & \ldots & 0 & a & 0 \\ b & 0 & 0 & \ldots & 0 & 0 & a \end{array} \right|$
\end{problem}
\begin{solution}
%%%%%%%%%%%%%%%%%%%%%%%%%%%%%%%%%%%%%%%%%%%%%%%%%%%%
%% Ваше решение задачи здесь

Разложим по первому столбцу \\
$
\left|
\begin{array}{rrrrrrr}
a & 0 & 0 & \ldots & 0 & 0 & b \\
0 & a & 0 & \ldots & 0 & b & 0 \\
0 & 0 & a & \ldots & b & 0 & 0 \\
\vdots & \vdots & \vdots & \ddots & \vdots & \vdots & \vdots \\
0 & 0 & b & \ldots & a & 0 & 0 \\
0 & b & 0 & \ldots & 0 & a & 0 \\
b & 0 & 0 & \ldots & 0 & 0 & a
\end{array}
\right|
=
a
\left|
\begin{array}{rrrrrrr}
a & 0 & 0 & \ldots & 0 & 0 & b \\
0 & a & 0 & \ldots & 0 & b & 0 \\
0 & 0 & a & \ldots & b & 0 & 0 \\
\vdots & \vdots & \vdots & \ddots & \vdots & \vdots & \vdots \\
0 & 0 & b & \ldots & a & 0 & 0 \\
0 & b & 0 & \ldots & 0 & a & 0 \\
b & 0 & 0 & \ldots & 0 & 0 & a
\end{array}
\right|
- b
\left|
\begin{array}{rrrrrrr}
a & 0 & 0 & \ldots & 0 & 0 & b \\
0 & a & 0 & \ldots & 0 & b & 0 \\
0 & 0 & a & \ldots & b & 0 & 0 \\
\vdots & \vdots & \vdots & \ddots & \vdots & \vdots & \vdots \\
0 & 0 & b & \ldots & a & 0 & 0 \\
0 & b & 0 & \ldots & 0 & a & 0 \\
b & 0 & 0 & \ldots & 0 & 0 & a
\end{array}
\right|
= \\
$
\\
Теперь размерность матрицы - $2n - 1$ \\
Вынесем определитель за скобки и разложим по последнему столбику \\ (чтобы сохранить симметричность) \\
$
=
(a - b)
\left|
\begin{array}{rrrrrrr}
a & 0 & 0 & \ldots & 0 & 0 & b \\
0 & a & 0 & \ldots & 0 & b & 0 \\
0 & 0 & a & \ldots & b & 0 & 0 \\
\vdots & \vdots & \vdots & \ddots & \vdots & \vdots & \vdots \\
0 & 0 & b & \ldots & a & 0 & 0 \\
0 & b & 0 & \ldots & 0 & a & 0 \\
b & 0 & 0 & \ldots & 0 & 0 & a
\end{array}
\right|
=
(a - b)
\left(
b
\left|
\begin{array}{rrrrrrr}
a & 0 & 0 & \ldots & 0 & 0 & b \\
0 & a & 0 & \ldots & 0 & b & 0 \\
0 & 0 & a & \ldots & b & 0 & 0 \\
\vdots & \vdots & \vdots & \ddots & \vdots & \vdots & \vdots \\
0 & 0 & b & \ldots & a & 0 & 0 \\
0 & b & 0 & \ldots & 0 & a & 0 \\
b & 0 & 0 & \ldots & 0 & 0 & a
\end{array}
\right|
+ a
\left|
\begin{array}{rrrrrrr}
a & 0 & 0 & \ldots & 0 & 0 & b \\
0 & a & 0 & \ldots & 0 & b & 0 \\
0 & 0 & a & \ldots & b & 0 & 0 \\
\vdots & \vdots & \vdots & \ddots & \vdots & \vdots & \vdots \\
0 & 0 & b & \ldots & a & 0 & 0 \\
0 & b & 0 & \ldots & 0 & a & 0 \\
b & 0 & 0 & \ldots & 0 & 0 & a
\end{array}
\right|
\right)
$
\\ \\
Вынесем определитель за скобки \\
$
(a - b)(a + b)
\left|
\begin{array}{rrrrrrr}
a & 0 & 0 & \ldots & 0 & 0 & b \\
0 & a & 0 & \ldots & 0 & b & 0 \\
0 & 0 & a & \ldots & b & 0 & 0 \\
\vdots & \vdots & \vdots & \ddots & \vdots & \vdots & \vdots \\
0 & 0 & b & \ldots & a & 0 & 0 \\
0 & b & 0 & \ldots & 0 & a & 0 \\
b & 0 & 0 & \ldots & 0 & 0 & a
\end{array}
\right|
$
\\ \\
Размерность оставшейся матрицы $2(n - 1)$ \\
Повторив операцию n раз получим ответ: \\
$
(a^2 - b^2)^n
$

%%%%%%%%%%%%%%%%%%%%%%%%%%%%%%%%%%%%%%%%%%%%%%%%%%%%
\end{solution} 

%%%%%%%%%%%%%%%%%%%%%%%%%%%%%%%%%%%%%%%%%%%%%%%%%%%%
% Задача 2
\begin{problem}{296}
Вычислить определитель 
$\left| \begin{array}{cccccc}1 & 2 & 3 & \ldots & n-1 & n \\ 1 & 1 & 1 & \ldots & 1 & 1-n\\ 1 & 1 & 1 & \ldots & 1-n & 1 \\ \vdots & \vdots & \vdots & \ddots & \vdots \\ 1 & 1-n & 1 & \ldots & 1 & 1 \end{array} \right|$
\end{problem}
\begin{solution}
%%%%%%%%%%%%%%%%%%%%%%%%%%%%%%%%%%%%%%%%%%%%%%%%%%%%
%% Ваше решение задачи здесь



%%%%%%%%%%%%%%%%%%%%%%%%%%%%%%%%%%%%%%%%%%%%%%%%%%%%
\end{solution} 

%%%%%%%%%%%%%%%%%%%%%%%%%%%%%%%%%%%%%%%%%%%%%%%%%%%%
% Задача 3
\begin{problem}{297}
Вычислить определитель 
$\left| \begin{array}{ccccc}1 & 2 & 3 & \ldots & n \\ 2 & 3 & 4 & \ldots & 1 \\ 3 & 4 & 5 & \ldots & 2 \\ \vdots & \vdots & \vdots & \ddots & \vdots \\ n & 1 & 2 & \ldots & n-1 \end{array} \right|$
\end{problem}
\begin{solution}
%%%%%%%%%%%%%%%%%%%%%%%%%%%%%%%%%%%%%%%%%%%%%%%%%%%%
%% Ваше решение задачи здесь

Вычтем из $i$ строки $i-1$ю \\
$
\left|
\begin{array}{ccccc}
1 & 2 & 3 & \ldots & n \\
2 & 3 & 4 & \ldots & 1 \\
3 & 4 & 5 & \ldots & 2 \\
\vdots & \vdots & \vdots & \ddots & \vdots \\
n & 1 & 2 & \ldots & n-1 
\end{array}
\right|
=
\left|
\begin{array}{cccccc}
1 & 2 & 3 & \ldots & n-1 & n \\
1 & 1 & 1 & \ldots & 1 & 1-n \\
1 & 1 & 1 & \ldots & 1-n & 1 \\
\vdots & \vdots & \vdots & \ddots & \vdots \\
1 & 1-n & 1 & \ldots & 1 & 1
\end{array}
\right|
$

%%%%%%%%%%%%%%%%%%%%%%%%%%%%%%%%%%%%%%%%%%%%%%%%%%%%
\end{solution} 

%%%%%%%%%%%%%%%%%%%%%%%%%%%%%%%%%%%%%%%%%%%%%%%%%%%%
% Задача 4
\begin{problem}{374(b)}
Вычислить определитель $\Delta$ посредством умножения на определитель $\delta$\\
$\Delta=\left| \begin{array}{rrrr} -1 & -9 & -2 & 3\\ -5 & 5 & 3 & -2\\ -12 & -6 & 1 & 1 \\ 9 & 0 & -2 & 1\end{array}\right|$;
$\delta=\left| \begin{array}{rrrr} 1 & 0 & 0 & 0\\ -2 & 1 & 0 & 0\\ 3 & 2 & 1 & 0 \\ -3 & 4 & 2 & 1\end{array}\right|$
\end{problem}
\begin{solution}
%%%%%%%%%%%%%%%%%%%%%%%%%%%%%%%%%%%%%%%%%%%%%%%%%%%%
%% Ваше решение задачи здесь

$
\Delta=
\left|
\begin{array}{rrrr}
-1 & -9 & -2 & 3 \\
-5 & 5 & 3 & -2 \\
-12 & -6 & 1 & 1 \\
9 & 0 & -2 & 1
\end{array}
\right|
$;
$
\delta=
\left|
\begin{array}{rrrr}
1 & 0 & 0 & 0 \\
-2 & 1 & 0 & 0 \\
3 & 2 & 1 & 0 \\
-3 & 4 & 2 & 1
\end{array}
\right|
\\
\Delta * \delta = 
\left|
\left(
\begin{array}{rrrr}
-1 & -9 & -2 & 3 \\
-5 & 5 & 3 & -2 \\
-12 & -6 & 1 & 1 \\
9 & 0 & -2 & 1
\end{array}
\right)
*
\left(
\begin{array}{rrrr}
1 & 0 & 0 & 0 \\
-2 & 1 & 0 & 0 \\
3 & 2 & 1 & 0 \\
-3 & 4 & 2 & 1
\end{array}
\right)
\right|
=
\left|
\left(
\begin{array}{rrrr}
2 & -1 & 4 & 3 \\
0 & 3 & -1 & -2 \\
0 & 0 & 3 & 1 \\
0 & 0 & 0 & 1 
\end{array}
\right)
\right|
= 2 * 3 * 3 * 1 = 18
$

%%%%%%%%%%%%%%%%%%%%%%%%%%%%%%%%%%%%%%%%%%%%%%%%%%%%
\end{solution} 

%%%%%%%%%%%%%%%%%%%%%%%%%%%%%%%%%%%%%%%%%%%%%%%%%%%%
% Задача 5
\begin{problem}{391}
Доказать, что $\det\left(\begin{array}{cc}E_m & B\\C & D\end{array}\right)=\det(D-CB)$.
Здесь $B$ и $C$~-- произвольные $m\times n$- и $n\times m$-матрицы, $D$~-- квадратная матрица порядка $n$.
\end{problem}
\begin{solution}
%%%%%%%%%%%%%%%%%%%%%%%%%%%%%%%%%%%%%%%%%%%%%%%%%%%%
%% Ваше решение задачи здесь

$
\left|
\begin{array}{rr}
  E_m & B \\
  C   & D
\end{array}
\right|
=
\left|
\begin{array}{rr}
  E_m & B \\
  0   & D - CB
\end{array}
\right|
= \det{E_m}*\det{D-CB}
= \det{D-CB}
$

%%%%%%%%%%%%%%%%%%%%%%%%%%%%%%%%%%%%%%%%%%%%%%%%%%%%
\end{solution} 

%------------------------------------------------
\end{document}
