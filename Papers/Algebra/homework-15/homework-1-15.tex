\documentclass[a4paper, 12pt]{article}
%----------------------------------------------------------------------------------------
%	PACKAGES AND OTHER DOCUMENT CONFIGURATIONS
%----------------------------------------------------------------------------------------
\usepackage[a4paper, total={7in, 10in}]{geometry}
\setlength{\parskip}{0pt}
\setlength{\parindent}{0in}

\usepackage[T2A]{fontenc}% Внутренняя T2A кодировка TeX
\usepackage[utf8]{inputenc}% кодировка файла
\usepackage[russian]{babel}% поддержка переносов в русском языке
\usepackage{amsthm, amsmath, amssymb} % Mathematical typesetting
\usepackage{float} % Improved interface for floating objects
\usepackage{graphicx, multicol} % Enhanced support for graphics
\usepackage{xcolor} % Driver-independent color extensions
\usepackage{mdframed}

\usepackage[yyyymmdd]{datetime} % Uses YEAR-MONTH-DAY format for dates
\renewcommand{\dateseparator}{.} % Sets dateseparator to '.'

\usepackage{fancyhdr} % Headers and footers
\pagestyle{fancy} % All pages have headers and footers
\fancyhead{}\renewcommand{\headrulewidth}{0pt} % Blank out the default header
\fancyfoot[L]{} % Custom footer text
\fancyfoot[C]{} % Custom footer text
\fancyfoot[R]{\thepage} % Custom footer text

\newenvironment{problem}[2][Задача]
    { \begin{mdframed}[backgroundcolor=gray!10] \textbf{#1 #2.} \\}
    {  \end{mdframed}}

\newenvironment{solution}
    {\textit{Решение: }}
    {\noindent\rule{7in}{1.5pt}}

\begin{document}

%-------------------------------
%	TITLE SECTION
%-------------------------------

\fancyhead[C]{}
\hrule \medskip % Upper rule
\begin{minipage}{0.295\textwidth}
\raggedright\footnotesize
Цуканов Михаил \hfill\\
st117303 \hfill\\
st117303@student.spbu.ru
\end{minipage}
\begin{minipage}{0.4\textwidth}
\centering\large
Homework Assignment 15\\
\normalsize
Алгебра и геометрия, 1 семестр\\
\end{minipage}
\begin{minipage}{0.295\textwidth}
\raggedleft
\today\hfill\\
\end{minipage}
\medskip\hrule
\bigskip

%------------------------------------------------
%	CONTENTS
%------------------------------------------------



%%%%%%%%%%%%%%%%%%%%%%%%%%%%%%%%%%%%%%%%%%%%%%%%%%%%
% Задача 1
\begin{problem}{371}
Написать уравнение прямой, параллельной прямой ${2x+5y=0}$ и образующей вместе с осями системы координат треугольник,
площадь которого равна 5. Система координат прямоугольная.
\end{problem}
\begin{solution}
%%%%%%%%%%%%%%%%%%%%%%%%%%%%%%%%%%%%%%%%%%%%%%%%%%%%
%% Ваше решение задачи здесь

Возьмем прямую, принадлежащую плоскости $Oxy$ и параллельную данной прямой. Например $2x+5y-10=0$. \\
Эта прямая будет пересекать оси $Ox$ и $Oy$ на расстоянии 5 и 2 соотв. \\
А значит площадь получившегося треугольника 5. \\
Ответ: $2x+5y-10=0$

%%%%%%%%%%%%%%%%%%%%%%%%%%%%%%%%%%%%%%%%%%%%%%%%%%%%
\end{solution}

%%%%%%%%%%%%%%%%%%%%%%%%%%%%%%%%%%%%%%%%%%%%%%%%%%%%
% Задача 2
\begin{problem}{378}
Даны уравнения $3x-2y+1=0$, $x-y+1=0$ двух сторон треугольника и уравнение $2x-y-1=0$ медианы,
выходящей из вершины, не лежащей на первой стороне. Составить уравнение третьей стороны треугольника.
Система координат аффинная.
\end{problem}
\begin{solution}
%%%%%%%%%%%%%%%%%%%%%%%%%%%%%%%%%%%%%%%%%%%%%%%%%%%%
%% Ваше решение задачи здесь

Найдем середину первой стороны: \\
$3x+1=2y \land 2x-1=y <=> \\ 3x+1=y \land 4x-2=2y <=> \\ x = 3 \land y = 5$ \\
Найдем точку, принадлежащую обеим известным сторонам: \\
$3x+1=2y \land x+1=y <=> \\ 3x+1=2y \land 2x+2=2y <=> \\ x = 1 \land y = 2$ \\
Между первой точкой отрезка и его серединой вектор $\{-2, -3\}$.
Значит между второй отчкой и его серединой вектор $\{2, 3\}$.
Значит координаты второй точки: $(5, 8)$. \\
Третья вершина треугольника лежит на пересечении медианы и второй стороны. Найдем ее координаты: \\
$x+1=y \land 2y-1=y <=> x = 2 \land y = 3$ \\
Тгда уравнение последней стороны - уравнение прямой, проходящей через точки $(2, 3)$ и $(5, 8)$: \\
$\frac{5}{3}x-\frac{1}{3}=y$ \\
Ответ:
$5x-3y-1=0$

%%%%%%%%%%%%%%%%%%%%%%%%%%%%%%%%%%%%%%%%%%%%%%%%%%%%
\end{solution}

%%%%%%%%%%%%%%%%%%%%%%%%%%%%%%%%%%%%%%%%%%%%%%%%%%%%
% Задача 3
\begin{problem}{379}
Дано уравнение $x-2y+7=0$ стороны треугольника и уравнения $x+y-5=0$, $2x+y-11=0$ медиан,
выходящих из вершин тругольника, лежащих на данной прямой. Составить уравнение двух других сторон треугольника.
Система координат аффинная.
\end{problem}
\begin{solution}
%%%%%%%%%%%%%%%%%%%%%%%%%%%%%%%%%%%%%%%%%%%%%%%%%%%%
%% Ваше решение задачи здесь

Найдем точки пересечения медиан и данного уравнения - вершины, лежащие на данной стороне. \\
$x-2y+7=0 \land x+y-5=0 <=>
x=2y-7 \land x=5-y <=> \\
x = 1 \land y = 4$ \\ $(1, 4)$ \\
$x-2y+7=0 \land 2x+y-11=0 <=>
x=2y-7 \land 2x=11-y <=>
2x=4y-14 \land 2x=11-y <=> \\
x = 3 \land y = 5$ \\ $(3, 5)$ \\
Найдем точку пересечения медиан: \\
$x+y-5=0 \land 2x+y-11=0 <=> y=5-x \land y = 11-2x <=> x = 6 \land y = -1$ \\ $(6, -1)$ \\
Медианы делятся точкой пересечения в отношении 2 к 1.
Значит если между вершиной и точкой пересечения вектор $\{5, -5\}$, то между точкой пересечения и серединой противоположной стороны $\{2.5, -2.5\}$. \\
Значит если между вершиной и точкой пересечения вектор $\{3, -6\}$, то между точкой пересечения и серединой противоположной стороны $\{1.5, -3\}$. \\
Уравнения неизвестных сторон - уравнения прямых, проходящих через точки $(1, 4)$ и $(7.5, -4)$; $(3, 5)$ и $(8.5, -3.5)$ соответственно. \\
Найдем эти уравнения: \\
$-\frac{8}{6.5}x+\frac{34}{6.5}=y$ \\
$-8x+34=6.5y$ \\
$-16x - 13y +68 = 0$; \\
$-\frac{17}{11}x + \frac{4}{11} = y$ \\
$-17x - 11y + 4 = 0$; \\
Ответ: $-16x - 13y +68 = 0$ и $-17x - 11y + 4 = 0$

%%%%%%%%%%%%%%%%%%%%%%%%%%%%%%%%%%%%%%%%%%%%%%%%%%%%
\end{solution}

%------------------------------------------------
\end{document}
