\documentclass[a4paper, 12pt]{article}
%----------------------------------------------------------------------------------------
%	PACKAGES AND OTHER DOCUMENT CONFIGURATIONS
%----------------------------------------------------------------------------------------
\usepackage[a4paper, total={7in, 10in}]{geometry}
\setlength{\parskip}{0pt}
\setlength{\parindent}{0in}

\usepackage[T2A]{fontenc}% Внутренняя T2A кодировка TeX
\usepackage[utf8]{inputenc}% кодировка файла
\usepackage[russian]{babel}% поддержка переносов в русском языке
\usepackage{amsthm, amsmath, amssymb} % Mathematical typesetting
\usepackage{float} % Improved interface for floating objects
\usepackage{graphicx, multicol} % Enhanced support for graphics
\usepackage{xcolor} % Driver-independent color extensions
\usepackage{mdframed}
\usepackage{polynom}

\usepackage[yyyymmdd]{datetime} % Uses YEAR-MONTH-DAY format for dates
\renewcommand{\dateseparator}{.} % Sets dateseparator to '.'

\usepackage{fancyhdr} % Headers and footers
\pagestyle{fancy} % All pages have headers and footers
\fancyhead{}\renewcommand{\headrulewidth}{0pt} % Blank out the default header
\fancyfoot[L]{} % Custom footer text
\fancyfoot[C]{} % Custom footer text
\fancyfoot[R]{\thepage} % Custom footer text

\newenvironment{problem}[2][Задача]
    { \begin{mdframed}[backgroundcolor=gray!10] \textbf{#1 #2.} \\}
    {  \end{mdframed}}

\newenvironment{solution}
    {\textit{Решение: }}
    {\noindent\rule{7in}{1.5pt}}

\begin{document}

%-------------------------------
%	TITLE SECTION
%-------------------------------

\fancyhead[C]{}
\hrule \medskip % Upper rule
\begin{minipage}{0.295\textwidth}
\raggedright\footnotesize
Михаил Цуканов \hfill\\
st117303 \hfill\\
st117303@student.spbu.ru
\end{minipage}
\begin{minipage}{0.4\textwidth}
\centering\large
Homework Assignment 4\\
\normalsize
Алгебра и геометрия, 1 семестр\\
\end{minipage}
\begin{minipage}{0.295\textwidth}
\raggedleft
\today\hfill\\
\end{minipage}
\medskip\hrule
\bigskip

%------------------------------------------------
%	CONTENTS
%------------------------------------------------

%%%%%%%%%%%%%%%%%%%%%%%%%%%%%%%%%%%%%%%%%%%%%%%%%%%%
% Пример. Найти наибольший общий делитель $f(x)=x^4+3x^3\hm-x^2-4x-3$ и $g(x)=3x^3+10x^2+2x-3$.
%
%$$
%\begin{array}{rrrrrrrrrrl}
%x^4 & + & 3x^3 & - & x^2 & - & 4x & - & 3 & \vline &3x^3 + 10x^2 + 2x - 3 \\ \cline{11-11}
%x^4 & + & \frac{10}{3}x^3 & + & \frac{2}{3}x^2 & - & x & \, & \, & \vline & \frac{1}{3}x - \frac{1}{9} \\ \cline{1-7}
%\, & - & \frac{1}{3}x^3 & - & \frac{5}{3}x^2 & - & 3x & - & 3 & & \\
%\, & - & \frac{1}{3}x^3 & - & \frac{10}{9}x^2 & - & \frac{2}{9}x & + & \frac{1}{3} & & \\ \cline{2-9}
%\, & \,& \, & - & \frac{5}{9}x^2 & - &\frac{25}{9}x & - & \frac{10}{3} & &
%\end{array}
%$$
%Следовательно  $f(x)=\frac{1}{9}g(x)q_1(x)-\frac{5}{9}r_1(x)$, где $q_1(x)=3x-1$, $r_1(x)\hm=x^2+5x+6$.
%
%$g(x)=r_1(x)q_2(x)+9r_2(x)$, где $q_2(x)=3x-5$, $r_2(x)\hm=x+3$.
%
%$r_1(x)=r_2(x)(x+2)$.


%%%%%%%%%%%%%%%%%%%%%%%%%%%%%%%%%%%%%%%%%%%%%%%%%%%%
% Задача 1
\begin{problem}{546(b)}
Выполнить деление с остатком $x^3-3x^2-x-1$ на $3x^2-2x+1$.
\end{problem}
\begin{solution}
%%%%%%%%%%%%%%%%%%%%%%%%%%%%%%%%%%%%%%%%%%%%%%%%%%%%
%% Ваше решение задачи здесь

\polylongdiv[tutor=true, style=D]{x^3-3x^2-x-1}{3x^2-2x+1}

%%%%%%%%%%%%%%%%%%%%%%%%%%%%%%%%%%%%%%%%%%%%%%%%%%%%
\end{solution}

%%%%%%%%%%%%%%%%%%%%%%%%%%%%%%%%%%%%%%%%%%%%%%%%%%%%
% Задача 2
\begin{problem}{549(c)}
Пользуясь схемой Горнера, разложить полином $f(x)=x^5$ по степеням $x-1$.
\end{problem}
\begin{solution}
%%%%%%%%%%%%%%%%%%%%%%%%%%%%%%%%%%%%%%%%%%%%%%%%%%%%
%% Ваше решение задачи здесь

$
\polyhornerscheme[showbase=bottom, showmiddlerow=false, x=1]{x^5} \\
\polyhornerscheme[showbase=bottom, showmiddlerow=false, x=1]{x^4 + x^3 + x^2 + x^1 + 1} \\
\polyhornerscheme[showbase=bottom, showmiddlerow=false, x=1]{x^3 + 2x^2 + 3x^1 + 4} \\
\polyhornerscheme[showbase=bottom, showmiddlerow=false, x=1]{x^2 + 3x^1 + 6} \\
\polyhornerscheme[showbase=bottom, showmiddlerow=false, x=1]{x^1 + 4} \\
\text{Получается: } (x-1)^5 + 5(x-1)^4 + 10(x-1)^3 + 10(x-1)^2 + 5(x-1) + 1
$

%%%%%%%%%%%%%%%%%%%%%%%%%%%%%%%%%%%%%%%%%%%%%%%%%%%%
\end{solution}

%%%%%%%%%%%%%%%%%%%%%%%%%%%%%%%%%%%%%%%%%%%%%%%%%%%%
% Задача 3
\begin{problem}{549(d)}
Выполнить деление с остатком $x^3-x^2-x$ на $x-1+2i$.
\end{problem}
\begin{solution}
%%%%%%%%%%%%%%%%%%%%%%%%%%%%%%%%%%%%%%%%%%%%%%%%%%%%
%% Ваше решение задачи здесь

$
\begin{array}{rrrrrrrrrrrrl}
x^3 & - & x^2 & - & x & \, & \, & \, & \, & \vline &x - 1 + 2i \\ \cline{11-12}
x^3 & - & x^2 & + & 2x^2i & \, & \, & \, & \, & \vline & x^2 - 2xi - 5 - 2i & \\ \cline{1-5}
\, & - & 2x^2i & - & x & &\\
\, & - & 2x^2i & + & 2xi & + & 4x & \\ \cline{3-7}
\, & \,& \, & - & 5x & - & 2xi & \\
\, & \,& \, & - & 5x & + & 5 & - & 10i & \\ \cline{4-9}
\, & \,& \, & - & 2xi & - & 5 & + & 10i & \\
\, & \,& \, & - & 2xi & + & 4 & + & 2i & \\ \cline{5-10}
\, & \, & \, & \, & \, & - & 9 & + & 8i & \\
\end{array}
$

%%%%%%%%%%%%%%%%%%%%%%%%%%%%%%%%%%%%%%%%%%%%%%%%%%%%
\end{solution}

%%%%%%%%%%%%%%%%%%%%%%%%%%%%%%%%%%%%%%%%%%%%%%%%%%%%
% Задача 4
\begin{problem}{557(e)}
Определить наибольший общий делитель полиномов: $x^6+2x^4-4x^3-3x^2+8x-5$ и $x^5+x^2-x+1$.
\end{problem}
\begin{solution}
%%%%%%%%%%%%%%%%%%%%%%%%%%%%%%%%%%%%%%%%%%%%%%%%%%%%
%% Ваше решение задачи здесь

$
\polylonggcd{x^6+2x^4-4x^3-3x^2+8x-5}{x^5+x^2-x+1} \\
$
Итого ответ: $2x-5$


%%%%%%%%%%%%%%%%%%%%%%%%%%%%%%%%%%%%%%%%%%%%%%%%%%%%
\end{solution}

%%%%%%%%%%%%%%%%%%%%%%%%%%%%%%%%%%%%%%%%%%%%%%%%%%%%
% Задача 5
\begin{problem}{557(f)}
Определить наибольший общий делитель полиномов: $x^5+3x^4-12x^3-52x^2-52x-12$ и $x^4+3x^3-6x^2-22x-12$.
\end{problem}
\begin{solution}
%%%%%%%%%%%%%%%%%%%%%%%%%%%%%%%%%%%%%%%%%%%%%%%%%%%%
%% Ваше решение задачи здесь

$
\polylonggcd{x^5+3x^4-12x^3-52x^2-52x-12}{x^4+3x^3-6x^2-22x-12} \\
$

Итого ответ: $x + 1$

%%%%%%%%%%%%%%%%%%%%%%%%%%%%%%%%%%%%%%%%%%%%%%%%%%%%
\end{solution}

%%%%%%%%%%%%%%%%%%%%%%%%%%%%%%%%%%%%%%%%%%%%%%%%%%%%
% Задача 6
\begin{problem}{578(c)}
Пользуясь алгоритмом Евклида, подобрать полиномы $M_1(x)$ и $M_2(x)$ так,
чтобы $f_1(x)M_1(x)+f_2(x)M_2(x)=\delta(x)$, где $\delta(x)$~-- наибольший общий делитель полиномов $f_1(x)$ и $f_2(x)$.\\
$f_1(x)=x^6-4x^5+11x^4-27x^3+37x^2-35x+35$, \\ $f_2(x)=x^5-3x^4+7x^3-20x^2+10x-25$.
\end{problem}
\begin{solution}
%%%%%%%%%%%%%%%%%%%%%%%%%%%%%%%%%%%%%%%%%%%%%%%%%%%%
%% Ваше решение задачи здесь

$\displaystyle x^6-4x^5+11x^4-27x^3+37x^2-35x+35 = (x^5-3x^4+7x^3-20x^2+10x-25)(x - 1) + x^4 + 7x^2 + 10$
\\
$\displaystyle x^5-3x^4+7x^3-20x^2+10x-25 = (x^4 + 7x^2 + 10)(x - 3) + x^2 + 5$
\\
$\displaystyle x^4 + 7x^2 + 10 = (x^2 + 5)(x^2 + 2) \text{ - НОД}$
\\
$\displaystyle f_1 = f_2q_1 + r_1$
\\
$\displaystyle f_2 = r_1q_2 + r_2$
\\
$\displaystyle r_2 = f_2 - r_1q_2 = -f_1q_2 + f_2(1 + q_1q_2)$
\\
$\displaystyle M_1 = -q_2 = 3 - x$
\\
$\displaystyle M_2 = 1 + q_1q_2 = 1 + (x - 1)(x - 3) = x^2 - 4x + 4$
\\
$\displaystyle f_1(x)(3 - x) + f_2(x)(x^2 - 4x + 4) = x^2 + 5$
\\
Ответ: $\displaystyle 3 - x; x^2 - 4x + 4$

%%%%%%%%%%%%%%%%%%%%%%%%%%%%%%%%%%%%%%%%%%%%%%%%%%%%
\end{solution}

%%%%%%%%%%%%%%%%%%%%%%%%%%%%%%%%%%%%%%%%%%%%%%%%%%%%
% Задача 7
\begin{problem}{578(d)}
Пользуясь алгоритмом Евклида, подобрать полиномы $M_1(x)$ и $M_2(x)$ так,
чтобы $f_1(x)M_1(x)+f_2(x)M_2(x)=\delta(x)$, где $\delta(x)$~-- наибольший общий делитель полиномов $f_1(x)$ и $f_2(x)$.\\
$f_1(x)=3x^7+6x^6-3x^5+4x^4+14x^3-6x^2-4x+4$, \\ $f_2(x)=3x^6-3x^4+7x^3-6x+2$.
\end{problem}
\begin{solution}
%%%%%%%%%%%%%%%%%%%%%%%%%%%%%%%%%%%%%%%%%%%%%%%%%%%%
%% Ваше решение задачи здесь

$\displaystyle f_1(x)M_1(x) + f_2(x)M_2(x) = \delta(x)$
\\
$\displaystyle 3x^7+6x^6-3x^5+4x^4+14x^3-6x^2-4x+4 = (3x^6-3x^4+7x^3-6x+2)(x + 2) + 3x^4 + 6x$
\\
$\displaystyle 3x^6-3x^4+7x^3-6x+2 = (3x^4 + 6x)(x^2 - 1) + x^3 + 2$
\\
$\displaystyle 3x^4 + 6x = (x^3 + 2)3x \text{ - НОД}$
\\
$\displaystyle f_1 = f_2q_1 + r_1$
\\
$\displaystyle f_2 = r_1q_2 + r_2$
\\
$\displaystyle r_2 = f_2 - r_1q_2 = -f_1q_2 + f_2(1 + q_1q_2)$
\\
$\displaystyle M_1 = -q_2 = 1 - x^2$
\\
$\displaystyle M_2 = 1 + q_1q_2 = 1 + (x + 2)(x^2 - 1) = x^3 + 2x^2 - x - 1$
\\
$\displaystyle f_1(x)(1 - x^2) + f_2(x)(x^3 + 2x^2 - x - 1) = x^3 + 2$
\\
Ответ: $\displaystyle 1 - x^2; x^3 + 2x^2 - x - 1$

%%%%%%%%%%%%%%%%%%%%%%%%%%%%%%%%%%%%%%%%%%%%%%%%%%%%
\end{solution}

%%%%%%%%%%%%%%%%%%%%%%%%%%%%%%%%%%%%%%%%%%%%%%%%%%%%
% Задача 8
\begin{problem}{583(b)}
Определить полином наименьшей степени, дающий в остатке $x^2+x+1$ при делении на $x^4-2x^3-2x^2+10x-7$ и $2x^2-3$ при делении на $x^4-2x^3-3x^2+13x-10$.
\end{problem}
\begin{solution}
%%%%%%%%%%%%%%%%%%%%%%%%%%%%%%%%%%%%%%%%%%%%%%%%%%%%
%% Ваше решение задачи здесь

$
\left\{
\begin{aligned}
    f &= g_1(x^4-2x^3-2x^2+10x-7) + x^2+x+1,\\
    f &= g_2(x^4-2x^3-3x^2+13x-10) + 2x^2 - 3
\end{aligned}
\right.;
$
\\
$
\left\{
\begin{aligned}
    s_1(x^4-2x^3-2x^2+10x-7)+s_2(x^4-2x^3-3x^2+13x-10) &= 1,\\
    g_1(x^4-2x^3-2x^2+10x-7) - g_2(x^4-2x^3-3x^2+13x-10) &= x^2-x-4
\end{aligned}
\right.
$
\\
$
\left\{
\begin{aligned}
    (x^2-x-4)s_1(x^4-2x^3-2x^2+10x-7)+(x^2-x-4)s_2(x^4-2x^3-3x^2+13x-10) &= x^2-x-4,\\
    g_1(x^4-2x^3-2x^2+10x-7) - g_2(x^4-2x^3-3x^2+13x-10) &= x^2-x-4
\end{aligned}
\right.
$
\\
$\displaystyle (x^4-2x^3-2x^2+10x-7)g_1 - (x^4-2x^3-3x^2+13x-10)g_2 = (x^2-x-4)(s_1(x^4-2x^3-2x^2+10x-7)+s_2(x^4-2x^3-3x^2+13x-10))$
\\
$\displaystyle (x^4-2x^3-2x^2+10x-7)(g_1-(x^2-x-4)s_1) = (x^4-2x^3-3x^2+13x-10)(g_2+(x^2-x-4)s_2)$
\\
$\displaystyle s_1(x^4-2x^3-2x^2+10x-7)+s_2(x^4-2x^3-3x^2+13x-10) = 1$
\\
$\displaystyle x^4-2x^3-2x^2+10x-7 = (x^4-2x^3-3x^2+13x-10) + x^2 -3x + 3$
\\
$\displaystyle x^4-2x^3-3x^2+13x-10 = (x^2 -3x + 3)(x^2+x-3) + x - 1$
\\
$\displaystyle x^2-3x+3 = (x-1)(x-2)+1$
\\
$\displaystyle f_1 = f_2q_1 + r_1 \Rightarrow r_1 = f_1 - f_2q_1$
\\
$\displaystyle f_2 = r_1q_2 + r_2 \Rightarrow r_2 = f_2 - r_1q_2 = f_2 - f_1q_2 + f_2q_1q_1$
\\
$\displaystyle r_1 = r_2q_3 + r_3 \Rightarrow r_3 = r_1 - r_2q_3 = f_1 - f_2q_1 - (f_2 - f_1q_2 + f_2q_1q_2)q_3 = f_1 - f_2q_1 - f_2q_3 + f_1q_2q_3 - f_2q_1q_2q_3 = f_1(1 + q_2q_3) + f_2(-q_1-q_3-q_1q_2q_3)$
\\
$\displaystyle q_1 = 1$
\\
$\displaystyle q_2 = x^2 + x - 3$
\\
$\displaystyle q_3 = x - 2$
\\
$\displaystyle s_1 = 1 + (x^2+x-3)(x-2) = x^3-x^2-5x+7$
\\
$\displaystyle s_2 = -1 - x + 2 - x^3 + x^2 + 5x - 6 = -x^3 + x^2 + 4x - 5$
\\
$\displaystyle (x^4-2x^3-2x^2+10x-7)(g_1-(x^2-x-4)s_1) = (x^4-2x^3-3x^2+13x-10)(g_2+(x^2-x-4)s_2)$
\\
$\left\{
\begin{aligned}
    g_1-(x^2-x-4)s_1 &= \alpha(x^4-2x^3-3x^2+13x-10),\\
    g_2+(x^2-x-4)s_2 &= \alpha(x^4-2x^3-2x^2+10x-7)
\end{aligned}
\right.$
\\
$\displaystyle g_1 = \alpha(x^4-2x^3-3x^2+13x-10) + (x^2-x-4)(x^3-x^2-5x+7) = \alpha x^4 - 2\alpha x^3 - 3\alpha x^2 + 13\alpha x - 10\alpha + x^5 - 2x^4 - 8x^3 + 16x^2 + 13x - 28 = x^5 + (\alpha - 2)x^4 - (2\alpha + 8)x^3 + (16 - 3\alpha)x^2 + (13\alpha + 13)x - (10\alpha + 28)$
\\
$\displaystyle\text{Пусть $\alpha$ = -x, тогда}$
\\
$\displaystyle g_1 = x^5 - x^5 - 2x^4 + 2x^4 - 8x^3 + 16x^2 + 3x^2 - 13x^2 + 13x + 10x - 28 = -5x^3 + 3x^2 + 23x - 28$
\\
$\displaystyle f = g_1(x^4-2x^3-2x^2+10x-7) + x^2 + x + 1 = (-5x^3 + 3x^2 + 23x - 28)(x^4-2x^3-2x^2+10x-7) + x^2 + x + 1 = -5x^7 + 13x^6 + 27x^5 - 130x^4 + 75x^3 + 266x^2 - 440x + 197$
\\
Ответ: $\displaystyle -5x^7 + 13x^6 + 27x^5 - 130x^4 + 75x^3 + 266x^2 - 440x + 197$

%%%%%%%%%%%%%%%%%%%%%%%%%%%%%%%%%%%%%%%%%%%%%%%%%%%%
\end{solution}


%------------------------------------------------
\end{document}
