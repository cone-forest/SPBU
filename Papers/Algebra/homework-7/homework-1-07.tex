\documentclass[a4paper, 12pt]{article}
%----------------------------------------------------------------------------------------
%	PACKAGES AND OTHER DOCUMENT CONFIGURATIONS
%----------------------------------------------------------------------------------------
\usepackage[a4paper, total={7in, 10in}]{geometry}
\setlength{\parskip}{0pt}
\setlength{\parindent}{0in}

\usepackage[T2A]{fontenc}% Внутренняя T2A кодировка TeX
\usepackage[utf8]{inputenc}% кодировка файла
\usepackage[russian]{babel}% поддержка переносов в русском языке
\usepackage{amsthm, amsmath, amssymb} % Mathematical typesetting
\usepackage{float} % Improved interface for floating objects
\usepackage{graphicx, multicol} % Enhanced support for graphics
\usepackage{xcolor} % Driver-independent color extensions
\usepackage{mdframed}

\usepackage[yyyymmdd]{datetime} % Uses YEAR-MONTH-DAY format for dates
\renewcommand{\dateseparator}{.} % Sets dateseparator to '.'

\usepackage{fancyhdr} % Headers and footers
\pagestyle{fancy} % All pages have headers and footers
\fancyhead{}\renewcommand{\headrulewidth}{0pt} % Blank out the default header
\fancyfoot[L]{} % Custom footer text
\fancyfoot[C]{} % Custom footer text
\fancyfoot[R]{\thepage} % Custom footer text

\newenvironment{problem}[2][Задача]
    { \begin{mdframed}[backgroundcolor=gray!10] \textbf{#1 #2.} \\}
    {  \end{mdframed}}

\newenvironment{solution}
    {\textit{Решение: }}
    {\noindent\rule{7in}{1.5pt}}

\begin{document}

%-------------------------------
%	TITLE SECTION
%-------------------------------

\fancyhead[C]{}
\hrule \medskip % Upper rule
\begin{minipage}{0.295\textwidth} 
\raggedright\footnotesize
YOURNAME \hfill\\   
YOURSTUDENTID \hfill\\
YOURMAIL
\end{minipage}
\begin{minipage}{0.4\textwidth} 
\centering\large 
Homework Assignment 7\\ 
\normalsize 
Алгебра и геометрия, 1 семестр\\ 
\end{minipage}
\begin{minipage}{0.295\textwidth} 
\raggedleft
\today\hfill\\
\end{minipage}
\medskip\hrule 
\bigskip

%------------------------------------------------
%	CONTENTS
%------------------------------------------------



%%%%%%%%%%%%%%%%%%%%%%%%%%%%%%%%%%%%%%%%%%%%%%%%%%%%
% Задача 1
\begin{problem}{276}
$\left| \begin{array}{rrrr}1 & 1 & 1 & 1 \\ 1 & 2 & 3 & 4 \\ 1 & 3 & 6 & 10 \\ 1 & 4 & 10 & 20 \end{array} \right|$
\end{problem}
\begin{solution}
%%%%%%%%%%%%%%%%%%%%%%%%%%%%%%%%%%%%%%%%%%%%%%%%%%%%
%% Ваше решение задачи здесь

Воспользуемся методом Гаусса \\
$
\left|
\begin{array}{rrrr}
1 & 1 & 1 & 1 \\
1 & 2 & 3 & 4 \\
1 & 3 & 6 & 10 \\
1 & 4 & 10 & 20
\end{array}
\right|
=
\left|
\begin{array}{rrrr}
1 & 1 & 1 & 1 \\
0 & 1 & 2 & 3 \\
0 & 2 & 5 & 9 \\
0 & 3 & 9 & 19
\end{array}
\right|
=
\left|
\begin{array}{rrrr}
1 & 1 & 1 & 1 \\
0 & 1 & 2 & 3 \\
0 & 0 & 1 & 3 \\
0 & 0 & 3 & 10
\end{array}
\right|
=
\left|
\begin{array}{rrrr}
1 & 1 & 1 & 1 \\
0 & 1 & 2 & 3 \\
0 & 0 & 1 & 3 \\
0 & 0 & 0 & 1
\end{array}
\right|
=
1 * 1 * 1 * 1 = 1
$

%%%%%%%%%%%%%%%%%%%%%%%%%%%%%%%%%%%%%%%%%%%%%%%%%%%%
\end{solution}

%%%%%%%%%%%%%%%%%%%%%%%%%%%%%%%%%%%%%%%%%%%%%%%%%%%%
% Задача 2
\begin{problem}{295}
$\left| \begin{array}{rrrrr}1 & 2 & 2 & \ldots & 2\\ 2 & 2 & 2 & \ldots & 2 \\ 2 & 2 & 3 & \ldots & 2 \\ \vdots & \vdots & \vdots & \ddots & \vdots \\ 2 & 2 & 2 & \ldots & n \end{array} \right|$
\end{problem}
\begin{solution}
%%%%%%%%%%%%%%%%%%%%%%%%%%%%%%%%%%%%%%%%%%%%%%%%%%%%
%% Ваше решение задачи здесь

$
\left|
\begin{array}{rrrrr}
1 & 2 & 2 & \ldots & 2 \\
2 & 2 & 2 & \ldots & 2 \\
2 & 2 & 3 & \ldots & 2 \\
\vdots & \vdots & \vdots & \ddots & \vdots \\
2 & 2 & 2 & \ldots & n
\end{array}
\right|
=
\left|
\begin{array}{rrrrr}
1 & 2 & 2 & \ldots & 2 \\
0 & -2 & -2 & \ldots & -2 \\
0 & -2 & -1 & \ldots & -2 \\
\vdots & \vdots & \vdots & \ddots & \vdots \\
0 & -2 & -2 & \ldots & n - 4
\end{array}
\right|
=
\left|
\begin{array}{rrrrr}
-2 & -2 & \ldots & -2 \\
-2 & -1 & \ldots & -2 \\
\vdots & \vdots & \ddots & \vdots \\
-2 & -2 & \ldots & n - 4
\end{array}
\right|
=
\left|
\begin{array}{rrrrr}
-2 & -2 & \ldots & -2 \\
0 & 1 & \ldots & 0 \\
\vdots & \vdots & \ddots & \vdots \\
0 & 0 & \ldots & n - 2
\end{array}
\right|
= \\
= -2
\left|
\begin{array}{rrrrr}
1 & \ldots & -2 \\
\vdots & \vdots & \ddots & \vdots \\
0 & \ldots & n - 2
\end{array}
\right|
= -2(n-2)!
$

%%%%%%%%%%%%%%%%%%%%%%%%%%%%%%%%%%%%%%%%%%%%%%%%%%%%
\end{solution}

%%%%%%%%%%%%%%%%%%%%%%%%%%%%%%%%%%%%%%%%%%%%%%%%%%%%
% Задача 3
\begin{problem}{280}
$\left| \begin{array}{rrrrr}2 & 1 & 1 & 1 & 1 \\ 1 & 3 & 1 & 1 & 1 \\ 1 & 1 & 4 & 1 & 1 \\ 1 & 1 & 1 & 5 & 1 \\ 1 & 1 & 1 & 1 & 6 \end{array} \right|$
\end{problem}
\begin{solution}
%%%%%%%%%%%%%%%%%%%%%%%%%%%%%%%%%%%%%%%%%%%%%%%%%%%%
%% Ваше решение задачи здесь

$
\left|
\begin{array}{rrrrr}
2 & 1 & 1 & 1 & 1 \\
1 & 3 & 1 & 1 & 1 \\
1 & 1 & 4 & 1 & 1 \\
1 & 1 & 1 & 5 & 1 \\
1 & 1 & 1 & 1 & 6
\end{array}
\right|
=
\left|
\begin{array}{rrrrr}
2 & 1 & 1 & 1 & 1 \\
-1 & 2 & 0 & 0 & 0 \\
-1 & 0 & 3 & 0 & 0 \\
-1 & 0 & 0 & 4 & 0 \\
-1 & 0 & 0 & 0 & 5
\end{array}
\right|
= \\
$
Выразим определитель через перестановки: \\
Возьмем i элемент первой строчки. \\
i=1 => Произведение = 2*5! = 240 \\
Иначе произведение = $\frac{5!}{i}$ (-1 от четности перестановок, тк кол-во обращений = 1) \\
Сложим произведения:
$
240(2 + \frac{1}{2} + \frac{1}{3} + \frac{1}{4} + \frac{1}{5}) =
240(2 + frac{25 + 50}{60}) = 788\\
$ \\

%%%%%%%%%%%%%%%%%%%%%%%%%%%%%%%%%%%%%%%%%%%%%%%%%%%%
\end{solution}

%%%%%%%%%%%%%%%%%%%%%%%%%%%%%%%%%%%%%%%%%%%%%%%%%%%%
% Задача 4
\begin{problem}{306}
$\left| \begin{array}{cccccc}\alpha & \alpha\beta & 0 & \ldots & 0 & 0 \\ 1 & \alpha+\beta & \alpha\beta & \ldots & 0 &  0 \\ 0 & 1 & \alpha+\beta & \ldots & 0 & 0 \\ \ldots & \ldots & \ldots & \ldots & \ldots & \ldots \\ 0 & 0 & 0 & \ldots & 1 & \alpha+\beta \end{array} \right|$
\end{problem}
\begin{solution}
%%%%%%%%%%%%%%%%%%%%%%%%%%%%%%%%%%%%%%%%%%%%%%%%%%%%
%% Ваше решение задачи здесь

% $
% \left|
% \begin{array}{cccccc}
% \alpha & \alpha\beta & 0 & \ldots & 0 & 0 \\
% 1 & \alpha+\beta & \alpha\beta & \ldots & 0 &  0 \\
% 0 & 1 & \alpha+\beta & \ldots & 0 & 0 \\
% \ldots & \ldots & \ldots & \ldots & \ldots & \ldots \\
% 0 & 0 & 0 & \ldots & 1 & \alpha+\beta
% \end{array}
% \right|
% $

Будем домножать i строчку на $\alpha^{i - 1}$
тогда можно будет вычесть из i строчки i - 1 \\
При этом нужно делить выражение на $\alpha^{i - 1}$ \\
Тогда получим: \\
$
\alpha^{\frac{-n(n - 1)}{2}}
\left|
\begin{array}{cccccc}
\alpha & \alpha\beta & 0 & \ldots & 0 & 0 \\
0 & \alpha^2 & \alpha^2\beta & \ldots & 0 &  0 \\
0 & 0 & \alpha^3 & \ldots & 0 & 0 \\
\ldots & \ldots & \ldots & \ldots & \ldots & \ldots \\
0 & 0 & 0 & \ldots & 0 & \alpha^n
\end{array}
\right|
=
\alpha^{\frac{-n(n - 1)}{2}} *
\alpha^{\frac{n(n + 1)}{2}}
=
\alpha^{n}
$

%%%%%%%%%%%%%%%%%%%%%%%%%%%%%%%%%%%%%%%%%%%%%%%%%%%%
\end{solution}


%------------------------------------------------
\end{document}
