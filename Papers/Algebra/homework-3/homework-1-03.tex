\documentclass[a4paper, 12pt]{article}
%----------------------------------------------------------------------------------------
%	PACKAGES AND OTHER DOCUMENT CONFIGURATIONS
%----------------------------------------------------------------------------------------
\usepackage[a4paper, total={7in, 10in}]{geometry}
\setlength{\parskip}{0pt}
\setlength{\parindent}{0in}

\usepackage[T2A]{fontenc}% Внутренняя T2A кодировка TeX
\usepackage[utf8]{inputenc}% кодировка файла
\usepackage[russian]{babel}% поддержка переносов в русском языке
\usepackage{amsthm, amsmath, amssymb} % Mathematical typesetting
\usepackage{float} % Improved interface for floating objects
\usepackage{graphicx, multicol} % Enhanced support for graphics
\usepackage{xcolor} % Driver-independent color extensions
\usepackage{mdframed}
\usepackage{polynom}

\usepackage[yyyymmdd]{datetime} % Uses YEAR-MONTH-DAY format for dates
\renewcommand{\dateseparator}{.} % Sets dateseparator to '.'

\usepackage{fancyhdr} % Headers and footers
\pagestyle{fancy} % All pages have headers and footers
\fancyhead{}\renewcommand{\headrulewidth}{0pt} % Blank out the default header
\fancyfoot[L]{} % Custom footer text
\fancyfoot[C]{} % Custom footer text
\fancyfoot[R]{\thepage} % Custom footer text

\newenvironment{problem}[2][Задача]
    { \begin{mdframed}[backgroundcolor=gray!10] \textbf{#1 #2.} \\}
    {  \end{mdframed}}

\newenvironment{solution}
    {\textit{Решение: }}
    {\noindent\rule{7in}{1.5pt}}

\begin{document}

%-------------------------------
%	TITLE SECTION
%-------------------------------

\fancyhead[C]{}
\hrule \medskip % Upper rule
\begin{minipage}{0.295\textwidth}
\raggedright\footnotesize
Цуканов Михаил \hfill\\
st117303 \hfill\\
st117303@student.spbu.ru
\end{minipage}
\begin{minipage}{0.4\textwidth}
\centering\large
Homework Assignment 3\\
\normalsize
Алгебра и геометрия, 1 семестр\\
\end{minipage}
\begin{minipage}{0.295\textwidth}
\raggedleft
\today\hfill\\
\end{minipage}
\medskip\hrule
\bigskip

%------------------------------------------------
%	CONTENTS
%------------------------------------------------

%{\bf Пример.}
%
%Разложить многочлен $f(x)=x^5-5x^4+7x^3-2x^2+4x-8$ по степеням $(x-2)$
%
%По схеме Горнера:
%
%\noindent
%\begin{tabular}{cccccccc}
%& \vline & $1$ & $-5$ & $7$ & $-2$ & $4$ & $-8$ \\
%\hline
%$2$& \vline & $1$ & $-3$ &  $1$ &  $0$ & $4$ & $0$ \\
%   & \vline & $1$ & $-1$ & $-1$ & $-2$ & $0$ &  \\
%   & \vline & $1$ &  $1$ &  $1$ &  $0$ &  &  \\
%   & \vline & $1$ &  $3$ &  $7$ &   &  &  \\
%   & \vline & $1$ &  $5$ &   &   &  &  \\
%   & \vline & $1$ &   &   &   &  &  \\
%\end{tabular}
%
%Следовательно $f(x)=7(x-2)^3+5(x-2)^4+(x-2)^5$.


%%%%%%%%%%%%%%%%%%%%%%%%%%%%%%%%%%%%%%%%%%%%%%%%%%%%
% Задача 1
\begin{problem}{550(b)}
Пользуясь схемой Горнера, вычислить $f(x_0)$ $f(x)=x^5+(1+2i)x^4-(1+3i)x^2+7$, $x_0=-2-i$.
\end{problem}
\begin{solution}
%%%%%%%%%%%%%%%%%%%%%%%%%%%%%%%%%%%%%%%%%%%%%%%%%%%%
%% Ваше решение задачи здесь

$
\begin{array}{c|cccccc}
  -2-i & 1 & 1+2i &   0 & -1-3i & 0 & 7 \\
  -2-i & 1 & -1+i & 3-i & -8-4i & 12+16i & -1-44i \\
\end{array}
$
Ответ: $-1-44i$

%%%%%%%%%%%%%%%%%%%%%%%%%%%%%%%%%%%%%%%%%%%%%%%%%%%%
\end{solution}

%%%%%%%%%%%%%%%%%%%%%%%%%%%%%%%%%%%%%%%%%%%%%%%%%%%%
% Задача 2
\begin{problem}{551(b)}
Пользуясь схемой Горнера, разложить полином $f(x)=x^5$ по степеням $x-1$.
\end{problem}
\begin{solution}
%%%%%%%%%%%%%%%%%%%%%%%%%%%%%%%%%%%%%%%%%%%%%%%%%%%%
%% Ваше решение задачи здесь

$
\polyhornerscheme[showbase=bottom, showmiddlerow=false, x=1]{x^5} \\
\polyhornerscheme[showbase=bottom, showmiddlerow=false, x=1]{x^4 + x^3 + x^2 + x^1 + 1} \\
\polyhornerscheme[showbase=bottom, showmiddlerow=false, x=1]{x^3 + 2x^2 + 3x^1 + 4} \\
\polyhornerscheme[showbase=bottom, showmiddlerow=false, x=1]{x^2 + 3x^1 + 6} \\
\polyhornerscheme[showbase=bottom, showmiddlerow=false, x=1]{x^1 + 4} \\
$
Значит $x^5 = (x - 1)^5 + 5(x-1)^4 + 10(x-1)^3 + 10(x-1)^2 + 5(x-1) + 1$

%%%%%%%%%%%%%%%%%%%%%%%%%%%%%%%%%%%%%%%%%%%%%%%%%%%%
\end{solution}

%%%%%%%%%%%%%%%%%%%%%%%%%%%%%%%%%%%%%%%%%%%%%%%%%%%%
% Задача 3
\begin{problem}{553(b)}
Разложить $f(x)=(x-2)^4+4(x-2)^3+6(x-2)^2+10(x-2)+20$ по степеням $x$.
\end{problem}
\begin{solution}
%%%%%%%%%%%%%%%%%%%%%%%%%%%%%%%%%%%%%%%%%%%%%%%%%%%%
%% Ваше решение задачи здесь

$
y = x - 2 \rightarrow x = y + 2 \\
\polyhornerscheme[showbase=bottom, showmiddlerow=false, x=-2]{x^4+4x^3+6x^2+10x+20} \\
\polyhornerscheme[showbase=bottom, showmiddlerow=false, x=-2]{x^3+2x^2+2x^1+6} \\
\polyhornerscheme[showbase=bottom, showmiddlerow=false, x=-2]{x^2+2} \\
\polyhornerscheme[showbase=bottom, showmiddlerow=false, x=-2]{x-2} \\
$
Значит $(x-2)^4+4(x-2)^3+6(x-2)^2+10(x-2)+20 = x^4-4x^3+6x^2+2x+8$

%%%%%%%%%%%%%%%%%%%%%%%%%%%%%%%%%%%%%%%%%%%%%%%%%%%%
\end{solution}

%%%%%%%%%%%%%%%%%%%%%%%%%%%%%%%%%%%%%%%%%%%%%%%%%%%%
% Задача 4
\begin{problem}{555(b)}
Чему равен показатель кратности корня $-2$ для полинома $x^5+7x^4+16x^3+8x^2-16x-16$?
\end{problem}
\begin{solution}
%%%%%%%%%%%%%%%%%%%%%%%%%%%%%%%%%%%%%%%%%%%%%%%%%%%%
%% Ваше решение задачи здесь

$
\polyhornerscheme[showbase=bottom, showmiddlerow=false, x=-2]{x^5+7x^4+16x^3+8x^2-16x-16} \\
\polyhornerscheme[showbase=bottom, showmiddlerow=false, x=-2]{x^4+5x^3+6x^2-4x^1-8} \\
\polyhornerscheme[showbase=bottom, showmiddlerow=false, x=-2]{x^3+3x^2-4} \\
\polyhornerscheme[showbase=bottom, showmiddlerow=false, x=-2]{x^2+x^1-2} \\
$
Остался полином $x-1$ значит кратность корня - 4

%%%%%%%%%%%%%%%%%%%%%%%%%%%%%%%%%%%%%%%%%%%%%%%%%%%%
\end{solution}

%%%%%%%%%%%%%%%%%%%%%%%%%%%%%%%%%%%%%%%%%%%%%%%%%%%%
% Задача 5
\begin{problem}{559(b)}
Доказать, что полином $f(x)=x^{2n+1}-(2n+1)x^{n+1}+(2n+1)x^{n}-1$ имеет число $1$ тройным корнем.
\end{problem}
\begin{solution}
%%%%%%%%%%%%%%%%%%%%%%%%%%%%%%%%%%%%%%%%%%%%%%%%%%%%
%% Ваше решение задачи здесь

$
\begin{array}{c|cccccc}
  1 & 1 & 0...0 (\text{n-1 нулей}) & -(2n+1) & 2n+1 & 0...0 (\text{n-1 нулей}) & -1 \\
  1 & 1.. & ..1 (\text{n единиц})  & -2n     &  1.. & ..1 (\text{n единиц})    & 0 \\
  1 & 1, 2, 3.. & ..n              &  -n     &  -n+1.. & ..-3,-2,-1 & 0 \\
  1 & 1, 3, 6.. & ..\frac{n(n+1)}{2} & \frac{n(n-1)}{2} & \frac{(n-2)(n-1)}{2},\frac{(n-2)(n-3)}{2}.. & ..\frac{2 * 3}{2},\frac{2*1}{2} & 0 \\
1 & 1, 5, 15.. & ..\frac{n(n+1)(n+2)(n+3)}{24} & \text{Дальше получается} & \text{ пос-ть полиномов с } & \text{растущими} & \text{ коэф-ми}
\end{array}
$ \\
(Формулы были найдены с помощью интерполяционного полинома) \\
Значит 1 - тройной корень. \\

%%%%%%%%%%%%%%%%%%%%%%%%%%%%%%%%%%%%%%%%%%%%%%%%%%%%
\end{solution}

%%%%%%%%%%%%%%%%%%%%%%%%%%%%%%%%%%%%%%%%%%%%%%%%%%%%
% Задача 6
\begin{problem}{569}
Доказать, что полином делится на свою производную в том и только том случае, когда он равен $a_0(x-x_0)^n$.
\end{problem}
\begin{solution}
%%%%%%%%%%%%%%%%%%%%%%%%%%%%%%%%%%%%%%%%%%%%%%%%%%%%
%% Ваше решение задачи здесь



%%%%%%%%%%%%%%%%%%%%%%%%%%%%%%%%%%%%%%%%%%%%%%%%%%%%
\end{solution}

%%%%%%%%%%%%%%%%%%%%%%%%%%%%%%%%%%%%%%%%%%%%%%%%%%%%
% Задача 7
\begin{problem}{570}
Доказать, что полином $f(x)=1+\frac{x}{1}+\frac{x^2}{1\cdot 2}+\ldots +\frac{x^n}{n!}$ не имеет кратных корней.
\end{problem}
\begin{solution}
%%%%%%%%%%%%%%%%%%%%%%%%%%%%%%%%%%%%%%%%%%%%%%%%%%%%
%% Ваше решение задачи здесь



%%%%%%%%%%%%%%%%%%%%%%%%%%%%%%%%%%%%%%%%%%%%%%%%%%%%
\end{solution}

%%%%%%%%%%%%%%%%%%%%%%%%%%%%%%%%%%%%%%%%%%%%%%%%%%%%
% Задача 8
\begin{problem}{114}
Решить уравнения и левые части их разложить на множители с вещественными коэффициентами:\\
a) $x^4 + 6x^3 + 9x^2 + 100 = 0$,\\
b) $x^4 + 2x^2 - 24x + 72 = 0$.
\end{problem}
\begin{solution}
%%%%%%%%%%%%%%%%%%%%%%%%%%%%%%%%%%%%%%%%%%%%%%%%%%%%
%% Ваше решение задачи здесь

$
a) \\
x^4 + 6x^3 + 9x^2 + 100 =
(x^2 + ax + b)(x^2 + cx + d) = \\
x^4 + (c + a)x^3 + (b + d + ac)x^2 + (ad + cb)x + bd \leftrightarrow \\
\left\{
  \begin{array}{rrr}
    c + a = 6 \\
    b + d + ac = 9 \\
    ad + cb = 0 \\
    db = 100 \\
  \end{array}
\right.
\left\{
  \begin{array}{rrr}
    a = -2 \\
    b = 5 \\
    c = 8 \\
    d = 20 \\
  \end{array}
\right. \\
x^4 + 6x^3 + 9x^2 + 100 = (x^2-2x+5)(x^2+8x+20) \\
\text{Проверим, имеют ли эти полиномы вещественные корни} \\
1) D = 4-20 < 0 \rightarrow \text{нет} \\
2) D = 64-100 < 0 \rightarrow \text{нет}
$
\\
\\
$
b) \\
x^4 + 2x^2 - 24x + 72 = (x^2 + ax + b)(x^2 + cx + d) \\
\left\{
  \begin{array}{rrr}
    c + a = 0 \\
    b + d + ac = 2 \\
    ad + cb = -24 \\
    db = 72 \\
  \end{array}
\right.
\left\{
  \begin{array}{rrr}
    a = -4 \\
    b = 6 \\
    c = 4 \\
    d = 12 \\
  \end{array}
\right. \\
x^4 + 2x^2 - 24x + 72 = (x^2 - 4x + 6)(x^2 + 4x + 12) \\
\text{Провери, имеют ли эти полиномы вещественные корни} \\
1) D = 16-24 < 0 \rightarrow \text{нет} \\
2) D = 16-48 < 0 \rightarrow \text{нет}
$


%%%%%%%%%%%%%%%%%%%%%%%%%%%%%%%%%%%%%%%%%%%%%%%%%%%%
\end{solution}

%%%%%%%%%%%%%%%%%%%%%%%%%%%%%%%%%%%%%%%%%%%%%%%%%%%%
% Задача 9
\begin{problem}{2}
Разложить на неприводимые вещественные множители полиномы:\\
c)~$x^4+4x^3+4x^2+1$,\\
d)~$x^6+27$.\\
\end{problem}
\begin{solution}
%%%%%%%%%%%%%%%%%%%%%%%%%%%%%%%%%%%%%%%%%%%%%%%%%%%%
%% Ваше решение задачи здесь

$
c) \\
x^4+4x^3+4x^2+1 = (x^2 + ax + b)(x^2 + cx + d) \\
\left\{
  \begin{array}{rrr}
    c + a = 4 \\
    b + d + ac = 4 \\
    ad + cb = 0 \\
    db = 1 \\
  \end{array}
\right.
\left\{
  \begin{array}{rrr}
    a = 2 \\
    b = -i \\
    c = 2 \\
    d = i \\
  \end{array}
\right.\\
x^4+4x^3+4x^2+1 = (x^2 + 2x - i)(x^2 + 2x + i) \\
$
\\
\\
$
d) \\
x^6 + 27 \\
\text{Нужно найти корни} \\
x^6 = -27 \leftrightarrow x^6 = (27, \pi) \leftrightarrow
x = (27, \frac{\pi}{2} + \frac{\pi k}{3}), k \in \{0, 1, 2, 3, 4, 5\} \\
\text{В получившемся выражении из 6 множителей перемножаем множители с сопряженными корнями} \\
\text{Получается: } (x^2 - 3x + 1)(x^2 + 3)(x^2 + 3x + 3)
$

%%%%%%%%%%%%%%%%%%%%%%%%%%%%%%%%%%%%%%%%%%%%%%%%%%%%
\end{solution}

%------------------------------------------------
\end{document}
