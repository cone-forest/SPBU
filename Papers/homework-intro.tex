\documentclass[a4paper, 12pt]{article}
%----------------------------------------------------------------------------------------
%	PACKAGES AND OTHER DOCUMENT CONFIGURATIONS
%----------------------------------------------------------------------------------------
\usepackage[a4paper, total={7in, 10in}]{geometry}
\setlength{\parskip}{0pt}
\setlength{\parindent}{0in}

\usepackage[T2A]{fontenc}% Внутренняя T2A кодировка TeX
\usepackage[utf8]{inputenc}% кодировка файла
\usepackage[russian]{babel}% поддержка переносов в русском языке
\usepackage{amsthm, amsmath, amssymb} % Mathematical typesetting
\usepackage{float} % Improved interface for floating objects
\usepackage{graphicx, multicol} % Enhanced support for graphics
\usepackage{xcolor} % Driver-independent color extensions
\usepackage{mdframed}

\usepackage[yyyymmdd]{datetime} % Uses YEAR-MONTH-DAY format for dates
\renewcommand{\dateseparator}{.} % Sets dateseparator to '.'

\usepackage{fancyhdr} % Headers and footers
\pagestyle{fancy} % All pages have headers and footers
\fancyhead{}\renewcommand{\headrulewidth}{0pt} % Blank out the default header
\fancyfoot[L]{} % Custom footer text
\fancyfoot[C]{} % Custom footer text
\fancyfoot[R]{\thepage} % Custom footer text

% команды для определения списка литературы.
\renewenvironment{thebibliography}{\vskip12pt\noindent{\centerline{{\bf ЛИТЕРАТУРА}}}%
\setlength{\leftmargin}{-30pt}
  \begin{enumerate}\parskip=-2pt%
} { %
  \end{enumerate}%
}%
% элементы списка
\renewcommand{\bibitem}[2]{\item\label{#1}{{\it #2 }}}%

\newcommand{\TASKS}{\vskip12pt{\noindent\bf{ЗАДАЧИ}}\ }%
\newcommand{\HOMETASKS}{\vskip16pt{\noindent\bf{ДОМАШНЕЕ ЗАДАНИЕ}}\ }%
\newcounter{task}%
 \newenvironment{task}[1]{\vskip0pt\noindent\refstepcounter{task}{\bf\thetask\ (#1).}\ }{\smallbreak}%
% \newenvironment{task}[1]{\vskip12pt\noindent\refstepcounter{task}{{{\bf\thetask.}}}\ }{\smallbreak}%
 \newenvironment{task*}{\vskip12pt\noindent\refstepcounter{task}\ }{\smallbreak}%
 \renewcommand{\thetask}{\arabic{task}}%

\newcommand{\answer}{\noindent{\bf Ответ:}\ }%

\newenvironment{solution}{\noindent{\bf Решение.\ }}%{\smallbreak}%

\begin{document}

%-------------------------------
%	TITLE SECTION
%-------------------------------

\fancyhead[C]{}
\hrule \medskip % Upper rule
\begin{minipage}{0.295\textwidth} 
\raggedright\footnotesize
YOURNAME \hfill\\   
YOURSTUDENTID \hfill\\
YOURMAIL
\end{minipage}
\begin{minipage}{0.4\textwidth} 
\centering\large 
ЗАНЯТИЕ 1\\ 
\normalsize 
Алгебра и геометрия, 1 семестр\\ 
\end{minipage}
\begin{minipage}{0.295\textwidth} 
\raggedleft
\today\hfill\\
\end{minipage}
\medskip\hrule 
\bigskip

%------------------------------------------------
%	CONTENTS
%------------------------------------------------

\section{Действия над комплексными числами}
Для того чтобы решить уравнение ${x^2+1=0}$ действительных чисел недостаточно.
В качестве материала, из которого будет строиться новая система чисел,
возьмем точки на плоскости. Пусть на плоскости выбрана прямоугольная система координат.
Рассмотрим две точки ${\alpha=(a,b)}$ и ${\beta=(c,d)}$.

Суммой будем называть: $(a,b)+(c,d)=(a+c,b+d)$.

Произведением будем называть: $(a,b)\cdot(c,d)=(ac-bd, ad+bc)$.

Рассмотрим точку $(0,1)$. По введенному свойству произведения: $(0,1)\cdot(0,1)=(-1,0)$. 
Обозначим точку $(0,1)$ через {\flqq}$i${\frqq}, тогда $i^2=-1$.

Таким образом $(a,b)=(a,0)+(0,b)=a+ib$.

Свойства комплексных чисел:
\begin{enumerate}
 \item $(a+ib)+(c+id)=(a+c)+i(b+d)$;
 \item $(a+ib)-(c+id)=(a-c)+i(b-d)$;
 \item $(a+ib)\cdot(c+id)=(ac-bd)+i(ad+bc)$;
 \item $\displaystyle\frac{(a+ib)}{(c+id)}=\displaystyle\frac{(ac+bd)}{c^2+d^2}+i\displaystyle\frac{(bc-ad)}{c^2+d^2}$.
\end{enumerate}

Для комплексного числа $\alpha=a+ib$ число $\overline{\alpha}=a-ib$ называется {\it сопряженным} 
(геометрически~-- точки симметричные относительно действительной оси).

\TASKS

\begin{task*} {101.}
Вычислить $(2+3i)(4-5i)+(2-3i)(4+5i)$.
\end{task*}
\answer  46.

\begin{task*} {105(a).}
Вычислить $(1+2i)^6$.
\end{task*}
\answer  $117+44i$.

\begin{task*} {105(b).}
Вычислить $(2+i)^7+(2-i)^7$.
\end{task*}
\answer  $-556$.

\begin{task*} {107(c).}
Вычислить $\displaystyle\frac{(1+2i)^2-(2-i)^3}{(1-i)^3+(2+i)^2}$.
\end{task*}
\answer  $5+5i$.

\begin{task*} {103.}
$(1+2i)x+(3-5i)y=1-3i$. Найти $x$ и $y$, считая их вещественными.
\end{task*}
\answer  $x=-\frac{4}{11}$, $y=\frac{5}{11}$.

\begin{task*} {108(a).}
Решить систему уравнений
$$
\left\{ 
\begin{aligned}
  (3-i)x+(4+2i)y &=2+6i,\\
  (4+2i)x-(2+3i)y &=5+4i.
\end{aligned}
\right.
$$
\end{task*}
\answer  $x=1+i$, $y=i$.

\begin{task*} {112(a,h).}
Вычислить: a) $\sqrt{2i}$, h) $\sqrt{1-i\sqrt{3}}$.
\end{task*}
\answer  a) $\pm(1+i)$; h) $\pm\left( \sqrt{\frac{3}{2}}-i\sqrt{\frac{1}{2}}\right)$.


\begin{task*} {113(a).}
Решить уравнение $x^2-(2+i)x+(-1+7i)=0$.
\end{task*}
\answer $x_1=3-i$, $x_2=-1+2i$.

\HOMETASKS

\begin{task*} {102.}
Вычислить $(x-1-i)(x-1+i)(x+1+i)(x+1-i)$.
\end{task*}

\begin{task*} {105(c).}
Вычислить $(1+2i)^5-(1-2i)^5$.
\end{task*}

\begin{task*} {107(d).}
Вычислить $\displaystyle\frac{(1-i)^5-1}{(1+i)^5+1}$.
\end{task*}

\begin{task*} {108(b).}
Решить систему уравнений
$$
\left\{ 
\begin{aligned}
  (2+i)x+(2-i)y &=6,\\
  (3+2i)x+(3-2i)y &=8.
\end{aligned}
\right.
$$
\end{task*}

\begin{task*} {112(b,g).}
Вычислить: b) $\sqrt{-8i}$, g) $\sqrt{2-3i}$.
\end{task*}

\begin{task*} {113(b).}
Решить уравнение $x^2-(3-2i)x+(5-5i)=0$.
\end{task*}

\begin{thebibliography}
\bibitem{Fadeev}{Фаддеев Д. К., Соминский И. С.} {Задачи по высшей алгебре.--- СПб.: Изд-во Лань, 2004.--- 288 с.}
\end{thebibliography}

%------------------------------------------------
\end{document}
